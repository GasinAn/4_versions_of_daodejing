\documentclass[a4paper]{ctexart}
\usepackage{geometry}
\geometry{a5paper}
\title{通假字集录}
\author{安梯西登}
\date{}
\begin{document}
    \maketitle

    妙:眇

    有:又

    徼:噭

    謂:胃 √

    惡:亞 √

    斯:訾 √

    音:意 √

    隨:隋 √

    聖:聲、𦔻 √

    作:昔 √

    恃:志、侍、寺

    智:知

    淵:潚 √

    似:始 √

    挫:銼

    銳:兑

    紛:芬

    姓:省、生 √

    屈:淈、詘

    動:蹱、勭、重、僮 √

    愈:俞 √

    谷:浴

    勤:堇

    退:芮 √

    爭:静、諍

    持:㨁、√

    揣:𢵦

    保:葆、󱀘

    驕:䮦 √

    遂:述 √

    魄:柏

    離:离、雞

    專:槫

    致:至

    滌:脩

    覽:藍、監

    治:栝、之、知

    埏:𡑋 √

    盲:明 √

    獵:臘 √

    貨:𧷴、𦠽 √

    妨:方、仿 √

    爽:𠷹 √

    彼:罷、皮、被 √

    寵:龍、弄 √

    患:梡 √

    何:苛 √

    託:𨒙、橐 √

    微:󱁚 √

    名:命

    詰:計

    混:𡇯、𦀲、湷 √

    皦:攸、謬

    昧:忽

    惚:沕、忽 √
    
    恍:望、朢 √

    識:志

    豫:與 √

    鄰:𠳵、󱁉 √

    釋:澤

    敦:渾、沌

    曠:𣼥 √

    靜:情、清、靚

    安:女

    徐:余

    蔽:𧝬

    篤:督

    並:旁

    妄:㠵、芒

    没:沕
    
    殆:怠

    侮:母 √

    孝:畜 √

    慈:兹 √

    昏:𨴽、󱁝 √

    哉:才 √

    熙:巸 √

    泊:博

    兆:佻、垗

    餘:余

    沌:惷、湷 √

    俗:鬻 √

    察:蔡 √

    悶:𨴽、󱁅 √

    頑:䦎

    窈:幼 √

    冥:鳴 √

    枉:汪 √

    敝:𧝬 √

    是:視

    彰:章

    豈:幾

    飄:𠠕 √

    終:冬、衆

    況:兄 √

    企:炊

    餘:𥺌 √

    寂:繡、蕭 √

    寥:繆、漻 √

    改:󱁡

    逝:筮

    輕:巠 √

    躁:趮 √

    輜:甾 √

    雖:唯

    觀:官

    超:昭

    轍:勶、達 √

    籌:檮 √

    策:𥮥、󱁦 √

    關:󱁥 √

    救:㤹

    襲:𢘽、曳 √

    資:齎 √

    忒:貣

    樸:楃、握 √

    吹:䂳

    奢:楮、諸 √

    泰:大

    棘:朸 √

    早:蚤 √

    恬:銛

    淡:襲、𢤱

    偏:便 √

    哀:依 √

    泣:立 ×

    勝:朕 √

    露:洛 √

    譬:俾、卑

    亡:忘 √

    氾:渢 √

    客:格 √

    淡:談 √

    固:古 √

    柔:󱁩 √

    脱:説 √

    示:視 √

    化:𢡺

    鎮:闐

    欲:辱

    應:𤻮 √

    扔:乃 √

    後:后、句 √

    薄:泊 √

    靈:霝 √

    裂:蓮 √

    竭:渴 √

    蹶:欮 √

    穀:𥞤 √

    琭:禄 √

    珞:硌 √

    昧:費

    隅:禺

    形:刑

    沖:中

    物:勿

    損:𢿃、云 √

    亦:夕 √

    我:議 √

    梁:良 √

    至:致 √

    吾:五 √

    窮:𡩫 √

    清:請

    禍:𢢸 √

    闚:規、𧠹 √

    彌:𢑃、󱁁 √

    攝:執

    甲:革

    投:椯 √

    措:昔 √

    爪:蚤 √

    覆:復

    既:𢟪 √

    殆:佁

    兑:󱁂、㙂

    濟:齊

    救:棘

    殃:央

    徑:解、𠎿

    豐:夆

    蜂:逢
    
    蠆:𢔯、癘
    
    虺:𧍥、䖝
    
    蛇:地
    
    螫:赫

    猛:孟

    攫:㩴、據

    搏:捕

    嗄:𢖻

    賤:淺 √

    奇:畸

    滋:兹

    政:正

    迷:𢘻

    廉:兼

    肆:紲

    燿:眺

    根:󱁆

    柢:氐

    烹:亨 √

    莅:立

    神:申 √

    寶:󱀘、葆、𤥯 √

    加:賀

    拱:共

    駟:四

    味:未

    諾:若

    層:成

    幾:其

    推:隼、誰

    厭:猒 √

    肖:宵

    儉:檢

    戰:單 √

    配:肥

    攘:襄

    敵:適

    懷:褱 √

    威:畏

    活:栝

    網:罔 √

    恢:𧙔 √

    懼:愳、䂂

    税:𨓚、𨁑 √

    堅:賢 √

    脆:椊 √

    枯:𣒞 √

    槁:𩫓 √

    抑:印

    垢:訽

    和:禾

    契:介、芥 √

    徹:勶

    寡:󱁈 √

    舟:周 √

    19、66、46、30、15、64.2、64.3、37、63、2、32.1、32.2、25、5、16、64.1、56、57、55、44、40、9

    絕智弃𠓥(鞭、辯)。民利百伓(倍)。絕攷(巧)棄利。覜(盜)惻(賊)亡又(有)。絕𢡺(僞)棄慮(慮)。民复(復)季(孝)子(慈)。三言以爲史(使)不足。或命之或唬(乎)豆(屬)。視索(素)保僕(樸)。少厶(私)須(寡)欲。

    江𣳠(海)所以爲百浴(谷)王。以其能爲百浴(谷)下。是以能爲百浴(谷)王。聖人之才(在)民前也。以身後之。其才(在)民上也。以言下之。其才(在)民上也。民弗 厚也。其在民前也。民弗 害也。天下樂進而弗詀(厭)。以其不靜(爭)也。古(故)天下莫能與之靜(爭)。

    辠(罪)莫 厚唬(乎)甚欲。咎莫僉(憯)唬(乎)谷(欲)得。化(禍)莫大唬(乎)不智(知)足。智(知)足之爲足。此𠄨(恒)足矣。

    以𧗟(道)差(佐)人宔(主)者。不谷(欲)以兵强(強)於天下。善者果而已。不以取強。果而弗癹(伐)。果而弗喬(驕)。果而弗 (矜)。是胃(謂)果而弗强(強)。其事好。

    長古之善爲士者。必非(微)溺玄達。深不可志(識)。是以爲之頌(容)。夜(豫)唬(乎)奴(若)冬涉川。猷(猶)唬(乎)其奴(若)愄(畏)四𠳵(鄰)。敢唬(乎)其奴(若)客。 (渙)唬(乎)其奴(若)懌(釋)。屯唬(乎)其奴(若)樸。坉唬(乎)其奴(若)濁。竺(孰)能濁以朿(靜)者。𨟻(將)舍(徐)清。竺(孰)能庀以迬者。𨟻(將)舍(徐)生。保此𧗟(道)者。不谷(欲)(尚)呈(盈)。

    爲之者 敗之。執之者遠之。是以聖人亡爲古(故)亡敗。亡執古(故)亡󶴡(失)。臨事之紀。誓(慎)冬(終)女(如) (始)。此亡敗事矣。

    聖人谷(欲)不谷(欲)。不貴難𠭁(得)之貨。𡥈(學)不𡥈(學)。 復眾之所(過)。是古(故)聖人能尃(輔)萬勿(物)之自肰(然)而弗能爲。

    𧗟(道)𠄨(恒)亡爲也。侯王能守之。而萬勿(物)𨟻(將)自𢡺(化)。𢡺(化)而𨿜(欲) (作)。𨟻(將)貞(鎮)之以亡名之 (樸)。夫亦𨟻(將)智(知)足。智(知)足以朿(靜)。萬勿(物)𨟻(將)自定。

    爲亡(無)爲。事亡(無)事。未(味)亡(無)未(味)。大少(小)之多惕(易)必多戁(難)。是以聖人猷(猶)戁(難)之。古(故)終亡戁(難)。

    天下皆智(知)𢼸(美)之爲(美)也。亞(惡)已。皆智(知)善。此其不善已。又(有)亡之相生也。戁(難)惕(易)之相成也。長耑(短)之相型(形)也。高下之相浧(呈)也。音聖(聲)之相和也。先後之相墮(隨)也。是以聖人居亡爲之事。行不言之𡥈(教)。萬勿(物)(作)而弗(始)也。爲而弗志(恃)也。成而弗居。天〈夫〉唯弗居也。是以弗去也。

    道𠄨(恒)亡名。僕(樸)唯(雖)妻(細)。天(地)弗敢臣。侯王女(如)能獸(守)之。萬勿(物)𨟻(將)自(賓)。

    天(地)相合也。以逾(逾)甘𩂣(露)。民莫之命(令)天〈而〉自均安。(始)折(制)有名。名亦既又(有)。夫亦𨟻(將)智(知)𣥕(止)。智(知)𣥕(止)所以不(殆)。卑(譬)道之在天下也。猷(猶)少(小)浴(谷)之與江𣳠(海)。

    又(有)(道)蟲城(成)。先天(地)生。敓(穆)。蜀(獨)立不亥(改)。可以爲天下母。未智(知)其名。(字)之曰道。(吾)(强)爲之名曰大。大曰()。()曰()〈遠〉。()〈遠〉曰反(返)。天大。(地)大。道大。王亦大。(域)中又(有)四大安(焉)。王(居)一安(焉)。人灋(地)。(地)灋天。天灋道。道灋自肰(然)。

    天(地)之(間)。其猷(猶)(橐)〈籥〉與。虛而不屈。(動)而愈出。

    至虛𠄨(恒)也。獸(守)中(篤)也。萬勿(物)方(旁)(作)。居以須(復)也。天道員員。各󵯿(復)其堇(根)。

    其安也。易𣏔(持)也。其未(兆)也。易𢘃(謀)也。其(脆)也。易畔(判)也。其幾也。易㣤(散)也。爲之於其亡又(有)也。𥿆(治)之於其未亂。𣌭(合)☐☐☐☐☐☐末。九成之臺。作☐☐☐☐☐☐☐☐☐足下。

    智(知)之者弗言。言之者弗智(知)。𨳮〈閉〉其𨓚(兌)。賽(塞)其門。和其光。迵(同)其󶴤(塵)。󶴤其󶩴。解其紛。是胃(謂)玄同。古(故)不可得天〈而〉新(親)。亦不可得而疋(疏)。不可得而利。亦不可得而害。不可得而貴。亦可不可得而戔(賤)。古(故)爲天下貴。

    以正之(治)邦。以(奇)甬(用)兵。以亡事取天下。(吾)可(何)以智(知)其肰(然)也。夫天多期(忌)韋(諱)而民爾(彌)畔(叛)。民多利器而邦慈(滋)昏。人多智天〈而〉𢦪(何)勿(物)慈(滋)(起)。法勿(物)慈(滋)章(彰)。覜(盜)惻(賊)多又(有)。是以聖人之言曰。我亡事而民自(富)。我亡爲而民自(化)。我好青(靜)而民自正。我谷(欲)不谷(欲)而民自樸。

    酓(含)惪(德)之厚者。比於赤子。󶵎(螝)䘍蟲它(蛇)弗𧍷。攫鳥󶵏(猛)獸弗扣。骨溺(弱)蓳(筋)󶵐(柔)而捉固。未智(知)牝戊(牡)之合󶵑󶵒(怒)。精之至也。終日󶴋(號)而不𪬐(啞)。和之至也。和曰󶵓〈󼲗(常)〉。智(知)和曰明。賹(益)生曰羕(祥)。心󶴎(使)󶴓(氣)曰󶴔(强)。勿(物)𡒉(壯)則老。是胃(謂)不道。

    名與身䈞(孰)新(親)。身與貨䈞(孰)多。󰴼(得)與󶵔(亡)䈞(孰)󶓄(病)。甚㤅(愛)必大󶵖(費)。󶵗(厚)󶤖(藏)必多󶵔(亡)。古(故)智(知)足不辱。智(知)止不怠(殆)。可以長舊(久)。

    返也者。道僮(動)也。溺(弱)也者。道之甬(用)也。天下之勿(物)生於又(有)。生於亡。

    𣏔(持)而浧(盈)之。不不若已。湍而群之。不可長保也。金玉浧(盈)室。莫能獸(守)也。貴(富)喬(驕)。自遺咎也。攻(功)述(遂)身退。天之道也。

    59、48、20、13、41、52、45.1、45.2、54

    紿(治)人事天。莫若嗇。夫唯嗇。是以󶵙(早)是以󶵙備(服)。是胃(謂)⋯⋯[無]不克。[無]不克則莫智(知)其𠄨(極)。莫智(知)其𠄨(極)。可以又(有)󼷜(國)。又(有)󼷜(國)之母。可以長⋯⋯長生舊(久)視之道也。

    學者日益。爲道者日員(損)。員(損)之或員(損)。以至亡爲也。亡爲而亡不爲。

    󶴐(絕)學無𢝊(憂)。唯與可(呵)。相去幾可(何)。𡵂(美)與亞(惡)。相去可(何)若。人之所𥚸(畏)。亦不可以不𥚸(畏)人。

    𢤲(寵)辱若纓(驚)。貴大患若身。可(何)胃(謂)𢤲(寵)辱。𢤲(寵)爲下也。得之若纓(驚)。󶴡(失)之若纓(驚)。是胃(謂)寵辱纓(驚)。☐☐☐☐☐若身。󼾲(吾)所以又(有)大患者。爲󼾲(吾)又(有)身。﨤(及)󼾲(吾)亡身。或可(何)☐☐☐☐☐☐爲天下。若可以厇(託)天下矣。㤅(愛)以身爲天下。若可以迲天下矣。

    上士昏(聞)道。堇(勤)能行於其中。中士昏(聞)道。若昏(聞)若亡。下士昏(聞)道。大𦬫(笑)之。弗大𦬫(笑)。不足以爲道矣。是以建言又(有)之:明道女(如)孛(昧)。遲(夷)道[如繢]。☐道若退。上惪(德)女(如)浴(谷)。大白女(如)辱。󼧊(廣)惪(德)女(如)不足。建惪(德)女(如)󲳴☐。☐貞(真)女(如)愉。大方亡禺(隅)。大器曼(慢)成。大音祗聖(聲)。天(大)象亡坓(形)。道⋯⋯

    閟(閉)其門。賽(塞)其𨓚(兌)。終身不󼲆。啟其𨓚(兌)。賽(塞)其事。終身不󶵠。

    大成若夬(缺)。其甬(用)不󶵢(弊)。大浧(盈)若中(盅)。其甬(用)不󶵣(窮)。大攷(巧)若㑁(拙)。大成若詘。大植(直)若屈。

    善建者不拔。善伂(抱)者不兌(脫)。子孫以其祭祀不乇。攸(修)之身。其惪(德)乃貞(真)。攸(修)之𧱌(家)。其惪(德)又(有)舍(餘)。攸(修)之向(鄉)。其惪(德)乃長。攸(修)之邦。其惪(德)乃奉(豐)。攸(修)之天下☐☐☐☐☐☐☐𧱌(家)。以向(鄉)觀向(鄉)。以邦觀邦。以天下觀天下。󼾲(吾)可以智(知)天☐☐☐☐☐。

    17\&{}18、35、31、64.2\&{}64.3

    大上下智(知)有之。其即(次)新(親)譽之。其既〈即(次)〉愄(畏)之。其即(次)󶵡(侮)之。信不足。安又(有)不信。猷(猶)唬(乎)其貴言也。成事述(遂)𬒬(功)。而百眚(姓)曰我自肰(然)也。古(故)大道癹(廢)。安又(有)󶴑(仁)義。六新(親)不和。安又有孝𡥝(慈)。邦𧱌(家)緍(昏)☐。安又(有)正臣。

    埶大象。天下往。往而不害。安坪(平)大。樂與餌。󶵨(過)客止。古(故)道☐☐☐。啖可(呵)其無味也。視之不足見。聖(聽)之不足𦖞(聞)。而不可既也。

    君子居則貴左。甬(用)兵則貴右。古(故)曰兵者☐☐☐☐☐☐得已而甬(用)之。銛󶵩爲上。弗󶴴(美)也。󶵪〈美〉之。是樂殺人。夫樂☐☐☐以得志於天下。古(故)吉事上左。喪事上右。是以卞(偏)𨟻(將)軍居左。上𨟻(將)軍居右。言以喪豊(禮)居之也。古(故)殺☐(人)☐(眾)則以𢙇(哀)悲位(蒞)之。戰󼡲(勝)喪豊(禮)居之。

    爲之者敗之。執之者󶴡(失)之。聖人無爲。古(故)無敗也。無執。古(故)☐☐☐。󶴤(慎)終若󶴪(始)。則無敗事喜(矣)。人之敗也。𠄨(恒)於其𠭯(且)成也敗之。是以☐(聖)人欲不欲。不貴戁(難)得之貨。學不學。復眾之所󶴭(過)。是以能㭪(輔)󼧕(萬)勿(物)之自肰(然)而弗敢爲。

    \section*{参考文献}
    《郭店楚簡老子集釋》,彭裕商、吴毅强 集釋,巴蜀書社。

    《帛書老子校注》,高明 撰,中華書局。

    《老子道德經注校釋》,樓宇烈 校釋,中華書局。
\end{document}
