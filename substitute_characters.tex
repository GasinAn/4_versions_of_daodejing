\documentclass[a4paper]{ctexart}
\usepackage{geometry}
\geometry{a5paper}
\title{通假字集录}
\author{安梯西登}
\date{}
\begin{document}
    \maketitle

    妙:眇

    有:又

    徼:噭

    謂:胃

    惡:亞

    斯:訾

    音:意

    隨:隋

    聖:聲、𦔻

    作:昔

    辭:始

    恃:志、侍、寺

    智:知

    淵:潚

    似:始

    挫:銼

    銳:兑

    紛:芬

    姓:省

    屈:淈

    動:蹱、勭、重、僮

    愈:俞

    谷:浴

    勤:堇

    退:芮

    爭:静、争

    持:㨁

    揣:𢵦

    保:葆

    驕:䮦

    遂:述

    魄:柏

    離:离、雞

    專:槫

    致:至

    滌:脩

    覽:藍、監

    治:栝、之

    埏:𡑋

    盲:明

    獵:臘

    貨:𧷴

    妨:方、仿

    爽:𠷹

    彼:罷、皮

    寵:龍、弄

    患:梡

    何:苛

    託:𨒙、橐

    微:󱁚

    名:命

    詰:計

    混:𡇯、𦀲

    皦:攸、謬

    昧:忽

    惚:沕、忽
    
    恍:望、朢

    識:志

    豫:與

    鄰:𠳵

    釋:澤

    敦:󱁜、沌

    曠:𣼥

    混:湷

    靜:情、清、靚

    安:女

    徐:余

    蔽:𧝬

    篤:督

    並:旁

    妄:㠵、芒

    没:沕
    
    殆:怠

    侮:母

    孝:畜

    慈:兹

    昏:𨴽、󱁝

    哉:才

    熙:巸

    泊:博

    兆:佻、垗

    餘:余

    沌:惷、湷

    俗:鬻

    察:蔡

    悶:𨴽、󱁅

    頑:䦎

    窈:幼

    冥:鳴

    全:金

    枉:汪

    敝:𧝬

    是:視

    彰:章

    豈:幾

    飄:𠠕

    終:冬、衆

    況:兄

    企:炊

    餘:𥺌

    寂:繡、蕭

    寥:繆、漻

    改:󱁡

    逝:筮

    輕:巠

    躁:趮

    輜:甾

    雖:唯

    觀:官

    超:昭

    轍:勶、達

    籌:檮、

    策:𥮥、󱁦

    關:󱁥

    救:㤹

    襲:𢘽、曳

    資:齎

    忒:貣

    樸:楃、握

    吹:䂳

    奢:楮、諸

    泰:大

    棘:朸

    早:蚤

    恬:銛

    淡:襲、𢤱

    偏:便

    哀:依

    泣:立

    勝:朕

    露:洛

    譬:俾、卑

    亡:忘

    氾:渢

    客:格

    淡:談

    固:古

    柔:󱁩

    脱:説

    示:視

    化:𢡺

    鎮:闐

    欲:辱

    應:𤻮

    扔:乃

    後:后、句

    薄:泊

    靈:霝

    裂:蓮

    竭:渴

    蹶:欮

    穀:𥞤

    琭:禄

    珞:硌

    昧:費

    隅:禺

    形:刑

    沖:中

    物:勿

    損:𢿃、云

    益:

    亦:夕

    我:議

    梁:良

    至:致

    吾:五

    窮:𡩫

    清:請

    禍:𢢸

    闚:規、𧠹

    彌:𢑃、󱁁

    攝:執

    甲:革

    投:椯

    措:昔

    爪:蚤

    覆:復

    既:𢟪

    殆:佁

    兑:󱁂、㙂

    濟:齊

    救:棘

    殃:央

    徑:解、𠎿

    豐:夆

    蜂:逢
    
    蠆:𢔯、癘
    
    虺:𧍥、䖝
    
    蛇:地
    
    螫:赫

    猛:孟

    攫:㩴、據

    搏:捕

    嗄:𢖻

    賤:淺

    奇:畸

    滋:兹

    政:正

    迷:𢘻

    廉:兼

    肆:紲

    燿:眺

    根:󱁆

    柢:氐

    烹:亨

    莅:立

    神:申

    \section{}
    王弼本:

    大國者下流。天下之交。天下之牝。牝常以靜勝牡。以靜爲下。故大國以下小國。則取小國。小國以下大國。則取大國。故或下以取。或下而取。大國不過欲兼畜人。小國不過欲入事人。夫兩者各得其所欲。大者宜爲下。

    河上公本:

    大國者下流。天下之交。天下之牝。牝常以靜勝牡。以静爲下。故大國以下小國。則取小國。小國以下大國。則取大國。或下以取。或下而取。大國不過欲兼畜人。小國不過欲入事人。夫兩者各得其所欲。大者宜爲下。

    帛书甲本:

    大邦者下流也。天下之牝。天下之郊也。牝恒以靚勝牡。爲其靚〔也。故〕宜爲下。大邦〔以〕下小〔國〕。則取小邦。小邦以下大邦。則取於大邦。故或下以取。或下而取。〔故〕大邦者不過欲兼畜人。小邦者不過欲入事人。夫皆得其欲。〔大者宜〕爲下。

    帛书乙本:

    大國〔者下流也。天下之〕牝也。天下之交也。牝恒以静朕牡。爲其静也。故宜爲下也。故大國以下〔小〕國。則取小國。小國以下大國。則取於大國。故或下〔以取。或〕下而取。故大國者不〔過〕欲并畜人。小國不過欲入事人。夫〔皆得〕其欲。則大者宜爲下。

    \section{}
    王弼本:

    道者萬物之奥。善人之寶。不善人之所保。美言可以市。尊行可以加人。人之不善。何棄之有。故立天子。置三公。雖有拱璧以先駟馬。不如坐進此道。古之所以貴此道者何不曰以求得。有罪以免邪。故爲天下貴。

    河上公本:

    道者萬物之奥。善人之寶。不善人之所保。美言可以市。尊行可以加人。人之不善。何棄之有。故立天子。置三公。雖有拱璧以先駟馬。不如坐進此道。古之所以貴此道者何不日以求得。有罪以免耶。故爲天下貴。

    帛书甲本:

    〔道〕者萬物之注也。善人之󱀘也。不善人之所󱀘也。美言可以市。尊行可以賀人。人之不善也。何〔棄之〕有。故立天子。置三卿。雖有共之璧以先四馬。不善坐而進此。古之所以貴此者何也。不胃〔求以〕得。有罪以免輿。故爲天下貴。

    帛书乙本:

    道者萬物之注也。善人之󱀘也。不善人之所保也。美言可以市。尊行可以賀人。人之不善。何〔棄之有〕。〔故〕立天子。置三鄉。雖有〔共之〕璧以先四馬。不若坐而進此。古〔之所以貴此者何也〕。不胃求以得。有罪以免與。故爲天下貴。

    \section{}
    王弼本:

    爲無爲。事無事。味無味。大小多少。報怨以德。圖難於其易。爲大於其細。天下難事必作於易。天下大事必作於細。是以聖人終不爲大。故能成其大。夫輕諾必寡信。多易必多難。是以聖人猶難之。故終無難矣。

    河上公本:

    爲無爲。事無事。味無味。大小多少。報怨以德。圖難於其易。爲大於其細。天下難事必作於易。天下大事必作於細。是以聖人終不爲大。故能成其大。夫輕諾必寡信。多易必多難。是以聖人猶難之。故終無難。

    帛书甲本:

    爲无爲。事无事。味无未。大小多少。報怨以德。圖難乎〔其易也。爲大乎其細也〕。天下之難作於易。天下之大作於細。是以聖人冬不爲大。故能〔成其大〕。〔夫輕若必寡信。多易〕必多難。是〔以𦔻〕人猶難之。故終於无難。

    帛书乙本:

    爲无爲。〔事无事。味无未。大小多少。報怨以德〕。〔圖難乎其易也。爲大〕乎其細也。天下之〔難作於〕易。天下之大〔作於細。是以聖人冬不爲大。故能成其大〕。夫輕若〔必寡〕信。多易必多難。是以𦔻人〔猶難〕之。故〔終於无難〕。

    楚简甲本:

    爲亡(無)爲。事亡(無)事。未(味)亡(無)未(味)。大少(小)之多惕(易)必多󶴱(難)。是以聖人猷(猶)󶴱(難)之。古(故)終亡󶴱(難)。

    \section{}
    王弼本:

    其安易持。其未兆易謀。其脆易泮。其微易散。爲之於未有。治之於未亂。合抱之木。生於毫末。九層之臺。起於累土。千里之行。始於足下。爲者敗之。執者失之。是以聖人無爲故無敗。無執故無失。民之從事。常於幾成而敗之。慎終如始。則無敗事。是以聖人欲不欲。不貴難得之貨。學不學。復衆人之所過。以輔萬物之自然。而不敢爲。

    河上公本:

    其安易持。其未兆易謀。其脆易破。其微易散。爲之於未有。治之於未亂。合抱之木。生於毫末。九層之臺。起於累土。千里之行。始於足下。爲者敗之。執者失之。聖人無爲故無敗。无執故無失。民之從事。常於幾成而敗之。慎終如始。則無敗事。是以聖人欲不欲。不貴難得之貨。學不學。復衆人之所過。以輔萬物之自然。而不敢爲。

    帛书甲本:

    其安也。易持也。⋯⋯。〔合抱之木。生於〕毫末。九成之臺。作於羸土。百仁之高。台於足〔下〕。〔爲之者敗之。執者失之。是以𦔻人无爲〕也。〔故〕无敗〔也〕。无執也。故无失也。民之從事也。恒於其成事而敗之。故慎終若始。則〔无敗事矣〕。〔是以𦔻人〕欲不欲。而不貴難得之𦠽。學不學。而復衆人之所過。能輔萬物之自〔然。而〕弗敢爲。

    帛书乙本:

    ⋯⋯。〔合抱之〕木。生於毫末。九成之臺。作於虆土。百千之高。始於足下。爲之者敗之。執者失之。是以𦔻人无爲〔也。故无敗也。无執也。故无失也〕。民之從事也。恒於其成而敗之。故曰慎冬若始。則无敗事矣。是以𦔻人欲不欲。而不貴難得之貨。學不學。復衆人之所過。能輔萬物之自然。而弗敢爲。

    楚简甲本:

    其安也。易𣏔(持)也。其未󶵆(兆)也。易𢘃(謀)也。其󶵇(脆)也。易畔(判)也。其幾也。易㣤(散)也。爲之於其亡又(有)也。𥿆(治)之於其未亂。𣌭(合)☐☐☐☐☐☐末。九成之臺。作☐☐☐☐☐☐☐☐☐足下。

    爲之者敗之。執之者遠之。是以聖人亡爲古(故)亡敗。亡執古(故)亡󶴡(失)。臨事之紀。誓(慎)冬(終)女(如)󶴢(始)。此亡敗事矣。

    聖人谷(欲)不谷(欲)。不貴難󶴫(得)之貨。𡥈(學)不𡥈(學)。復眾之所󶴬(過)。是古(故)聖人能尃(輔)萬勿(物)之自肰(然)而弗能爲。

    楚简丙本:

    爲之者敗之。執之者󶴡(失)之。聖人無爲。古(故)無敗也。無執。古(故)☐☐☐。󶴤(慎)終若󶴪(始)。則無敗事喜(矣)。人之敗也。𠄨(恒)於其𠭯(且)成也敗之。是以☐(聖)人欲不欲。不貴戁(難)得之貨。學不學。復眾之所󶴭(過)。是以能㭪(輔)󼧕(萬)勿(物)之自肰(然)而弗敢爲。

    \section{}
    王弼本:

    古之善爲道者。非以明民。將以愚之。民之難治。以其智多。故以智治國。國之賊。不以智治國。國之福。知此兩者亦稽式。常知稽式。是謂玄德。玄德深矣。遠矣。與物反矣。然後乃至大順。

    河上公本:

    古之善爲道者。非以明民。將以愚之。民之難治。以其智多。以智治國。國之賊。不以智治國。國之福。知此兩者亦楷式。常知楷式。是謂玄德。玄德深矣。遠矣。與物反矣。乃至大順。

    帛书甲本:

    故曰爲道者非以明民也。將以愚之也。民之難〔治也。以其〕知也。故以知知邦。邦之賊也。以不知知邦。〔國之〕德也。恒知此兩者。亦稽式也。恒知稽式。此胃玄德。玄德深矣。遠矣。與物〔反〕矣。乃至大順。

    帛书乙本:

    古之爲道者。非以明〔民也。將以愚〕之也。夫民之難治也。以其知也。故以知知國。國之賊也。以不知知國。國之德也。恒知此兩者。亦稽式也。恒知稽式。是胃玄德。玄德深矣。遠矣。〔與〕物反也。乃至大順。

    \section{}
    王弼本:

    江海所以能爲百谷王者。以其善下之。故能爲百谷王。是以欲上民。必以言下之。欲先民。必以身後之。是以聖人處上而民不重。處前而民不害。是以天下樂推而不厭。以其不爭。故天下莫能與之爭。

    河上公本:

    江海所以能爲百谷王者。以其善下之。故能爲百谷王。是以聖人欲上民。必以〔其〕言下之。欲先民。必以〔其〕身後之。是以聖人處上而民不重。處前而民不害。是以天下樂推而不厭。以其不爭。故天下莫能與之爭。

    帛书甲本:

    〔江〕海之所以能爲百浴王者。以其善下之。是以能爲百浴王。是以聖人之欲上民也。必以其言下之。其欲先〔民也〕。必以其身後之。故居前而民弗害也。居上而民弗重也。天下樂隼而弗猒也。非以其无静與。〔故天下莫能與〕静。

    帛书乙本:

    江海所以能爲百浴〔王者。以〕其〔善〕下之也。是以能爲百浴王。是以𦔻人之欲上民也。必以其言下之。其欲先民也。必以其身後之。故居上而民弗重也。居前而民弗害。天下皆樂誰而弗猒也。不以其无争與。故〔天〕下莫能與争。

    楚简甲本:

    江𣳠(海)所以爲百浴(谷)王。以其能爲百浴(谷)下。是以能爲百浴(谷)王。聖人之才(在)民前也。以身後之。其才(在)民上也。以言下之。其才(在)民上也。民弗厚也。其在民前也。民弗害也。天下樂進而弗詀(厭)。以其不靜(爭)也。古(故)天下莫能與之靜(爭)。

    \section{}
    王弼本:

    天下皆謂我道大。似不肖。夫唯大。故似不肖。若肖久矣其細也夫。我有三寶。持而保之。一曰慈。二曰儉。三曰不敢爲天下先。慈故能勇。儉故能廣。不敢爲天下先。故能成器長。今舍慈且勇。舍儉且廣。舍後且先。死矣。夫慈。以戰則勝。以守則固。天將救之。以慈衛之。

    河上公本:

    天下皆謂我大。似不肖。夫唯大。故似不肖。若肖久矣其細〔也夫〕。我有三寶。持而保之。一曰慈。二曰儉。三曰不敢爲天下先。慈故能勇。儉故能廣。不敢爲天下先。故能成器長。今舍〔其〕慈且勇。舍〔其〕儉且廣。舍〔其〕後且先。死矣。夫慈。以戰則勝。以守則固。天將救之。以慈衛之。

    帛书甲本:

    〔天下皆胃我大。大而不宵〕。夫唯〔大〕。故不宵。若宵。細久矣。我恒有三葆之。一曰兹。二曰檢。〔三曰不敢爲天下先〕。〔夫兹故能勇。檢〕故能廣。不敢爲天下先。故能爲成事長。今舍其兹且勇。舍其後且先。則必死矣。夫兹。〔以單〕則勝。以守則固。天將建之。女以兹垣之。

    帛书乙本:

    天下〔皆〕胃我大。大而不宵。夫唯不宵。故能大。若宵。久矣其細也夫。我恒有三𤥯。市而𤥯之。一曰兹。二曰檢。三曰不敢爲天下先。夫兹故能勇。檢敢能廣。不敢爲天下先。故能爲成器長。今舍其兹且勇。舍其檢且廣。舍其後且先。則死矣。夫兹。以單則朕。以守則固。天將建之。如以兹垣之。

    \section{}
    王弼本:

    善爲士者不武。善戰者不怒。善勝敵者不與。善用人者爲之下。是謂不爭之德。是謂用人之力。是謂配天古之極。

    河上公本:

    善爲士者不武。善戰者不怒。善勝敵者不與。善用人者爲下。是謂不爭之德。是謂用人之力。是謂配天古之極。

    帛书甲本:

    善爲士者不武。善戰者不怒。善勝敵者弗〔與〕。善用人者爲之下。〔是〕胃不諍之德。是胃用人。是胃天。古之極也。

    帛书乙本:

    故善爲士者不武。善單者不怒。善朕敵者弗與。善用人者爲之下。是胃不爭〔之〕德。是胃用人。是胃肥天。古之極也。

    \section{}
    王弼本:

    用兵有言。吾不敢爲主而爲客。不敢進寸而退尺。是謂行無行。攘無臂。扔無敵。執無兵。禍莫大於輕敵。輕敵幾喪吾寶。故抗兵相加。哀者勝矣。

    河上公本:

    用兵有言。吾不敢爲主而爲客。不敢進寸而退尺。是謂行無行。攘無臂。仍無敵。執無兵。禍莫大於輕敵。輕敵幾喪吾寳。故抗兵相加。哀者勝矣。

    帛书甲本:

    用兵有言曰。吾不敢爲主而爲客。吾不進寸而芮尺。是胃行无行。襄无臂。執无兵。乃无敵矣。𢢸莫於於无適。无適斤亡吾吾葆矣。故稱兵相若。則哀者勝矣。

    帛书乙本:

    用兵又言曰。吾不敢爲主而爲客。不敢進寸而退尺。是胃行无行。攘无臂。執无兵。乃无敵。禍莫大於無敵。無敵近亡吾𤥯矣。故抗兵相若。而依者朕〔矣〕。

    \section{}
    王弼本:

    吾言甚易知。甚易行。天下莫能知。莫能行。言有宗。事有君。夫唯無知。是以不我知。知我者希。則我者貴。是以聖人被褐懷玉。

    河上公本:

    吾言甚易知。甚易行。天下莫能知。莫能行。言有宗。事有君。夫惟無知。是以不我知。知我者希。則我者貴。是以聖人被褐懷玉。

    帛书甲本:

    吾言甚易知也。甚易行也。而人莫之能知也。而莫之能行也。言有君。事有宗。夫唯无知也。是以不〔我知。知我者希。則〕我貴矣。是以聖人被褐而褱玉。

    帛书乙本:

    吾言易知也。易行也。而天下莫之能知也。莫之能行也。夫言又宗。事又君。夫唯无知也。是以不我知。知者希。則我貴矣。是以𦔻人被褐而褱玉。

    \section{}
    王弼本:

    知不知上。不知知病。夫唯病病。是以不病。聖人不病。以其病病。是以不病。

    河上公本:

    知不知上。不知知病。夫唯病病。是以不病。聖人不病。以其病病。是以不病。

    帛书甲本:

    知不知尚矣。不知不知病矣。是以聖人之不病。以其〔病病。是以不病〕。

    帛书乙本:

    知不知尚矣。不知知病矣。是以𦔻人之不〔病〕也。以其病病也。是以不病。

    \section{}
    王弼本:

    民不畏威。則大威至。無狎其所居。無厭其所生。夫唯不厭。是以不厭。是以聖人自知不自見。自愛不自貴。故去彼取此。

    河上公本:

    民不畏威。〔則〕大威至矣。無狹其所居。無厭其所生。夫惟不厭。是以不厭。是以聖人自知不自見。自愛不自貴。故去彼取此。

    帛书甲本:

    〔民之不〕畏畏。則大〔畏將至〕矣。毋閘其所居。毋猒其所生。夫唯弗猒。是〔以不猒〕。〔是以𦔻人自知而不自見也。自愛〕而不自貴也。故去被取此。

    帛书乙本:

    民之不畏畏。則大畏將至矣。毋𠇺其所居。毋猒其所生。夫唯弗猒。是以不猒。是以𦔻人自知而不自見也。自愛而不自貴也。故去罷而取此。

    \section{}
    王弼本:

    勇於敢則殺。勇於不敢則活。此兩者或利或害。天之所惡。孰知其故。是以聖人猶難之。天之道。不爭而善勝。不言而善應。不召而自來。繟然而善謀。天網恢恢。疏而不失。

    河上公本:

    勇於敢則殺。勇於不敢則活。此兩者或利或害。天之所惡。孰知其故。是以聖人猶難之。天之道。不爭而善勝。不言而善應。不召而自來。繟然而善謀。天網恢恢。踈而不失。

    帛书甲本:

    勇於敢者〔則殺。勇〕於不敢者則栝。〔此兩者或利或害。天之所亞。孰知其故〕。〔天之道。不戰而善勝〕。不言而善應。不召而自來。彈而善謀。〔天罔𧙔𧙔。疏而不失〕。

    帛书乙本:

    勇於敢則殺。勇於不敢則栝。〔此〕兩者或利或害。天之所亞。孰知其故。天之道。不單而善朕。不言而善應。弗召而自來。單而善謀。天罔𧙔𧙔。疏而不失。

    \section{}
    王弼本:

    民不畏死。奈何以死懼之。若使民常畏死。而爲奇者吾得執而殺之。孰敢。常有司殺者殺。夫代司殺者殺。是謂代大匠斲。夫代大匠斲者。希有不傷其手矣。

    河上公本:

    民不畏死。奈何以死懼之。若使民常畏死。而爲奇者吾得執而殺之。孰敢。常有司殺者。夫代司殺者。是謂代大匠斵。夫代大匠斵者。希有不傷手矣。

    帛书甲本:

    〔若民恒且不畏死〕。奈何以殺愳之也。若民恒是死。則而爲者吾將得而殺之。夫孰敢矣。若民〔恒且〕必畏死。則恒有司殺者。夫伐司殺者殺。是伐大匠斲也。夫伐大匠斲者。則〔希〕不傷其手矣。

    帛书乙本:

    若民恒且畏不畏死。若何以殺䂂之也。使民恒且畏死。而爲畸者〔吾〕得而殺之。夫孰敢矣。若民恒且必畏死。則恒又司殺者。夫代司殺者殺。是代大匠斲。夫代大匠斲。則希不傷其手。

    \section{}
    王弼本:

    民之饑。以其上食税之多。是以饑。民之難治。以其上之有爲。是以難治。民之輕死。以其求生之厚。是以輕死。夫唯無以生爲者。是賢於貴生。

    河上公本:

    民之飢。以其上食税之多。是以飢。民之難治。以其上有爲。是以難治。民之輕死。以其求生之厚。是以輕死。夫唯無以生爲者。是賢於貴生。

    帛书甲本:

    人之飢也。以其取食𨓚之多也。是以飢。百姓之不治也。以其上有以爲〔也〕。是以不治。民之巠死。以其求生之厚也。是以巠死。夫唯无以生爲者。是賢貴生。

    帛书乙本:

    人之飢也。以其取食𨁑之多。是以飢。百生之不治也。以其上之有以爲也。〔是〕以不治。民之輕死也。以其求生之厚也。是以輕死。夫唯无以生爲者。是賢貴生。

    \section{}
    王弼本:

    人之生也柔弱。其死也堅强。萬物草木之生也柔脆。其死也枯槁。故堅强者死之徒。柔弱者生之徒。是以兵强則不勝。木强則兵。强大處下。柔弱處上。

    河上公本:

    人之生也柔弱。其死也堅强。萬物草木之生也柔脆。其死也枯槁。故堅强者死之徒。柔弱者生之徒。是以兵强則不勝。木强則共。强大處下。柔弱處上。

    帛书甲本:

    人之生也柔弱。其死也𦵕仞賢强。萬物草木之生也柔脆。其死也𣒞𩫓。故曰堅强者死之徒也。柔弱微細生之徒也。兵强則不勝。木强則恒。强大居下。柔弱微細居上。

    帛书乙本:

    人之生也柔弱。其死也󱁌信堅强。萬〔物草〕木之生也柔椊。其死也𣒞槁。故曰堅强死之徒也。柔弱生之徒也。〔是〕以兵强則不朕。木强則競。故强大居下。柔弱居上。

    \section{}
    王弼本:

    天之道。其猶張弓與。高者抑之。下者舉之。有餘者損之。不足者補之。天之道損有餘而補不足。人之道則不然。損不足以奉有餘。孰能有餘以奉天下。唯有道者。是以聖人爲而不恃。功成而不處。其不欲見賢。

    河上公本:

    天之道。其猶張弓乎。高者抑之。下者舉之。有餘者損之。不足者益之。天之道損有餘而補不足。人之道則不然。損不足以奉有餘。孰能有餘以奉天下。唯有道者。是以聖人爲而不恃。功成而不處。其不欲見賢。

    帛书甲本:

    天下〔之道。酉張弓〕者也。高者印之。下者舉之。有餘者𢿃之。不足者補之。故天之道。𢿃有〔余而益不足。人之道則〕不然。𢿃〔不足而〕奉有餘。孰能有餘而有以取奉於天者乎。〔唯有道者乎〕。〔是以𦔻人爲而弗又。成功而弗居也。若此其不欲〕見賢也。

    帛书乙本:

    天之道。酉張弓也。高者印之。下者舉之。有余者云之。不足者〔補之〕。〔故天之道〕。云有余而益不足。人之道。云不足而奉又余。夫孰能又余而〔有以取〕奉於天者。唯又道者乎。是以𦔻人爲而弗又。成功而弗居也。若此其不欲見賢也。

    \section{}
    王弼本:

    天下莫柔弱於水。而攻堅强者莫之能勝。其無以易之。弱之勝强。柔之勝剛。天下莫不知。莫能行。是以聖人云。受國之垢。是謂社稷主。受國不祥。是爲天下王。正言若反。

    河上公本:

    天下柔弱莫過於水。而攻堅强者莫之能勝。其無以易之。弱之勝强。柔之勝剛。天下莫不知。莫能行。故聖人云。受國之垢。是謂社稷主。受國之不祥。是謂天下王。正言若反。

    帛书甲本:

    天下莫柔〔弱於水。而攻〕堅强者莫之能〔勝〕也。以其无〔以〕易〔之也〕。〔水之朕剛也。弱之〕勝强。天〔下莫弗知也。而莫能〕行也。故聖人之言云曰。受邦之訽。是胃社稷之主。受邦之不祥。是胃天下之王。〔正言〕若反。

    帛书乙本:

    天下莫柔弱於水。〔而攻堅强者莫之能勝〕。以其无以易之也。水之朕剛也。弱之朕强也。天下莫弗知也。而〔莫能行〕也。是故𦔻人言云曰。受國之訽。是胃社稷之主。受國之不祥。是胃天下之王。正言若反。

    \section{}
    王弼本:

    和大怨。必有餘怨。安可以爲善。是以聖人執左契。而不責於人。有德司契。無德司徹。天道無親。常與善人。

    河上公本:

    和大怨。必有餘怨。安可以爲善。是以聖人執左契。而不責於人。有德司契。無德司徹。天道無親。常與善人。

    帛书甲本:

    和大怨。必有餘怨。焉可以爲善。是以聖右介。而不以責於人。故有德司介。〔无〕德司勶。夫天道无親。恒與善人。

    帛书乙本:

    禾大〔怨。必有餘怨。焉可以〕爲善。是以聖人執左芥。而不以責於人。故又德司芥。无德司勶。〔夫天道无親。恒與善人〕。德三千𠦜一。

    \section{}
    王弼本:

    小國寡民。使有什伯之器而不用。使民重死而不遠徙。雖有舟輿。無所乘之。雖有甲兵。無所陳之。使人復結繩而用之。甘其食。美其服。安其居。樂其俗。鄰國相望。鷄犬之聲相聞。民至老死不相往來。

    河上公本:

    小國寡民。使〔民〕有什伯人之器而不用。使民重死而不遠徙。雖有舟輿。無所乘之。雖有甲兵。無所陳之。使民復結繩而用之。甘其食。美其服。安其居。樂其俗。鄰國相望。雞狗之聲相聞。民至老〔死〕不相往來。

    帛书甲本:

    小邦󱁈民。使十百人之器毋用。使民重死而遠徙。有車周无所乘之。有甲兵无所陳〔之。使民復結繩而〕用之。甘其食。美其服。樂其俗。安其居。󱁉邦相朢。鷄狗之聲相聞。民至〔老死不相往來〕。

    帛书乙本:

    小國寡民。使有十百人器而勿用。使民重死而遠徙。又周車无所乘之。有甲兵无所陳之。使民復結繩而用之。甘其食。美其服。樂其俗。安其居。𠳵國相望。鷄犬之〔聲相〕聞。民至老死不相往來。

    \section{}
    王弼本:

    信言不美。美言不信。善者不辯。辯者不善。知者不博。博者不知。聖人不積。既以爲人。己愈有。既以與人。己愈多。天之道。利而不害。聖人之道。爲而不爭。

    河上公本:

    信言不美。美言不信。善者不辯。辯者不善。知者不博。博者不知。聖人不積。既以爲人。己愈有。既以與人。己愈多。天之道。利而不害。聖人之道。爲而不爭。

    帛书甲本:

    〔信言不美。美言〕不〔信。知〕者不博。〔博〕者不知。善〔者不多。多〕者不善。聖人无積。〔既〕以爲〔人。己俞有。既以予人矣。己俞多〕。⋯⋯。

    帛书乙本:

    信言不美。美言不信。知者不博。博者不知。善者不多。多者不善。𦔻人无積。既以爲人。己俞有。既以予人矣。己俞多。故天之道。利而不害。人之道。爲而弗爭。

    \section*{参考文献}
    《郭店楚簡老子集釋》,彭裕商、吴毅强 集釋,巴蜀書社。

    《帛書老子校注》,高明 撰,中華書局。

    《老子道德經注校釋》,樓宇烈 校釋,中華書局。
\end{document}
