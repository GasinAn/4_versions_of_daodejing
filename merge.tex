\documentclass[a5paper]{ctexbook}
\usepackage{geometry}
\usepackage{hyperref}
\usepackage{ulem}
\usepackage{xcolor}
\geometry{a5paper}
\newcommand{\he}[1]{\textcolor[RGB]{255,0,255}{#1}}
\newcommand{\bo}[1]{\textcolor[RGB]{0,255,255}{#1}}
\newcommand{\jian}[1]{\textcolor[RGB]{0,255,0}{#1}}
\def\del{\sout}
\title{《道德经》整理}
\author{安梯西登}
\date{}
\begin{document}

    \maketitle

    \tableofcontents

    \chapter{}

    道可道。非\bo{恒}道。名可名。非\bo{恒}名。無名。\bo{萬物}之始。有名。萬物之母。故\bo{恒}無欲。以觀其\bo{眇}。\bo{恒}有欲。以觀其徼。\bo{兩者同出。異名同謂}。玄之又玄。衆\bo{眇}之門。

    \chapter{}

    天下皆知美之爲美。\jian{惡已}。\jian{皆知善。此其不善已}。\jian{\del{故}}有無相生。難易相成。長短相\he{形}。高下相\bo{盈}。音聲相和。\jian{先}後相隨。是以聖人處無爲之事。行不言之教。萬物\jian{作而不始}。\jian{\del{生而不有}}。爲而不\jian{志}。功成而弗居。夫唯弗居。是以不去。

    \chapter{}

    不尚賢。使民不爭。不貴難得之貨。使民不爲盗。不見可欲。使民\bo{\del{心}}不亂。是以聖人之治。虚其心。實其腹。弱其志。强其骨。\bo{恒}使民無知。無欲。\bo{使夫智不敢。不爲而已}。則無不治。

    \chapter{}

    道沖。而用之\bo{有}不盈。淵兮似萬物之宗。挫其銳。解其紛。和其光。同其塵。湛兮似或存。吾不知\bo{其}誰之子。象帝之先。

    \chapter{}

    天地不仁。以萬物爲芻狗。聖人不仁。以百姓爲芻狗。

    天地之間。其猶橐籥乎。虚而不屈。動而愈出。

    多\bo{聞}數窮。不如守中。

    \chapter{}

    \bo{浴}神不死。是謂玄牝。玄牝之門。是謂天地\bo{之}根。緜緜若存。用之不\bo{堇}。

    \chapter{}

    天長地久。天地所以能長且久者。以其不自生。故能長生。是以聖人\bo{退}其身而身先。外其身而身存。非以其無私邪。故能成其私。

    \chapter{}

    上善若水。水善利萬物而不爭。處衆人之所惡。故幾於道。居善地。心善淵。與善\bo{天}。言善信。正善治。事善能。動善時。夫唯不爭。故無尤。

    \chapter{}

    持而盈之。不如其已。揣而棁之。不可長保。金玉\bo{盈室}。莫之能守。富貴而驕。自遺其咎。功遂身退。天之道\jian{也}。

    \chapter{}

    載營魄抱一。能無離乎。專氣致柔。能嬰兒乎。滌除玄\bo{監}。能無疵乎。愛民治國。能\bo{無以知}乎。天門開闔。能\bo{爲雌}乎。明白四達。能\he{無知}乎。生之。畜之。生而不有。\bo{\del{爲而不恃}}。長而不宰。是謂玄德。

    \chapter{}

    三十輻共一轂。當其無。有車之用。埏埴以爲器。當其無。有器之用。鑿户牖以爲室。當其無。有室之用。故有之以爲利。無之以爲用。

    \chapter{}

    五色令人目盲。五音令人耳聾。五味令人口爽。馳騁畋獵令人心發狂。難得之貨令人行妨。\bo{是以聖人之治也。爲腹不爲目}。故去彼取此。

    \chapter{}

    寵辱若驚。貴大患若身。何謂寵辱\jian{\del{若驚}}。寵爲下。得之若驚。失之若驚。是謂寵辱若驚。何謂貴大患若身。吾所以有大患者。爲吾有身。及吾無身。吾有何患。故\bo{貴爲身於爲天下}。若可\bo{託}天下。愛以身爲天下。若可\bo{寄}天下。

    \chapter{}

    視之不見名曰\bo{微}。聽之不聞名曰希。\bo{捪}之不得名曰\bo{夷}。此三者不可致詰。故混而爲一。\bo{一者}。其上不皦。其下不昧。繩繩不可名。復歸於無物。是謂無狀之狀。無物之象。是謂惚恍。迎之不見其首。隨之不見其後。執古之道。以御今之有。能知古始。是謂道紀。

    \chapter{}

    古之善爲士者。\jian{必}微\bo{眇}玄通。深不可\jian{志}。\jian{是以爲之容}。豫焉若冬涉川。猶兮若畏四鄰。儼兮其若\jian{客}。涣兮若\bo{淩釋}。敦兮其若樸。\jian{\del{曠兮其若谷}}。混兮其若濁。\jian{孰能濁以靜者。將徐清}。\jian{孰能安以動者。將徐生}。保此道者。不欲\jian{尚}盈。\jian{\del{夫唯不盈。故能蔽不新成}}。

    \chapter{}

    致虚極\jian{也}。守\jian{中}篤\jian{也}。萬物並作。\jian{居以須復也}。\jian{天道員員。各復其根}。

    歸根曰靜。\bo{静}。是謂復命。\bo{復命常也。知常明也}。不知常。\bo{妄}。妄作。凶。知常容。容乃公。公乃王。王乃天。天乃道。道乃久。没身不殆。

    \chapter{}

    太上下知有之。其次親\jian{\del{而}}譽之。其次畏之。其次侮之。\bo{信不足。安有不信}。悠兮其貴言。功成事遂。\jian{而}百姓\jian{\del{皆}}謂我自然。

    \chapter{}

    \jian{故}大道廢。\jian{安}有仁義。\jian{\del{慧智出。有大僞}}。六親不和。\jian{安}有孝慈。國家昏亂。\jian{安}有\jian{正}臣。

    \chapter{}

    \jian{絕智弃卞}。民利百倍。\jian{絕爲棄慮}。民復孝慈。絶巧棄利。盗賊無有。此三者以爲\jian{史}不足。故令有所屬。見素抱樸。少私寡欲。

    \chapter{}

    絶學無憂。

    唯之與\bo{呵}。相去幾何。\jian{美}之與惡。相去\bo{何若}。人之所畏。\jian{亦不可以不畏人}。

    荒兮其未央哉。衆人熙熙。如享太牢。如春登臺。我獨泊兮其未兆。如嬰兒之未\bo{咳}。儽儽兮若無所歸。衆人皆有餘。而我獨\bo{\del{若}}遺。我愚人之心也哉。沌沌兮。俗人昭昭。我獨昏昏。俗人察察。我獨悶悶。\bo{惚}兮其若海。\bo{恍}兮若無止。衆人皆有以。而我獨頑\bo{以}鄙。我\bo{欲}獨異於人。而貴食母。

    \chapter{}

    孔德之容。惟道是從。道之\bo{\del{爲}}物。惟恍惟惚。惚兮恍兮。其中有象。恍兮惚兮。其中有物。窈兮冥兮。其中有\bo{請}。其\bo{請}甚真。其中有信。\bo{自今及古}。其名不去。以\bo{順}衆\bo{父}。吾何以知衆\bo{父}之\bo{然}哉。以此。

    \chapter{}

    曲則全。枉則\bo{正}。窪則盈。敝則新。少則得。多則惑。是以聖人\bo{執}一\bo{以}爲天下\bo{牧}。不自見故明。不自是故彰。不自伐故有功。不自矜故長。夫唯不爭。故\bo{\del{天下}}莫能與之爭。古之所謂曲則全者。豈\bo{\del{虚}}\bo{語}哉。誠全而歸之。

    \chapter{}

    希言自然。\bo{\del{故}}飄風不終朝。驟雨不終日。孰爲此者。\bo{\del{天地}}。天地尚不能久。而況於人乎。故從事\bo{而}\bo{\del{道者}}道者同於道。德者同於德。失者同於失。\bo{\del{同於道者。道亦樂得之}}。同於德者。\bo{道亦德之}。同於失者。\bo{道亦失之}。\bo{\del{信不足焉。有不信焉}}。

    \chapter{}

    企者不立。\bo{\del{跨者不行}}。自見者不明。自是者不彰。自伐者無功。自矜者不長。其在道也。曰餘食贅行。物或惡之。故有道者不處。

    \chapter{}

    有\jian{狀}混成。先天地生。寂\jian{\del{兮}}寥\jian{\del{兮}}。獨立不改。\jian{\del{周行而不殆}}。可以爲天下母。吾不知其名。字之曰道。强爲之名曰大。大曰逝。逝曰遠。遠曰反。\jian{\del{故}}道大。天大。地大。王亦大。域中有四大。而王居其一焉。人法地。地法天。天法道。道法自然。

    \chapter{}

    重爲輕根。靜爲躁君。是以\bo{君子}終日行。不離其輜重。雖有榮觀。燕處超然。奈何萬乘之主。而以身輕天下。輕則失本。躁則失君。

    \chapter{}

    善行無轍迹。善言無瑕讁。善數不用籌策。善閉無關楗而不可開。善結無繩約而不可解。是以聖人\bo{恒}善救人。而无棄人。\bo{物无棄財}。是謂襲明。故善人者。\bo{\del{不}}善人之師。不善人者。善人之資。不貴其師。不愛其資。雖智大迷。是謂\bo{眇要}。

    \chapter{}

    知其雄。守其雌。爲天下\bo{溪}。爲天下\bo{溪}。\bo{恒}德不離。\bo{恒德不離}。復歸於嬰兒。知其白。守其黑。爲天下式。爲天下式。\bo{恒}德不忒。\bo{恒德不忒}。復歸於無極。知其榮。守其辱。爲天下\bo{浴}。爲天下\bo{浴}。\bo{恒}德乃足。\bo{恒德乃足}。復歸於樸。樸散則爲器。聖人用之則爲官長。\bo{夫}大制不割。

    \chapter{}

    將欲取天下而爲之。吾見其不得已。天下神器。不可爲也。爲者敗之。執者失之。故物或行或隨。或歔或吹。或强或羸。或\bo{陪}或\bo{墮}。是以聖人去甚。去奢。去泰。

    \chapter{}

    以道佐人主者。不以兵强天下。其事好\jian{長}。

    師之所處。荆棘生焉。\bo{\del{大軍之後。必有凶年}}。

    善有果而已。不\jian{\del{敢}}以取强。果而勿矜。果而勿伐。果而勿驕。\jian{\del{果而不得已}}。\jian{是謂}果而勿强。

    物壯則老。是謂不道。不道早已。

    \chapter{}

    夫\bo{\del{佳}}兵者。不祥之器。物或惡之。故有道者不處。

    君子居則貴左。用兵則貴右。\jian{故曰}兵者。\jian{\del{不祥之器}}。非君子之器。不得已而用之。恬淡爲上。\jian{弗美也}。而美之者。是樂殺人。夫樂殺人者。則不可以得志於天下矣。\jian{故}吉事尚左。凶事尚右。\jian{是以}偏將軍居左。上將軍居右。言以喪禮處之。\jian{故}殺人之衆。以哀悲\jian{位}之。戰勝。以喪禮處之。

    \chapter{}

    道\bo{恒}無名。樸雖小。\jian{天地弗敢臣}。侯王若能守之。萬物將自賓。

    天地相合。以降甘露。民莫之令而自均。始制有名。名亦既有。夫亦將知止。知止\jian{所以}不殆。譬道之在天下。猶\jian{小}浴之於江海。

    \chapter{}

    知人者智。自知者明。勝人者有力。自勝者强。知足者富。强行者有志。不失其所者久。死而不亡者壽。

    \chapter{}

    \bo{\del{大}}道氾兮。其可左右。\bo{成功遂事而弗名有也}。\bo{萬物歸焉而不爲主}。\bo{則}\bo{恒}无欲。可名於小。萬物歸焉而不爲主。可名於大。\bo{是以聖人之能成大也}。以其\bo{\del{終}}不\bo{\del{自}}爲大。故能成\del{其}大。

    \chapter{}

    \jian{埶}大象。天下往。往而不害。安平太。樂與餌。過客止。\jian{故}道之出\bo{言}。淡呵其无味。視之不足見。聽之不足聞。\jian{而不可既也}。

    \chapter{}

    將欲歙之。必固張之。將欲弱之。必固强之。將欲\bo{去}之。必固\bo{與}之。將欲奪之。必固\bo{予}之。是謂微明。柔弱勝剛强。魚不可脱於淵。國之利器不可以示人。

    \chapter{}

    道\bo{恒}無爲\jian{\del{而無不爲}}。侯王\jian{\del{若}}能守之。\jian{而}萬物將自化。化而欲作。\jian{\del{吾}}將鎮之以無名之樸。\jian{\del{無名之樸}}。夫亦將\jian{知足}。\jian{知足}以靜。\jian{萬物}將自定。

    \chapter{}

    上德不德。是以有德。下德不失德。是以無德。上德無爲而無以爲。\bo{\del{下德爲之而有以爲}}。上仁爲之而無以爲。上義爲之而有以爲。上禮爲之而莫之應。則攘臂而扔之。故失道而後德。失德而後仁。失仁而後義。失義而後禮。夫禮者。忠信之薄而亂之首。前識者。道之華而愚之\bo{首}。是以大丈夫處其厚。不居其薄。處其實。不居其華。故去彼取此。

    \chapter{}

    昔之得一者。天得一以清。地得一以寧。神得一以靈。\bo{浴}得一以盈。\bo{\del{萬物得一以生}}。侯王得一以爲天下\bo{正}。其致之。天無\bo{已}清將恐裂。地無\bo{已}寧將恐發。神無\bo{已}靈將恐歇。\bo{浴}無\bo{已}盈將恐竭。\bo{\del{萬物無以生將恐滅}}。侯王無\bo{已}貴\bo{以}高將恐蹶。故\bo{必}貴以賤爲本。高以下爲基。是以侯王自謂孤寡不穀。此\bo{其賤之本}邪。非乎。故致數\bo{與}無\bo{與}。\bo{是故}不欲琭琭如玉。珞珞如石。

    \chapter{}

    反者。道之動。弱者。道之用。天下萬物生於有。有生於無。

    \chapter{}

    上士聞道。\jian{堇}\jian{能}行之。中士聞道。若存若亡。下士聞道。大笑之。不笑。不足以爲道。故建言有之。明道若昧。進道若退。夷道若纇。上德若\bo{浴}。大白若辱。廣德若不足。建德若偷。質真若渝。大方無隅。大器\jian{曼}成。大音希聲。大象無形。道\bo{襃}无名。夫唯道。\bo{善始且善成}。

    \chapter{}

    道生一。一生二。二生三。三生萬物。萬物負陰而抱陽。沖氣以爲和。人之所惡。唯孤寡不穀。而王公\bo{以自名也}。\bo{\del{故}}物或損之而益。\bo{\del{或}}益之而損。\bo{故}人之所教。\bo{亦我而教人}。\bo{故}强梁者不得其死。吾將以爲教父。

    \chapter{}

    天下之至柔。馳騁\bo{於}天下之至堅。無有入\bo{於}無\bo{間}。吾是以知無爲之有益。不言之教。無爲之益。天下希\bo{能}及之。

    \chapter{}

    名與身孰親。身與貨孰多。得與亡孰病。\jian{\del{是故}}甚愛必大費。多藏必厚亡。\jian{故}知足不辱。知止不殆。可以長久。

    \chapter{}

    大成若缺。其用不弊。大盈若沖。其用不窮。大直若屈。大巧若拙。\jian{大成若詘}。

    \chapter{}

    天下有道。卻走馬以糞。天下無道。戎馬生於郊。

    \jian{罪莫厚於甚欲}。禍莫大於不知足。咎莫\bo{憯}於欲得。\jian{\del{故}}\jian{知足之爲足}。\jian{此}\bo{恒}足矣。

    \chapter{}

    不出户。\bo{以}知天下。不闚牖。\bo{以}知天道。其出彌遠。其知彌少。是以聖人不行而知。不見而名。不爲而成。

    \chapter{}

    爲學日益。爲道日損。損之又損。以至於無爲。無爲而無不爲。

    取天下\bo{恒}以無事。及其有事。不足以取天下。

    \chapter{}

    聖人\bo{恒无心}。以百姓心爲心。善者\bo{\del{吾}}善之。不善者\bo{\del{吾}}亦善之。德善。信者\bo{\del{吾}}信之。不信者\bo{\del{吾}}亦信之。德信。聖人在天下歙歙。爲天下渾其心。\bo{百姓皆注其耳目焉}。聖人皆孩之。

    \chapter{}

    出生入死。生之徒十有三。死之徒十有三。\bo{而民生生。動皆之死地之十有三}。夫何故。以其\bo{生生}。蓋聞善攝生者。\bo{陵}行不\bo{辟}兕虎。入軍不被甲兵。兕無所投其角。虎無所措其爪。兵無所容其刃。夫何故。以其無死地。

    \chapter{}

    道生之。德畜之。物形之。\bo{器}成之。是以萬物莫不尊道而貴德。道之尊。德之貴。夫莫之\bo{爵}而\bo{恒}自然。\bo{\del{故}}道生之。\bo{\del{德}}畜之。長之。育之。亭之。毒之。養之。覆之。生而不有。爲而不恃。長而不宰。是謂玄德。

    \chapter{}

    天下有始。以爲天下母。既得其母。以知其子。既知其子。復守其母。没身不殆。

    塞其兑。閉其門。終身不勤。開其兑。濟其事。終身不救。

    見小曰明。守柔曰强。用其光。復歸其明。無遺身殃。是爲\bo{襲}常。

    \chapter{}

    使我\bo{介}有知。行於大道。唯施是畏。大道甚夷。而民好徑。朝甚除。田甚蕪。倉甚虚。服文采。帶利劍。厭飲食。財貨有餘。是謂盗夸。非道也哉。

    \chapter{}

    善建者不拔。善抱者不脱。子孫以祭祀不輟。修之於身。其德乃真。修之於家。其德乃餘。修之於鄉。其德乃長。修之於國。其德乃豐。修之於天下。其德乃普。\bo{\del{故}}以身觀身。以家觀家。以鄉觀鄉。以國觀國。以天下觀天下。吾何以知天下\bo{之}然哉。以此。

    \chapter{}

    含德之厚\jian{者}。比於赤子。蜂蠆虺蛇不螫。\bo{攫鳥猛獸不搏}。骨弱筋柔而握固。未知牝牡之合而\bo{朘怒}。精之至也。終日號而不\bo{嚘}。和之至也。\jian{\del{知}}和曰常。知\jian{和}曰明。益生曰祥。心使氣曰强。物壯則老。謂之不道。\jian{\del{不道早已}}。

    \chapter{}

    知者不言。言者不知。塞其兑。閉其門。挫其銳。解其\jian{紛}。和其光。同其塵。是謂玄同。故不可得而親。\jian{亦}不可得而\bo{疏}。不可得而利。\jian{亦}不可得而害。不可得而貴。\jian{亦}不可得而賤。故爲天下貴。

    \chapter{}

    以正治國。以奇用兵。以無事取天下。吾何以知其然哉。\jian{\del{以此}}。天下多忌諱。而民彌\jian{畔}。民多利器。國家滋昏。人多\bo{知}。奇物滋起。\bo{法物}滋彰。盗賊多有。故聖人云。我無爲而民自化。我好靜而民自正。我無事而民自富。我\bo{欲不欲}而民自樸。

    \chapter{}

    其政悶悶。其民淳淳。其政察察。其民缺缺。禍兮福之所倚。福兮禍之所伏。孰知其極。其無正。正復爲奇。善復爲妖。人之迷。其日固久。是以\bo{\del{聖人}}方而不割。廉而不劌。直而不肆。光而不燿。

    \chapter{}

    治人事天莫若嗇。夫唯嗇。是以早服。早服謂之重積德。重積德則無不克。無不克則莫知其極。莫知其極可以有國。有國之母。可以長久。是謂深根固柢。長生久視之道。

    \chapter{}

    治大國若烹小鮮。以道莅天下。其鬼不神。非其鬼不神。其神不傷人。非其神不傷人。聖人亦不傷人。夫兩不相傷。故德交歸焉。

    \chapter{}

    大國者下流。天下之\bo{牝}。天下之\bo{交}。牝\bo{恒}以靜勝牡。\bo{爲其静也。故宜爲下}。故大國以下小國。則取小國。小國以下大國。則取\bo{於}大國。故或下以取。或下而取。\bo{故}大國不過欲兼畜人。小國不過欲入事人。夫兩者各得其所欲。\bo{則}大者宜爲下。

    \chapter{}

    道者萬物之\bo{注}。善人之寶。不善人之所保。美言可以市。尊行可以加人。人之不善。何棄之有。故立天子。置三公。雖有拱璧以先駟馬。不如\bo{坐而進此}。古之所以貴\bo{此}者何。不曰\bo{求以得}。有罪以免邪。故爲天下貴。

    \chapter{}

    爲無爲。事無事。味無味。大小。多少。

    報怨以德。圖難於其易。爲大於其細。\bo{天下之難}\bo{\del{必}}作於易。\bo{天下之大}\bo{\del{必}}作於細。是以聖人終不爲大。故能成其大。

    夫輕諾必寡信。多易必多難。是以聖人猶難之。故終無難矣。

    \chapter{}

    其安易持。其未兆易謀。其脆易泮。其微易散。爲之於未有。治之於未亂。合抱之木。生於毫末。九層之臺。起於累土。千里之行。始於足下。爲者敗之。執者失之。是以聖人無爲故無敗。無執故無失。民之從事。常於幾成而敗之。慎終如始。則無敗事。是以聖人欲不欲。不貴難得之貨。學不學。復衆人之所過。以輔萬物之自然。而不敢爲。

    其安易持。其未兆易謀。其脆易泮。其微易散。爲之於未有。治之於未亂。合抱之木。生於毫末。九層之臺。起於累土。\bo{百仁之高}。始於足下。

    爲者敗之。執者失之。是以聖人無爲故無敗。無執故無失。民之從事。\bo{恒}於幾成而敗之。慎終如始。則無敗事。

    \jian{\del{是以}}聖人欲不欲。不貴難得之貨。學不學。復衆人之所過。\jian{是故}\jian{能}輔萬物之自然而不\jian{能}爲。

    \chapter{}

    古之\bo{\del{善}}爲道者。非以明民。將以愚之。民之難治。以其智\bo{\del{多}}\bo{也}。故以智治國。國之賊。\bo{以不智}治國。國之\bo{德}。\bo{恒}知此兩者亦\he{楷式}。\bo{恒}知\he{楷式}。是謂玄德。玄德深矣。遠矣。與物反矣。然後乃至大順。

    \chapter{}

    江海所以能爲百\bo{浴}王者。以其\jian{能爲百浴下}。故能爲百\bo{浴}王。\jian{聖人之在民前也。以身後之}。\jian{其在民上也。以言下之}。\jian{\del{是以}}\jian{\del{聖人}}處上而民不重。處前而民不害。\jian{\del{是以}}天下樂\jian{進}而不厭。以其不爭。故天下莫能與之爭。

    \chapter{}

    天下皆謂我\bo{\del{道}}大。\bo{大而不肖}。夫唯大。故\bo{\del{似}}不肖。若肖。久矣其細也夫。我\bo{恒}有三寶。持而寶之。一曰慈。二曰儉。三曰不敢爲天下先。慈。故能勇。儉。故能廣。不敢爲天下先。故能\bo{爲}成器長。今舍慈且勇。舍儉且廣。舍後且先。死矣。夫慈。以戰則勝。以守則固。天將\bo{建}之。\bo{如}以慈衛之。

    \chapter{}

    善爲士者不武。善戰者不怒。善勝敵者不與。善用人者爲之下。是謂不爭之德。是謂\bo{用人}。是謂配天。古之極\bo{也}。

    \chapter{}

    用兵有言。吾不敢爲主而爲客。不敢進寸而退尺。是謂行無行。攘無臂。\bo{執無兵。乃无敵}。禍莫大於\bo{無敵}。\bo{無敵}幾喪吾寶。故抗兵相\bo{若}。\bo{則}哀者勝矣。

    \chapter{}

    吾言甚易知。甚易行。\bo{而}天下莫能知。莫能行。言有宗。事有君。夫唯無知。是以不我知。知我者希。\bo{則我貴矣}。是以聖人被褐懷玉。

    \chapter{}

    知不知。上。不知知。病。\bo{\del{夫唯病病。是以不病}}。\bo{是以}聖人不病。以其病病。是以不病。

    \chapter{}

    民不畏威。則大威\bo{將}至。無狎其所居。無厭其所生。夫唯不厭。是以不厭。是以聖人自知不自見。自愛不自貴。故去彼取此。

    \chapter{}

    勇於敢則殺。勇於不敢則活。此兩者或利或害。天之所惡。孰知其故。\bo{\del{是以聖人猶難之}}。天之道。不戰而善勝。不言而善應。不召而自來。\bo{繟}而善謀。天網恢恢。疏而不失。

    \chapter{}

    \bo{若民恒且不畏死}。奈何以\bo{殺}懼之。若民\bo{恒}且畏死。而爲奇者吾得而殺之。孰敢。\bo{若民恒且必畏死。則恒有司殺者}。夫代司殺者殺。是謂代大匠斲。夫代大匠斲者。\bo{則}希有不傷其手矣。

    \chapter{}

    \bo{人}之饑。以其\bo{取}食税之多。是以饑。\bo{百姓之不治}。以其上之\bo{有以爲}\bo{也}。是以\bo{不治}。民之輕死。以其求生之厚。是以輕死。夫唯無以生爲者。是賢\bo{\del{於}}貴生。

    \chapter{}

    人之生也柔弱。其死也堅强。萬物草木之生也柔脆。其死也枯槁。故\bo{曰}堅强者死之徒。柔弱者生之徒。是以兵强則不勝。木强則\he{共}。强大處下。柔弱處上。

    \chapter{}

    天之道。其猶張弓與。高者抑之。下者舉之。有餘者損之。不足者補之。\bo{故}天之道。損有餘而補不足。人之道。損不足而奉有餘。\bo{孰能有餘而有以取奉於天者}。唯有道者。是以聖人爲而不\bo{又}。功成而不處。\bo{若此}。其不欲見賢也。

    \chapter{}

    天下莫柔弱於水。而攻堅强者莫之能勝。\bo{以其無以易之也}。弱之勝强。柔之勝剛。天下莫不知。\bo{而}莫能行。是以聖人云。受國之垢。是謂社稷主。受國不祥。是爲天下王。正言若反。

    \chapter{}

    和大怨。必有餘怨。安可以爲善。是以聖人執左契。而不以責於人。\bo{故}有德司契。無德司徹。天道無親。\bo{恒}與善人。

    \chapter{}

    小國寡民。\bo{使有十百人之器而不用}。使民重死而\bo{\del{不}}遠徙。雖有舟輿。無所乘之。雖有甲兵。無所陳之。使人復結繩而用之。甘其食。美其服。安其居。樂其俗。鄰國相望。鷄犬之聲相聞。民至老死不相往來。

    \chapter{}

    信言不美。美言不信。善者不\bo{多}。\bo{多}者不善。知者不博。博者不知。聖人\bo{无}積。既以爲人。己愈有。既以與人。己愈多。\bo{故}天之道。利而不害。\bo{\del{聖}}人之道。爲而不爭。

    \chapter*{参考文献}

    《老子古本合校》,楊丙安 著,中華書局。

    《老子道德經注校釋》,樓宇烈 校釋,中華書局。

    《老子道德經河上公章句》,王卡 點校,中華書局。

    《帛書老子校注》,高明 撰,中華書局。

    《郭店楚簡老子集釋》,彭裕商、吴毅强 集釋,巴蜀書社。

\end{document}
