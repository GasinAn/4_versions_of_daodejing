\documentclass[a5paper]{ctexbook}
\usepackage{geometry}
\usepackage{hyperref}
\usepackage{ulem}
\usepackage{xcolor}
\geometry{a5paper}
\newcommand{\he}[1]{\textcolor[RGB]{255,0,255}{#1}}
\newcommand{\bo}[1]{\textcolor[RGB]{0,0,255}{#1}}
\newcommand{\jian}[1]{\textcolor[RGB]{0,255,0}{#1}}
\def\del{\sout}
\title{《道德经》整理}
\author{安梯西登}
\date{}
\begin{document}

    \maketitle

    \tableofcontents

    \chapter{}

    道可道。非恒道。名可名。非恒名。無名萬物之始。有名萬物之母。故恒無欲。以觀其眇。恒有欲。以觀其所徼。兩者同出。異名同謂。玄之又玄。衆眇之門。

    \chapter{}

    天下皆知美之爲美。惡已。皆知善。此其不善已。有無相生。難易相成。長短相刑。高下相盈。音聲相和。先後相隨。是以聖人處無爲之事。行不言之教。萬物作而不始。爲而不志。成而弗居。夫唯弗居。是以不去。

    \chapter{}

    不尚賢。使民不爭。不貴難得之貨。使民不爲盗。不見可欲。使民不亂。是以聖人之治。虚其心。實其腹。弱其志。强其骨。常使民無知。無欲。使夫智不敢。不爲而已。則無不治。

    \chapter{}

    道沖而用之有不盈。淵兮似萬物之宗。挫其銳。解其紛。和其光。同其塵。湛兮似或存。吾不知其誰之子也。象帝之先。

    \chapter{}

    天地不仁。以萬物爲芻狗。聖人不仁。以百姓爲芻狗。
    
    天地之間。其猶橐籥乎。虚而不屈。動而愈出。
    
    多聞數窮。不如守於中。

    \chapter{}

    浴神不死。是謂玄牝。玄牝之門。是謂天地之根。緜緜若存。用之不勤。

    \chapter{}

    天長地久。天地之所以能長且久者。以其不自生。故能長生。是以聖人退其身而身先。外其身而身存。非以其無私邪。故能成其私。

    \chapter{}

    上善若水。水善利萬物而不爭。處衆人之所惡。故幾於道。居善地。心善淵。與善天。言善信。正善治。事善能。動善時。夫唯不爭。故無尤。

    \chapter{}

    持而盈之。不如其已。揣而棁(兑、允、羣)之。不可長保。金玉盈室。莫能守也。貴富而驕。自遺咎也。功遂身退。天之道也。

    \chapter{}

    載營魄抱一。能毋離乎。專氣致柔。能嬰兒乎。滌除玄監。能毋疵乎。愛民治國。能毋以知乎。天門開闔。能爲雌乎。明白四達。能毋以知乎。生之。畜之。生而不有。長而不宰。是謂玄德。

    \chapter{}

    三十輻共一轂。當其無。有車之用。埏埴以爲器。當其無。有埴器之用。鑿户牖。當其無。有室之用。故有之以爲利。無之以爲用。

    \chapter{}

    五色令人目盲。五音令人耳聾。五味令人口爽。馳騁畋獵令人心發狂。難得之貨令人行妨。是以聖人之治也。爲腹而不爲目。故去彼取此。

    \chapter{}

    寵辱若驚。貴大患若身。何謂寵辱。寵爲下。得之若驚。失之若驚。是謂寵辱若驚。何謂貴大患若身。吾所以有大患者。爲吾有身。及吾無身。有何患。故貴爲身於爲天下。若可託天下。愛以身爲天下。若可寄天下。

    \chapter{}

    視之不見名曰微。聽之不聞名曰希。捪之不得名曰夷。此三者不可致詰。故混而爲一。一者。其上不皦。其下不昧。繩繩呵不可名。復歸於無物。是謂無狀之狀。無物之象。是謂惚恍。迎之不見其首。隨之不見其後。執今之道。以御今之有。能(以)知古始。是謂道紀。

    \chapter{}

    古之善爲士者。必微妙玄達。深不可志。夫唯不可志。是以爲之頌。豫焉若冬涉川。猶兮若畏四鄰。儼兮其若客。涣兮若淩釋。敦兮其若樸。混兮其若濁。孰能濁以靜者。將徐清。孰能安以動者。將徐生。保此道者。不欲尚盈。

    \chapter{}

    至虚極(恒)也。守中篤也。萬物並(方)作。居以須復也。天道員員。各復其根。

    \chapter{}

    太上下知有之。其次親譽之。其次畏之。其次侮之。信不足。安有不信。悠兮其貴言。功成事遂。而百姓謂我自然。

    \chapter{}

    故大道廢。安有仁義。六親不和。安有孝慈。國家昏亂。安有正臣。

    \chapter{}

    絕智弃卞。民利百倍。絕爲棄慮。民復孝慈。絶巧棄利。盗賊無有。三言以爲史不足。故令之有所屬。視素保樸。少私寡欲。

    \chapter{}

    絶學無憂。
    
    唯與呵。相去幾何。美與惡。相去何若。人之所畏。亦不可以不畏人。
    
    荒兮其未央哉。衆人熙熙。如享(鄉)於太牢。而春登臺。我獨泊兮其未兆。如嬰兒之未咳。儽儽兮若無所歸。衆人皆有餘。我獨遺。我愚人之心也。沌沌兮。俗人昭昭。我獨昏昏。俗人察察。我獨悶悶。惚兮其若海。恍兮若無所止。衆人皆有以。而我獨頑以鄙。我欲獨異於人。而貴食母。

    \chapter{}

    孔德之容。惟道是從。道之物。惟恍惟惚。惚兮恍兮。中有象呵。恍兮惚兮。中有物呵。窈兮冥兮。其中有請。其請甚真。其中有信。自今及古。其名不去。以順衆父。吾何以知衆父之然哉。以此。

    \chapter{}

    曲則全。枉則正。窪則盈。敝則新。少則得。多則惑。是以聖人執一。以爲天下牧。不自是(視)故彰。不自見故明。不自伐故有功。不矜故長。夫唯不爭。故莫能與之爭。古之所謂曲全者。豈語哉。誠全歸之。

    \chapter{}

    希言自然。飄風不終朝。暴雨不終日。孰爲此者。天地尚不能久。有況於人乎。故從事而道者同於道。德者同於德。失者同於失。同於德者。道亦德之。同於失者。道亦失之。

    \chapter{}

    企者不立。自是(視)不彰。自見者不明。自伐者無功。自矜者不長。其在道也。曰餘食贅行。物或惡之。故有道(欲)者不處。

    \chapter{}

    有狀混成。先天地生。寂兮寥兮。獨立不亥。可以爲天下母。未知其名。字之曰道。吾强爲之名曰大。大曰逝。逝曰遠。遠曰反。道大。天大。地大。王亦大。域中有四大。王居一焉。人法地。地法天。天法道。道法自然。

    \chapter{}

    重爲輕根。靜爲躁君。是以聖人終日行。不離其輜重。雖有榮觀。燕處超然。奈何萬乘之主。而以身輕天下。輕則失本。躁則失君。

    \chapter{}

    善行無轍迹。善言無瑕讁。善數不用籌策。善閉無關楗而不可開。善結無繩約而不可解。是以聖人恒善救人。而无棄人。物无棄財。是謂襲明。故善人者。善人之師。不善人者。善人之資。不貴其師。不愛其資。雖智大迷。是謂眇要。

    \chapter{}

    知其雄。守其雌。爲天下谿。爲天下谿。恒德不離。恒德不離。復歸於嬰兒。知其白。守其黑。爲天下式。爲天下式。常恒不忒。常恒不忒。復歸於無極。知其榮。守其辱。爲天下浴。爲天下浴。常恒乃足。常恒乃足。復歸於樸。樸散則爲器。聖人用則爲官長。夫大制不割。

    \chapter{}

    將欲取天下而爲之。吾見其不得已。天下神器。非可爲者也。爲者敗之。執者失之。故物或行或隨。或歔或吹。或强或羸。或陪或墮。是以聖人去甚。去奢。去泰。

    \chapter{}

    以道佐人主者。不以兵强於天下。其事好還。
    
    師之所處。荆棘生焉。
    
    善有果而已。不敢以取强。果而勿矜。果而勿伐。果而勿驕。果而勿得已居。是謂果而不强

    物壯則老。是謂不道。不道早已。

    \chapter{}

    夫兵者。不祥之器。物或惡之。故有道者不處。

    君子居則貴左。用兵則貴右。故曰兵者。非君子之器。不得已而用之。恬淡爲上。弗美也。美之。是樂殺人。夫樂殺人。不可以得志於天下矣。是以吉事尚左。凶事尚右。是以偏將軍居左。上將軍居右。言以喪禮處之。故殺人衆。則以哀悲位之。戰勝喪禮處之。

    \chapter{}

    道恒無名。樸雖妻。天地弗敢臣。侯王若能守之。萬物將自賓。

    天地相合。以俞甘露。民莫之令而自均。始制有名。名亦既有。夫亦將知止。知止所以不殆。譬道之在天下。猶小浴之於江海。

    \chapter{}

    知人者智。自知者明。勝人者有力。自勝者强。知足者富。强行者有志。不失其所者久。死而不亡者壽。

    \chapter{}

    道氾兮。其可左右。成功遂事而弗名有也。萬物歸焉而弗爲主。則恒无欲也。可名於小。萬物歸焉而弗爲主。可名於大。是以聖人之能成大也。以其不爲大也。故能成大。

    \chapter{}

    埶(執)大象。天下往。往而不害。安平太。樂與餌。過客止。故道之出言。淡呵其无味。視之不足見。聽之不足聞。而不可既也。

    \chapter{}

    將欲歙之。必固張之。將欲弱之。必固强之。將欲去之。必固與之。將欲奪之。必固予之。柔弱勝强。魚不可脱於淵。國利器不可以示人。

    \chapter{}

    道恒無爲也。侯王能守之。而萬物將自化。化而欲作。將鎮之以無名之樸。夫亦將智足。智足以靜。萬物將自定。

    \chapter{}

    上德不德。是以有德。下德不失德。是以無德。上德無爲而無以爲。上仁爲之而無以爲。上義爲之而有以爲。上禮爲之而莫之應。則攘臂而扔之。故失道而後德。失德而後仁。失仁而後義。失義而後禮。夫禮者。忠信之薄而亂之首。前識者。道之華而愚之首。是以大丈夫處其厚。不居其薄。處其實。不居其華。故去彼取此。

    \chapter{}

    昔之得一者。天得一以清。地得一以寧。神得一以靈。浴得一以盈。侯王得一以爲天下正。其致之。天毋已清將恐裂。地毋已寧將恐發。神毋已靈將恐歇。浴毋已盈將恐竭。侯王毋已貴以高將恐蹶。故必貴以賤爲本。必高以下爲基。是以侯王自謂孤寡不穀。此其賤之本邪。非乎。故致數與无與。是故不欲琭琭如玉。珞珞如石。

    \chapter{}

    反者。道之動。弱者。道之用。天下之物生於有。有生於無。

    \chapter{}

    上士聞道。堇能行於其中。中士聞道。若存(聞)若亡。下士聞道。大笑之。弗大笑。不足以爲道。故建言有之。明道若昧。進道若退。夷道若纇。上德若浴。大白若辱。廣德若不足。建德若偷。質真(貞)若渝(愉)。大方無隅。大器曼成。大音希聲。大象無形。道襃无名。夫唯道善始且善成。

    \chapter{}

    道生一。一生二。二生三。三生萬物。萬物負陰而抱陽。沖氣以爲和。人之所惡。唯孤寡不穀。而王公以自名也。物或損之而益。益之而損。故人之所教。亦我而教人。故强梁者不得其死。吾將以爲教父。

    \chapter{}

    天下之至柔。馳騁於天下之至堅。無有入於無間。吾是以知無爲之有益。不言之教。無爲之益。天下希能及之。

    \chapter{}

    名與身孰親。身與貨孰多。得與亡孰病。甚愛必大費。多藏必厚亡。故知足不辱。知止不殆。可以長久。

    \chapter{}

    大成若缺。其用不幣(弊)。大盈若沖。其用不窮。大直若屈。大巧若拙。大辯若訥(大贏如㶧、大成若詘)。

    躁勝寒。靜勝熱。清靜爲天下正。

    \chapter{}

    天下有道。卻走馬以糞。天下無道。戎馬生於郊。
    
    罪莫厚於可欲。咎莫僉於欲得。禍莫大於不知足。知足之爲足。此恒足矣。

    \chapter{}

    不出於户。以知天下。不闚於牖。以知天道。其出彌遠。其知彌少。是以聖人不行而知。不見而名。不爲而成。

    \chapter{}

    爲學者日益。爲道者日損。損之又損。以至於無爲。無爲而無不爲。
    
    取天下恒以無事。及其有事。不足以取天下。

    \chapter{}

    聖人恒无心。以百姓心爲心。善者善之。不善者亦善之。德善。信者信之。不信者亦信之。德信。聖人在天下歙歙。爲天下渾其心。百姓皆注其耳目焉。聖人皆孩之。

    \chapter{}

    出生入死。生之徒十有三。死之徒十有三。而民生生。動皆之死地之十有三。夫何故。以其生生也。蓋聞善攝生者。陵行不辟兕虎。入軍不被甲兵。兕無所投其角。虎無所措其爪。兵無所容其刃。夫何故。以其無死地焉。

    \chapter{}

    道生之。德畜之。物形之。器成之。是以萬物莫不尊道而貴德。道之尊。德之貴。夫莫之爵而恒自然。道生之。(德)畜之。長之。育之。亭之。毒之。養之。覆之。生而不有。爲而不恃。長而不宰。是謂玄德。

    \chapter{}

    天下有始。以爲天下母。既得其母。以知其子。既知其子。復守其母。没身不殆。
    
    塞其兑。閉其門。終身不勤(嵍)。開其兑。濟(塞)其事。終身不救(逨)。
    
    見小曰明。守柔曰强。用其光。復歸其明。毋遺身殃。是謂襲常。

    \chapter{}

    使我介有知。行於大道。唯施是畏。大道甚夷。而民好徑。朝甚除。田甚蕪。倉甚虚。服文采。帶利劍。厭飲食。而齎財有餘。是謂盗夸。非道也哉。

    \chapter{}

    善建者不拔。善抱(伂)者不脱。子孫以祭祀不輟(乇)。修之身。其德乃真(貞)。修之家。其德乃餘。修之鄉。其德乃長。修之國。其德乃豐。修之天下。其德乃博。以身觀身。以家觀家。以鄉觀鄉。以國觀國。以天下觀天下。吾何以知天下之然哉。以此。

    \chapter{}

    含德之厚者。比於赤子。蜂蠆虺蛇不螫。攫鳥猛獸不扣。骨弱筋柔而捉固。未知牝牡之合而朘怒。精之至也。終日唬而不憂。和之至也。和曰常(同)。知和曰明。益生曰祥。心使氣曰强。物壯則老。是謂不道。

    \chapter{}

    知者不言。言者不知。塞其兑。閉其門。挫其銳。解其紛。和其光。同其塵。是謂玄同。故不可得而親。亦不可得而疏。不可得而利。亦不可得而害。不可得而貴。亦不可得而賤。故爲天下貴。

    \chapter{}

    以正治國。以奇用兵。以無事取天下。吾何以知其然哉。天下多忌諱。而民彌畔。民多利器。而國滋昏。人多知。而奇物滋起。法物滋彰。而盗賊多有。是以聖人之言曰。我無爲而民自化。我好靜而民自正。我無事而民自富。我欲不欲而民自樸。

    \chapter{}

    其政悶悶。其民淳淳。其政察察。其民缺缺。禍兮福之所倚。福兮禍之所伏。孰知其極。其無正。正復爲奇。善復爲妖。人之迷。其日固久。是以(聖人)方而不割。廉而不劌。直而不肆。光而不燿。

    \chapter{}

    治人事天莫若嗇。夫唯嗇。是以早服(備)。早服(備)是謂重積德。重積德則無不克。無不克則莫知其極。莫知其極可以有國。有國之母。可以長久。是謂深根固柢。長生久視之道。

    \chapter{}

    治大國若烹小鮮。以道莅天下。其鬼不神。非其鬼不神也。其神不傷人也。非其神不傷人也。聖人亦不傷也。夫兩不相傷。故德交歸焉。

    \chapter{}

    大國者下流。天下之牝。天下之交。牝恒以靜勝牡。爲其静也。故宜爲下。故大國以下小國。則取小國。小國以下大國。則取於大國。故或下以取。或下而取。故大國不過欲兼畜人。小國不過欲入事人。夫皆得其欲。則大者宜爲下。

    \chapter{}

    道者萬物之注。善人之寶。不善人之所保。美言可以市。尊行可以加人。人之不善。何棄之有。故立天子。置三卿。雖有拱璧以先駟馬。不如坐而進此。古之所以貴此者何。不曰求以得。有罪以免邪。故爲天下貴。

    \chapter{}

    爲無爲。事無事。味無味。大小。多少。
    
    報怨以德。圖難於其易。爲大於其細。天下之難作於易。天下之大作於細。是以聖人終不爲大。故能成其大。

    夫輕諾必寡信。多易必多難。是以聖人猶難之。故終無難矣。

    \chapter{}

    其安易持。其未兆易謀。其脆易泮。其微易散。爲之於未有。治之於未亂。合抱之木。生於毫末。九層之臺。起於累土。百仁之高。始於足下。

    爲者敗之。執者失之。是以聖人無爲故無敗。無執故無失。民之從事。恒於幾成而敗之。慎終如始。則無敗事。

    聖人欲不欲。不貴難得之貨。學不學。復衆之所過。是以能輔萬物之自然而不能爲。

    \chapter{}

    古之爲道者。非以明民。將以愚之。民之難治。以其智也。故以智治國。國之賊。以不智治國。國之德。恒知此兩者。亦稽式也。恒知稽式。是謂玄德。玄德深矣。遠矣。與物反矣。乃至大順。

    \chapter{}

    江海所以爲百浴王。以其能爲百浴下。是以能爲百浴王。聖人之在民前也。以身後之。其在民上也。以言下之。其在民上也。民弗厚也。其在民前也。民弗害也。天下樂進而弗詀。

    \chapter{}

    天下皆謂我道大。大而不肖。夫唯大。故不肖。若肖。久矣其細也夫。我恒有三寶。持而寶之。一曰慈。二曰儉。三曰不敢爲天下先。慈。故能勇。儉。故能廣。不敢爲天下先。故能爲成器長。今舍慈且勇。舍儉且廣。舍後且先。死矣。夫慈。以戰則勝。以守則固。天將建之。如以慈垣之。

    \chapter{}

    善爲士者不武。善戰者不怒。善勝敵者不與。善用人者爲之下。是謂不爭之德。是謂用人。是謂配天。古之極也。

    \chapter{}

    用兵有言。吾不敢爲主而爲客。不敢進寸而退尺。是謂行無行。攘無臂。執無兵。乃无敵。禍莫大於無敵。無敵近亡吾寶。故抗兵相若。而哀者勝矣。

    \chapter{}

    吾言甚易知。甚易行。而天下莫能知。莫能行。言有宗。事有君。夫唯無知。是以不我知。知我者希。則我貴矣。是以聖人被褐懷玉。

    \chapter{}

    知不知上。不知知病。是以聖人之不病。以其病病。是以不病。

    \chapter{}

    民不畏威。則大威將至。毋狎其所居。毋厭其所生。夫唯弗厭。是以不厭。是以聖人自知不自見。自愛不自貴。故去彼取此。

    \chapter{}

    勇於敢則殺。勇於不敢則活。此兩者或利或害。天之所惡。孰知其故。天之道。不戰而善勝。不言而善應。不召而自來。繟而善謀。天網恢恢。疏而不失。

    \chapter{}

    若民恒且畏不畏死。奈何以殺懼之。若民恒且畏死。而爲奇者吾得而殺之。孰敢。若民恒且必畏死。則恒有司殺者。夫代司殺者殺。是謂代大匠斲。夫代大匠斲者。則希不傷其手矣。

    \chapter{}

    人之饑。以其取食税之多。是以饑。百姓之不治。以其上之有以爲也。是以不治。民之輕死。以其求生之厚。是以輕死。夫唯無以生爲者。是賢貴生。

    \chapter{}

    人之生也柔弱。其死也堅强。萬物草木之生也柔脆。其死也枯槁。故曰堅强者死之徒。柔弱者生之徒。是以兵强則不勝。木强則競。强大處下。柔弱處上。

    \chapter{}

    天之道。猶張弓也。高者抑之。下者舉之。有餘者損之。不足者補之。故天之道。損有餘而益不足。人之道。損不足而奉有餘。孰能有餘而有以取奉於天者。唯有道者。是以聖人爲而不有。成功而不居。若此其不欲見賢也。

    \chapter{}

    天下莫柔弱於水。而攻堅强者莫之能勝。以其無以易之。柔(水)之勝剛。弱之勝强。天下莫不知也。而莫能行也。是以聖人之言云。受國之垢。是謂社稷主。受國不祥。是爲天下王。正言若反。

    \chapter{}

    和大怨。必有餘怨。安可以爲善。是以聖人執左(右)契。而不以責於人。故有德司契。無德司徹。天道無親。恒與善人。

    \chapter{}

    小國寡民。使有十百人之器而不用。使民重死而遠徙。有舟輿。無所乘之。有甲兵。無所陳之。使民復結繩而用之。甘其食。美其服。安其居。樂其俗。鄰國相望。鷄犬之聲相聞。民至老死不相往來。

    \chapter{}

    信言不美。美言不信。知者不博。博者不知。善者不多。多者不善。聖人无積。既以爲人。己愈有。既以與人。己愈多。故天之道。利而不害。人之道。爲而不爭。

    \chapter*{参考文献}

    《老子道德經注校釋》,樓宇烈 校釋,中華書局。

    《帛書老子校注》,高明 撰,中華書局。

    《郭店楚簡老子集釋》,彭裕商、吴毅强 集釋,巴蜀書社。

\end{document}
