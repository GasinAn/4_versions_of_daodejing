\documentclass[a5paper]{ctexbook}
\usepackage{geometry}
\usepackage{hyperref}
\geometry{a5paper}
\title{《道德经》整理}
\author{安梯西登}
\date{}
\begin{document}

    \maketitle

    \tableofcontents

    \chapter{}
    王弼本:

    道可道。非常道。名可名。非常名。無名天地之始。有名萬物之母。故常無欲。以觀其妙。常有欲。以觀其徼。此兩者同出而異名。同謂之玄。玄之又玄。衆妙之門。

    
    帛书甲本:

    道可道也。非恒道也。名可名也。非恒名也。无名萬物之始也。有名萬物之母也。〔故〕恒无欲也。以觀其眇。恒有欲也。以觀其所噭。兩者同出。異名同謂。玄之有玄。衆眇之〔門〕。

    帛书乙本:

    道可道也。〔非恒道也。名可名也。非〕恒名也。无名萬物之始也。有名萬物之母也。故恒无欲也。〔以觀其眇〕。恒又欲也。以觀其所噭。兩者同出。異名同謂。玄之又玄。衆眇之門。

    \chapter{}
    王弼本:

    天下皆知美之爲美。斯惡已。皆知善之爲善。斯不善已。故有無相生。難易相成。長短相較。高下相傾。音聲相和。前後相隨。是以聖人處無爲之事。行不言之教。萬物作焉而不辭。生而不有。爲而不恃。功成而弗居。夫唯弗居。是以不去。

    
    帛书甲本:

    天下皆知美之爲美。惡已。皆知善。斯不善矣。有无之相生也。難易之相成也。長短之相刑也。高下之相盈也。音聲之相和也。先後之相隨。恒也。是以聖人居无爲之事。行〔不言之教〕。〔萬物作而弗始〕也。爲而弗志也。成功而弗居也。夫唯居。是以弗去。

    帛书乙本:

    天下皆知美之爲美。惡已。皆知善。斯不善矣。〔有无之相〕生也。難易之相成也。長短之相刑也。高下之相盈也。音聲之相和也。先後之相隨。恒也。是以聖人居无爲之事。行不言之教。萬物作而弗始。爲而弗侍也。成功而弗居也。夫唯弗居。是以弗去。

    楚简甲本:

    天下皆智美之爲美也。惡已。皆智善。此其不善已。又亡之相生也。難易之相成也。長短之相型也。高下之相浧也。音聲之相和也。先後之相隨也。是以聖人居亡爲之事。行不言之学。萬勿作而弗怡也。爲而弗志也。成而弗居。天〈夫〉唯弗居也。是以弗去也。

    \chapter{}
    王弼本:

    不尚賢。使民不爭。不貴難得之貨。使民不爲盗。不見可欲。使民心不亂。是以聖人之治。虚其心。實其腹。弱其志。强其骨。常使民無知無欲。使夫智者不敢爲也。爲無爲。則無不治。

    
    帛书甲本:

    不上賢。〔使民不争。不貴難得之貨。使〕民不爲〔盗。不見可欲。使〕民不亂。是以聖人之〔治也。虚其心。實其腹。弱其志〕。强其骨。〔恒〕使民无知无欲也。使〔夫知不敢。弗爲而已。則无不治矣〕。

    帛书乙本:

    不上賢。使民不争。不貴難得之貨。使民不爲盗。不見可欲。使民不亂。是以聖人之治也。虚其心。實其腹。弱其志。强其骨。恒使民无知无欲也。使夫知不敢。弗爲而已。則无不治矣。

    \chapter{}
    王弼本:

    道沖而用之或不盈。淵兮似萬物之宗。挫其銳。解其紛。和其光。同其塵。湛兮似或存。吾不知誰之子。象帝之先。

    
    帛书甲本:

    〔道沖而用之有弗〕盈也。淵呵似萬物之宗。挫其銳。解其紛。和其光。同〔其塵〕。〔湛呵似〕或存。吾不知〔其誰之〕子也。象帝之先。

    帛书乙本:

    道沖而用之有弗盈也。淵呵似萬物之宗。挫其銳。解其紛。和其光。同其塵。湛呵似或存。吾不知其誰之子也。象帝之先。

    \chapter{}
    王弼本:

    天地不仁。以萬物爲芻狗。聖人不仁。以百姓爲芻狗。天地之間。其猶橐籥乎。虚而不屈。動而愈出。多言數窮。不如守中。

    
    帛书甲本:

    天地不仁。以萬物爲芻狗。聖人不仁。以百姓〔爲芻〕狗。天地〔之間。其〕猶橐籥與。虚而不淈。動而愈出。多聞數窮,不若守於中。

    帛书乙本:

    天地不仁。以萬物爲芻狗。聖人不仁。〔以〕百姓爲芻狗。天地之間。其猶橐籥與。虚而不淈。動而愈出。多聞數窮,不若守於中。

    楚简甲本:

    天地之間。其猶橐籥與。虛而不屈。動而愈出。

    \chapter{}
    王弼本:

    谷神不死。是謂玄牝。玄牝之門。是謂天地根。緜緜若存。用之不勤。

    
    帛书甲本:

    浴神〔不〕死。是謂玄牝。玄牝之門。是謂〔天〕地之根。緜緜呵若存。用之不堇。

    帛书乙本:

    浴神不死。是謂玄牝。玄牝之門。是謂天地之根。緜緜呵其若存。用之不堇。

    \chapter{}
    王弼本:

    天長地久。天地所以能長且久者。以其不自生。故能長生。是以聖人後其身而身先。外其身而身存。非以其無私邪。故能成其私。

    
    帛书甲本:

    天長地久。天地之所以能〔長〕且久者。以其不自生也。故能長生。是以聖人退其身而身先。外其身而身存。不以其无〔私〕與。故能成其私。

    帛书乙本:

    天長地久。天地之所以能長且久者。以其不自生也。故能長生。是以聖人退其身而身先。外其身而身先。外其身而身存。不以其无私與。故能成其私。

    \chapter{}
    王弼本:

    上善若水。水善利萬物而不爭。處衆人之所惡。故幾於道。居善地。心善淵。與善仁。言善信。正善治。事善能。動善時。夫唯不爭。故無尤。

    
    帛书甲本:

    上善治水。水善利萬物而有静。居衆之所惡。故幾於道矣。居善地。心善淵。予善信。正善治。事善能。動善時。夫唯不静。故无尤。

    帛书乙本:

    上善如水。水善利萬物而有争。居衆人之所惡。故幾於道矣。居善地。心善淵。予善天。言善信。正善治。事善能。動善時。夫唯不争。故无尤。

    \chapter{}
    王弼本:

    持而盈之。不如其已。揣而棁之。不可長保。金玉滿堂。莫之能守。富貴而驕。自遺其咎。功遂身退。天之道。

    
    帛书甲本:

    持而盈之。不〔若其已。揣而〕兑☐之。〔不〕可長保之。金玉盈室。莫之守也。貴富而驕。自遺咎也。功遂身退。天〔之道也〕。

    帛书乙本:

    持而盈之。不若其已。揣而允之。不可長保也。金玉〔盈〕室。莫之能守也。貴富而驕。自遺咎也。功遂身退。天之道也。

    楚简甲本:

    持而浧之。不不若已。湍而羣之。不可長保也。金玉浧室。莫能守也。貴富驕。自遺咎也。功遂身退。天之道也。

    \chapter{}
    王弼本:

    載營魄抱一。能無離乎。專氣致柔。能嬰兒乎。滌除玄覽。能無疵乎。愛民治國。能無知乎。天門開闔。能無雌乎。明白四達。能無爲乎。生之畜之。生而不有。爲而不恃。長而不宰。是謂玄德。

    
    帛书甲本:

    〔載營柏抱一。能毋离乎。槫氣至柔〕。能嬰兒乎。脩除玄藍。能毋疵乎。〔愛民栝國。能毋以知乎〕。⋯⋯。生之畜之。生而弗〔有。長而弗宰也。是謂玄〕德。

    帛书乙本:

    載營柏抱一。能毋离乎。槫氣至柔。能嬰兒乎。脩除玄監。能毋有疵乎。愛民栝國。能毋以知乎。天門啟闔。能爲雌乎。明白四達。能毋以知乎。生之畜之。生而弗有。長而弗宰也。是謂玄德。

    \chapter{}
    王弼本:

    三十輻共一轂。當其無。有車之用。埏埴以爲器。當其無。有器之用。鑿户牖以爲室。當其無。有室之用。故有之以爲利。無之以爲用。

    
    帛书甲本:

    卅〔輻同一轂。當〕其无。〔有車〕之用〔也〕。埏埴爲器。當其无。有埴器〔之用也〕。〔鑿户牖〕。當其无。有〔室之〕用也。故有之以爲利。无之以爲用。

    帛书乙本:

    卅輻同一轂。當其无。有車之用也。埏埴而爲器。當其无。有埴器之用也。鑿户牖。當其无。有室之用也。故有之以爲利。无之以爲用。

    \chapter{}
    王弼本:

    五色令人目盲。五音令人耳聾。五味令人口爽。馳騁畋獵令人心發狂。難得之貨令人行妨。是以聖人爲腹不爲目。故去彼取此。

    
    帛书甲本:

    五色使人目盲。馳騁田獵使人〔心發狂〕。難得之貨使人之行妨。五味使人之口爽。五音使人之耳聾。是以聖人之治也。爲腹不〔爲目〕。故去彼取此。

    帛书乙本:

    五色使人目盲。馳騁田獵使人心發狂。難得之貨使人之行妨。五味使人之口爽。五音使人耳〔聾〕。是以聖人之治也。爲腹而不爲目。故去彼而取此。

    \chapter{}
    王弼本:

    寵辱若驚。貴大患若身。何謂寵辱若驚。寵爲下。得之若驚。失之若驚。是謂寵辱若驚。何謂貴大患若身。吾所以有大患者。爲吾有身。及吾無身。吾有何患。故貴以身爲天下。若可寄天下。愛以身爲天下。若可託天下。

    
    帛书甲本:

    寵辱若驚。貴大患若身。何謂寵辱若驚。寵之爲下。得之若驚。失〔之〕若驚。是謂寵辱若驚。何謂貴大患若身。吾所以有大患者。爲吾有身也。及吾无身。有何患。故貴爲身於爲天下。若可以託天下矣。愛以身爲天下。如可以寄天下。

    帛书乙本:

    寵辱若驚。貴大患若身。何謂寵辱若驚。寵之爲下也。得之若驚。失之若驚。是謂寵辱若驚。何謂貴大患若身。吾所以有大患者。爲吾有身也。及吾無身。有何患。故貴爲身於爲天下。若可以託天下〔矣〕。愛以身爲天下。如可以寄天下矣。

    楚简乙本:

    寵辱若驚。貴大患若身。可謂寵辱。寵爲下也。得之若驚。失之若驚。是謂寵辱驚。☐☐☐☐☐若身。吾所以又大患者。爲吾又身。及吾亡身。或可☐☐☐☐☐☐爲天下。若可以託天下矣。愛以身爲天下。若可以寄天下矣。

    \chapter{}
    王弼本:

    視之不見名曰夷。聽之不聞名曰希。搏之不得名曰微。此三者不可致詰。故混而爲一。其上不皦。其下不昧。繩繩不可名。復歸於無物。是謂無狀之狀。無物之象。是謂惚恍。迎之不見其首。隨之不見其後。執古之道。以御今之有。能知古始。是謂道紀。

    
    帛书甲本:

    視之而弗見名之曰微。聽之而弗聞名之曰希。捪之而弗得名之曰夷。三者不可至詰。故混〔而爲一〕。一者。其上不皦。其下不昧。繩繩呵不可名也。復歸於无物。是謂无狀之狀。无物之〔象。是謂惚恍〕。〔隨而不見其後。迎〕而不見其首。執今之道。以御今之有。以知古始。是謂〔道紀〕。

    帛书乙本:

    視之而弗見〔名〕之曰微。聽之而弗聞命之曰希。捪之而弗得命之曰夷。三者不可至詰。故混而爲一。一者。其上不皦。其下不昧。繩繩呵不可命也。復歸於无物。是謂无狀之狀。无物之象。是謂惚恍。隨而不見其後。迎而不見其首。執今之道。以御今之有。以知古始。是謂道紀。

    \chapter{}
    王弼本:

    古之善爲士者。微妙玄通。深不可識。夫唯不可識。故强爲之容。豫焉若冬涉川。猶兮若畏四鄰。儼兮其若容。涣兮若冰之將釋。敦兮其若樸。曠兮其若谷。混兮其若濁。孰能濁以靜之徐清。孰能安以久動之徐生。保此道者不欲盈。夫唯不盈。故能蔽不新成。

    
    帛书甲本:

    〔古之善爲道者。微眇玄達〕。深不可志。夫唯不可志。故强爲之容。曰。豫呵其若冬〔涉水。猶呵其若〕畏四〔鄰。儼呵〕其若客。涣呵其若淩釋。敦呵其若樸。混〔呵其若濁。曠呵其〕若浴。濁而情之余清。女以動之余生。保此道不欲盈。夫唯不欲〔盈。是以能敝而不〕成。

    帛书乙本:

    古之善爲道者。微眇玄達。深不可志。夫唯不可志。故强爲之容。曰。豫呵其若冬涉水。猶呵其若畏四鄰。儼呵其若客。涣呵其若淩釋。敦呵其若樸。混呵其若濁。曠呵其若浴。濁而静之徐清。女以動之徐生。保此道〔不〕欲盈。是以能敝而不成。

    楚简甲本:

    古之善爲士者。必非溺玄達。深不可志。是以爲之頌。豫乎奴冬涉川。猶乎其奴畏四鄰。儼乎其奴客。渙乎其奴釋。屯乎其奴樸。混乎其奴濁。竺能濁以朿者。將舍清。竺能庀以迬者。將舍生。保此道者。不谷尚呈。

    \chapter{}
    王弼本:

    致虚極。守靜篤。萬物並作。吾以觀復。夫物芸芸。各復歸其根。歸根曰靜。是謂復命。復命曰常。知常曰明。不知常。妄作。凶。知常容。容乃公。公乃王。王乃天。天乃道。道乃久。没身不殆。

    
    帛书甲本:

    至虚極也。守静篤也。萬物旁作。吾以觀其復也。天物芸芸。各復歸於其〔根〕。〔曰静〕。静。是謂復命。復命常也。知常明也。不知常。㠵。㠵作。兇。知常容。容乃公。公乃王。王乃天。天乃道。〔道乃久〕。沕身不怠。

    帛书乙本:

    至虚極也。守静篤也。萬物旁作。吾以觀其復也。天物芸芸。各復歸於其根。曰静。静。是謂復命。復命常也。知常明也。不知常。芒。芒作。凶。知常容。容乃公。公乃王。〔王乃〕天。天乃道。道乃〔久〕。没身不殆。

    楚简甲本:

    至虛互也。守中篤也。萬勿方作。居以須復也。天道員員。各復其堇。

    \chapter{}
    王弼本:

    太上。下知有之。其次親而譽之。其次畏之。其次侮之。信不足焉有不信焉。悠兮其貴言。功成事遂。百姓皆謂我自然。

    
    帛书甲本:

    太上。下知有之。其次親譽之。其次畏之。其下侮之。信不足案有不信。〔猶呵〕其貴言也。成功遂事。而百姓謂我自然。

    帛书乙本:

    太上。下知又〔之。其次〕親譽之。其次畏之。其下侮之。信不足安有不信。猶呵其貴言也。成功遂事。而百姓謂我自然。

    楚简丙本:

    大上下智有之。其即親譽之。其既〈即〉畏之。其即侮之。信不足。安又不信。猶乎其貴言也。成事遂功。而百姓曰我自然也。(接下章)

    \chapter{}
    王弼本:

    大道廢。有仁義。慧智出。有大僞。六親不和。有孝慈。國家昏亂。有忠臣。

    
    帛书甲本:

    故大道廢。安有仁義。知慧出。安有大僞。六親不和。安有孝慈。邦家昏亂。安有貞臣。

    帛书乙本:

    故大道廢。安有仁義。知慧出。安有〔大僞〕。六親不和。安又孝慈。國家昏亂。安有貞臣。

    楚简丙本:

    (接上章) 古大道癹。安又仁義。六親不和。安又有孝慈。邦家昏☐。安又正臣。

    \chapter{}
    王弼本:

    絶聖棄智。民利百倍。絶仁棄義。民復孝慈。絶巧棄利。盗賊無有。此三者。以爲文不足。故令有所屬。見素抱樸。少私寡欲。

    
    帛书甲本:

    絶聖棄知。民利百倍。絶仁棄義。民復孝慈。絶巧棄利。盗賊无有。此三言也。以爲文未足。故令之有所屬。見素抱〔樸。少私而寡欲〕。

    帛书乙本:

    絶聖棄知。而民利百倍。絶仁棄義。而民復孝慈。絶巧棄利。盗賊无有。此三言也。以爲文未足。故令之有所屬。見素抱樸。少私而寡欲。

    楚简甲本:

    絕智弃鞭。民利百倍。絕巧棄利。盜賊亡又。絕化棄慮。民复季子。三言以爲史不足。或命之或乎屬。視素保樸。少私須欲。

    \chapter{}
    王弼本:

    絶學無憂。唯之與阿。相去幾何。善之與惡。相去若何。人之所畏。不可不畏。荒兮其未央哉。衆人熙熙。如享太牢。如春登臺。我獨泊兮其未兆。如嬰兒之未孩。儽儽兮若無所歸。衆人皆有餘。而我獨若遺。我愚人之心也哉。沌沌兮。俗人昭昭。我獨昏昏。俗人察察。我獨悶悶。澹兮其若海。飂兮若無止。衆人皆有以。而我獨頑似鄙。我獨異於人。而貴食母。

    
    帛书甲本:

    ⋯⋯。唯與訶。其相去幾何。美與惡。其相去何若。人之〔所畏〕。亦不〔可以不畏人〕。〔朢呵其未央哉〕。衆人熙熙。若鄉於大牢。而春登臺。我泊焉未佻。若〔嬰兒未咳〕。儽呵如〔无所歸〕。〔衆人〕皆有餘。我獨遺。我愚人之心也。沌沌呵。俗〔人昭昭。我獨若〕昏呵。俗人察察。我獨悶悶呵。忽呵其若〔海〕。朢呵其若无所止。〔衆人皆有以。我獨䦎〕以悝。我欲獨異於人。而貴食母。

    帛书乙本:

    絶學无憂。唯與呵。其相去幾何。美與惡。其相去何若。人之所畏。亦不可以不畏人。朢呵其未央哉。衆人熙熙。若鄉於大牢。而春登臺。我博焉未垗。若嬰兒未咳。儽呵似无所歸。衆人皆又余。我愚人之心也。沌沌呵。俗人昭昭。我獨若昏呵。俗人察察。我獨悶悶呵。沕呵其若海。朢呵若无所止。衆人皆有以。我獨䦎以鄙。吾欲獨異於人。而貴食母。

    楚简乙本:

    絕學無憂。唯與可。相去幾可。美與惡。相去可若。人之所畏。亦不可以不畏人。

    \chapter{}
    王弼本:

    孔德之容。惟道是從。道之爲物。惟恍惟惚。惚兮恍兮。其中有象。恍兮惚兮。其中有物。窈兮冥兮。其中有精。其精甚真。其中有信。自古及今。其名不去。以閲衆甫。吾何以知衆甫之狀哉。以此。

    
    帛书甲本:

    孔德之容。唯道是從。道之物。唯恍唯惚。〔惚呵恍〕呵。中有象呵。恍呵惚呵。中有物呵。窈呵冥呵。中有請吔。其請甚真。其中〔有信〕。自今及古。其名不去。以順衆父。吾何以知衆父之然。以此。

    帛书乙本:

    孔德之容。唯道是從。道之物。唯恍唯惚。惚呵恍呵。中又象呵。恍呵惚呵。中有物呵。窈呵冥呵。其中有請呵。其請甚真。其中有信。自今及古。其名不去。以順衆父。吾何以知衆父之然也。以此。

    \chapter{}
    王弼本:

    曲則全。枉則直。窪則盈。敝則新。少則得。多則惑。是以聖人抱一爲天下式。不自見故明。不自是故彰。不自伐故有功。不自矜故長。夫唯不爭。故天下莫能與之爭。古之所謂曲則全者。豈虚言哉。誠全而歸之。

    
    帛书甲本:

    曲則全。枉則定。窪則盈。敝則新。少則得。多則惑。是以聖人執一以爲天下牧。不〔自〕視故明。不自見故彰。不自伐故有功。弗矜故能長。夫唯不争。故莫能與之争。古〔之所謂曲全者。幾〕語哉。誠全歸之。

    帛书乙本:

    曲則全。枉則正。窪則盈。敝則新。少則得。多則惑。是以聖人執一以爲天下牧。不自視故彰。不自見也故明。不自伐故有功。弗矜故能長。夫唯不争。故莫能與之争。古之所謂曲全者。幾語哉。誠全歸之。

    \chapter{}
    王弼本:

    希言自然。故飄風不終朝。驟雨不終日。孰爲此者。天地。天地尚不能久。而況於人乎。故從事於道者。道者同於道。德者同於德。失者同於失。同於道者。道亦樂得之。同於德者。德亦樂得之。同於失者。失亦樂得之。信不足焉有不信焉。

    
    帛书甲本:

    希言自然。飄風不冬朝。暴雨不冬日。孰爲此。天地〔而弗能久。有況於人乎〕。故從事而道者同於道。德者同於德。失者同於失。同〔於德者〕。道亦德之。同於〔失〕者。道亦失之。

    帛书乙本:

    希言自然。飄風不冬朝。暴雨不冬日。孰爲此。天地而弗能久。有況於人乎。故從事而道者同於道。德者同於德。失者同於失。同於德者。道亦德之。同於失者。道亦失之。

    \chapter{}
    王弼本:

    企者不立。跨者不行。自見者不明。自是者不彰。自伐者無功。自矜者不長。其在道也。曰餘食贅行。物或惡之。故有道者不處。

    
    帛书甲本:

    企者不立。自視不彰。〔自〕見者不明。自伐者无功。自矜者不長。其在道。曰餘食贅行。物或惡之。故有欲者〔弗〕居。

    帛书乙本:

    企者不立。自視不彰。〔自〕見者不明。自伐者无功。自矜者不長。其在道也。曰餘食贅行。物或惡之。故有欲者弗居。

    \chapter{}
    王弼本:

    有物混成。先天地生。寂兮寥兮。獨立不改。周行而不殆。可以爲天下母。吾不知其名。字之曰道。强爲之名曰大。大曰逝。逝曰遠。遠曰反。故道大。天大。地大。王亦大。域中有四大。而王居其一焉。人法地。地法天。天法道。道法自然。

    
    帛书甲本:

    有物昆成。先天地生。寂呵寥呵。獨立〔而不改〕。可以爲天地母。吾未知其名。字之曰道。吾强爲之名曰大。大曰逝。逝曰〔遠。遠曰反〕。〔道大〕。天大。地大。王亦大。國中有四大。而王居一焉。人法地。地法〔天。天法道。道法自然〕。

    帛书乙本:

    有物昆成。先天地生。寂呵寥呵。獨立而不改。可以爲天地母。吾未知其名也。字之曰道。吾强爲之名曰大。大曰逝。逝曰遠。遠曰反。道大。天大。地大。王亦大。國中有四大。而王居一焉。人法地。地法天。天法道。道法自然。

    楚简甲本:

    又狀蟲成。先天地生。敓寥。獨立不改。可以爲天下母。未智其名。字之曰道。吾强爲之名曰大。大曰逝。逝曰遠。遠曰反。天大。地大。道大。王亦大。域中又四大安。王処一安。人法地。地法天。天法道。道法自然。

    \chapter{}
    王弼本:

    重爲輕根。靜爲躁君。是以聖人終日行不離輜重。雖有榮觀燕處超然。奈何萬乘之主。而以身輕天下。輕則失本。躁則失君。

    
    帛书甲本:

    〔重〕爲輕根。静爲躁君。是以君子衆日行。不蘺其輜重。唯有環官燕處〔則昭〕若。若何萬乘之王。而以身輕於天下。輕則失本。躁則失君。

    帛书乙本:

    重爲輕根。静爲躁君。是以君子冬日行。不遠其輜重。雖有環官燕處則昭若。若何萬乘之王。而以身輕於天下。輕則失本。躁則失君。

    \chapter{}
    王弼本:

    善行無轍迹。善言無瑕讁。善數不用籌策。善閉無關楗而不可開。善結無繩約而不可解。是以聖人常善救人。故無棄人。常善救物。故無棄物。是謂襲明。故善人者。不善人之師。不善人者。善人之資。不貴其師。不愛其資。雖智大迷。是謂要妙。

    
    帛书甲本:

    善行者无轍迹。〔善〕言者无瑕讁。善數者不以籌策。善閉者无關籥而不可啓也。善結者〔无繩〕約而不可解也。是以聖人恆善㤹人。而无棄人。物无棄財。是謂襲明。故善〔人。善人〕之師。不善人。善人之資也。不貴其師。不愛其資。唯知乎大迷。是謂眇要。

    帛书乙本:

    善行者无轍迹。善言者无瑕讁。善數者不用籌策。善閉者无關籥而不可啓也。善結者无繩約而不可解也。是以聖人恆善㤹人。而无棄人。物无棄財。是謂襲明。故善人。善人之師。不善人。善人之資也。不貴其師。不愛其資。雖知乎大迷。是謂眇要。

    \chapter{}
    王弼本:

    知其雄。守其雌。爲天下谿。爲天下谿。常德不離。復歸於嬰兒。知其白。守其黑。爲天下式。爲天下式。常德不忒。復歸於無極。知其榮。守其辱。爲天下谷。爲天下谷。常德乃足。復歸於樸。樸散則爲器。聖人用之則爲官長。故大制不割。

    
    帛书甲本:

    知其雄。守其雌。爲天下溪。爲天下溪。恆德不离。恆德不离。復歸嬰兒。知其榮。守其辱。爲天下浴。爲天下浴。恆德乃〔足〕。恒德乃〔足。復歸於樸〕。知其。守其黑。爲天下式。爲天下式。恒德不忒。恒德不忒。復歸於无極。樸散〔則爲器。聖〕人用則爲官長。夫大制无割。

    帛书乙本:

    知其雄。守其雌。爲天下溪。爲天下溪。恆德不离。恆德不离。復〔歸於嬰兒〕。〔知〕其榮。守其辱。爲天下浴。爲天下浴。恒德乃足。恆德乃足。復歸於樸。知其白。守其黑。爲天下式。爲天下式。恒德不忒。恒德不忒。復歸於无極。樸散則爲器。聖人用則爲官長。夫大制无割。

    \chapter{}
    王弼本:

    將欲取天下而爲之。吾見其不得已。天下神器。不可爲也。爲者敗之。執者失之。故物或行或隨。或歔或吹。或强或羸。或挫或隳。是以聖人去甚。去奢。去泰。

    
    帛书甲本:

    將欲取天下而爲之。吾見其弗〔得已。夫天下神〕器也。非可爲者也。爲者敗之。執者失之。物或行或隨。或炅或〔䂳。或强或羸〕。或杯或撱。是以聖人去甚。去大。去奢。

    帛书乙本:

    將欲取〔天下而爲之。吾見其弗〕得已。夫天下神器也。非可爲者也。爲之者敗之。執之者失之。故物或行或隨。或熱或䂳。或陪或墮。是以聖人去甚。去大。去奢。

    \chapter{}
    王弼本:

    以道佐人主者。不以兵强天下。其事好還。師之所處。荆棘生焉。大軍之後。必有凶年。善有果而已。不敢以取强。果而勿矜。果而勿伐。果而勿驕。果而不得已。果而勿强。物壯則老。是謂不道。不道早已。

    
    帛书甲本:

    以道佐人主。不以兵〔强於〕天下。〔其事好還。師之〕所居。楚棘生之。善者果而已矣。毋以取强焉。果而毋驕。果而勿矜。果而〔勿伐〕。果而毋得已居。是謂〔果〕而不强。物壯而老。是謂之不道。不道早已。

    帛书乙本:

    以道佐人主。不以兵强於天下。其〔事好還。師之所居。楚〕棘生之。善者果而已矣。毋以取强焉。果而毋驕。果而勿矜。果〔而勿〕伐。果而毋得已居。是謂果而强。物壯而老。謂之不道。不道早已。

    楚简甲本:

    以道佐人主者。不谷以兵彊於天下。善者果而已。不以取強。果而弗癹。果而弗驕。果而弗矜。是謂果而弗彊。其事好長。

    \chapter{}
    王弼本:

    夫佳兵者。不祥之器。物或惡之。故有道者不處。君子居則貴左。用兵則貴右。兵者。不祥之器。非君子之器。不得已而用之。恬淡爲上。勝而不美。而美之者。是樂殺人。夫樂殺人者。則不可以得志於天下矣。吉事尚左。凶事尚右。偏將軍居左。上將軍居右。言以喪禮處之。殺人之衆。以哀悲立之。戰勝。以喪禮處之。

    
    帛书甲本:

    夫兵者。不祥之器〔也〕。物或惡之。故有欲者弗居。君子居則貴左。用兵則貴右。故兵者非君子之器也。〔兵者〕不祥之器也。不得已而用之。銛襲爲上。勿美也。若美之。是樂殺人也。夫樂殺人。不可以得志於天下矣。是以吉事上左。喪事上右。是以偏將軍居左。上將軍居右。言以喪禮居之也。殺人衆。以悲哀立之。戰勝。以喪禮處之。

    帛书乙本:

    夫兵者。不祥之器也。物或惡〔之。故有欲者弗居〕。〔君子〕居則貴左。用兵則貴右。故兵者非君子之器。兵者不祥〔之〕器也。不得已而用之。銛襲爲上。勿美也。若美之。是樂殺人也。夫樂殺人。不可以得志於天下矣。是以吉事〔上左。喪事上右〕。是以偏將軍居左。而上將軍居右。言以喪禮居之也。殺〔人衆。以悲哀〕立之。〔戰〕勝。而以喪禮處之。

    楚简丙本:

    君子居則貴左。用兵則貴右。古曰兵者☐☐☐☐☐☐得已而用之。銛龔爲上。弗美也。美之。是樂殺人。夫樂☐☐☐以得志於天下。古吉事上左。喪事上右。是以偏將軍居左。上將軍居右。言以喪禮居之也。古殺☐☐則以哀悲立之。戰勝喪禮居之。

    \chapter{}
    王弼本:

    道常無名。樸雖小。天下莫能臣也。侯王若能守之。萬物將自賓。天地相合以降甘露。民莫之令而自均。始制有名。名亦既有。夫亦將知止。知止可以不殆。譬道之在天下。猶川谷之於江海。

    
    帛书甲本:

    道恆无名。樸唯〔小。而天下弗敢臣。侯〕王若能守之。萬物將自賓。天地相合。以俞甘露。民莫之〔令而自均〕焉。始制有〔名。名亦既〕有。夫〔亦將知止。知止〕所以不〔殆〕。譬道之在〔天下也。猶小〕浴之與江海也。

    帛书乙本:

    道恆无名。樸唯小。而天下弗敢臣。侯王若能守之。萬物將自賓。天地相合。以俞甘露。〔民莫之〕令而自均焉。始制有名。名亦既有。夫亦將知止。知止所以不殆。譬〔道之〕在天下也。猶小浴之與江海也。

    楚简甲本:

    道互亡名。僕唯妻。天地弗敢臣。侯王女能守之。萬勿將自賓。

    天地相合也。以逾甘露。民莫之命天〈而〉自均安。詒折有名。名亦既又。夫亦將智止。智止所以不詒。譬道之在天下也。猶少浴之與江海。

    \chapter{}
    王弼本:

    知人者智。自知者明。勝人者有力。自勝者强。知足者富。强行者有志。不失其所者久。死而不亡者壽。

    
    帛书甲本:

    知人者知也。自知〔者明也。勝人〕者有力也。自勝者〔强也〕。〔知足者富〕也。强行者有志也。不失其所者久也。死不亡者壽也。

    帛书乙本:

    知人者知也。自知明也。勝人者有力也。自勝者强也。知足者富也。强行者有志也。不失其所者久也。死而不亡者壽也。

    \chapter{}
    王弼本:

    大道氾兮。其可左右。萬物恃之而生而不辭。功成不名有。衣養萬物而不爲主。常無欲。可名於小。萬物歸焉而不爲主。可名爲大。以其終不自爲大。故能成其大。

    
    帛书甲本:

    道〔氾呵。其可左右也。成功〕遂事而弗名有也。萬物歸焉而弗爲主。則恒无欲也。可名於小。萬物歸焉〔而弗〕爲主。可名於大。是〔以〕聖人之能成大也。以其不爲大也。故能成大。

    帛书乙本:

    道氾呵。其可左右也。成功遂〔事而〕弗名有也。萬物歸焉而弗爲主。則恒无欲也。可名於小。萬物歸焉而弗爲主。可命於大。是以聖人之能成大也。以其不爲大也。故能成大。

    \chapter{}
    王弼本:

    執大象。天下往。往而不害。安平太。樂與餌。過客止。道之出口。淡乎其無味。視之不足見。聽之不足聞。用之不足既。

    
    帛书甲本:

    執大象。〔天下〕往。往而不害。安平太。樂與餌。過客止。故道之出言也。曰淡呵其无味也。〔視之〕不足見也。聽之不足聞也。用之不可既也。

    帛书乙本:

    執大象。天下往。往而不害。安平太。樂與〔餌〕。過客止。故道之出言也。曰淡呵其无味也。視之不足見也。聽之不足聞也。用之不可既也。

    楚简丙本:

    埶大象。天下往。往而不害。安平大。樂與餌。過客止。古道☐☐☐。淡可其無味也。視之不足見。聽之不足聞。而不可既也。

    \chapter{}
    王弼本:

    將欲歙之。必固張之。將欲弱之。必固强之。將欲廢之。必固興之。將欲奪之。必固與之。是謂微明。柔弱勝剛强。魚不可脱於淵。國之利器不可以示人。

    
    帛书甲本:

    將欲歙之。必固張之。將欲弱之。〔必固〕强之。將欲去之。必固與之。將欲奪之。必固予之。是謂微明。柔弱勝强。魚不〔可〕脱於淵。邦利器不可以示人。

    帛书乙本:

    將欲歙之。必固張之。將欲弱之。必固强之。將欲去之。必固與之。將欲奪之。必固予〔之〕。是謂微明。柔弱勝强。魚不可脱於淵。國利器不可以示人。

    \chapter{}
    王弼本:

    道常無爲而無不爲。侯王若能守之。萬物將自化。化而欲作。吾將鎮之以無名之樸。無名之樸。夫亦將無欲。不欲以靜。天下將自定。

    
    帛书甲本:

    道恒无名。侯王若守之。萬物將自化。化而欲〔作。吾將闐之以无〕名之樸。〔闐之以〕无名之樸。夫將不辱。不辱以情。天地將自正。

    帛书乙本:

    道恒无名。侯王若能守之。萬物將自化。化而欲作。吾將闐之以无名之樸。闐之以无名之樸。夫將不辱。不辱以静。天地將自正。

    楚简甲本:

    道互亡爲也。侯王能守之。而萬勿將自化。化而欲作。將貞之以亡名之樸。夫亦將智足。智足以朿。萬勿將自定。

    \chapter{}

    上德不德。是以有德。下德不失德。是以無德。上德無爲而無以爲。上仁爲之而無以爲。上義爲之而有以爲。上禮爲之而莫之應。則攘臂而扔之。故失道而後德。失德而後仁。失仁而後義。失義而後禮。夫禮者。忠信之薄而亂之首。前識者。道之華而愚之首。是以大丈夫處其厚。不居其薄。處其實。不居其華。故去彼取此。

    \chapter{}

    昔之得一者。天得一以清。地得一以寧。神得一以靈。浴得一以盈。侯王得一以爲天下正。其致之。天毋已清將恐裂。地毋已寧將恐發。神毋已靈將恐歇。浴毋已盈將恐竭。侯王毋已貴以高將恐蹶。故必貴以賤爲本。必高以下爲基。是以侯王自謂孤寡不穀。此其賤之本邪。非乎。故致數與无與。是故不欲琭琭如玉。珞珞如石。

    \chapter{}

    反者。道之動。弱者。道之用。天下之物生於有。有生於無。

    楚简甲本:

    返也者。道動也。溺也者。道之用也。天下之勿生於又。生於亡。

    \chapter{}

    上士聞道。堇(勤)能行之。中士聞道。若存若亡。下士聞道。大笑之。不笑不足以爲道。故建言有之。明道若昧。進道若退。夷道若纇。上德若浴。大白若辱。廣德若不足。建德若偷。質真若渝。大方無隅。大器免成。大音希聲。大象無形。道襃无名。夫唯道善始且善成。

    楚简乙本:

    上士聞道。堇能行於其中。中士聞道。若昏(聞)若亡。下士聞道。大笑之。弗大笑。不足以爲道矣。是以建言又之。明道女孛。遲道[如繢]。☐道若退。上德女浴。大白女辱。廣德女不足。建德女☐。☐貞女愉。大方亡禺。大器曼成。大音祗聲。天象亡坓。道⋯⋯

    \chapter{}

    道生一。一生二。二生三。三生萬物。萬物負陰而抱陽。沖氣以爲和。人之所惡。唯孤寡不穀。而王公以自名也。物或損之而益。益之而損。故人之所教。亦我而教人。故强梁者不得其死。吾將以爲教父。

    \chapter{}

    天下之至柔。馳騁於天下之至堅。無有入於無間。吾是以知無爲之有益。不言之教。無爲之益。天下希能及之。

    \chapter{}
    王弼本:

    名與身孰親。身與貨孰多。得與亡孰病。甚愛必大費。多藏必厚亡。故知足不辱。知止不殆。可以長久。

    楚简甲本:

    名與身孰親。身與貨孰多。得與亡孰病。甚愛必大費。厚藏必多亡。古智足不辱。智止不怠。可以長久。

    \chapter{}

    大成若缺。其用不幣。大盈若沖。其用不窮。大直若屈。大巧若拙。大辯若訥(大贏如㶧)。躁勝寒。靜勝熱。清靜爲天下正。

    楚简乙本:

    大成若缺。其用不幣。大浧若中。其用不窮。大巧若㑁。大成若詘。大植若屈。

    躁勝蒼。青勝然。清清爲天下定。

    \chapter{}

    天下有道。卻走馬以糞。天下無道。戎馬生於郊。
    
    罪莫大於可欲。禍莫大於不知足。咎莫大(憯)於欲得。故知足之足。恆足矣。

    楚简甲本:

    罪莫厚乎甚欲。咎莫僉乎谷得。禍莫大乎不智足。智足之爲足。此互足矣。

    \chapter{}

    不出於户。以知天下。不闚於牖。以知天道。其出彌遠。其知彌少。是以聖人不行而知。不見而名。不爲而成。

    \chapter{}

    爲學者日益。爲(聞)道者日損。損之又損。以至於無爲。無爲而無不爲。
    
    取天下恒以無事。及其有事。不足以取天下。

    楚简乙本:

    學者日益。爲道者日損。損之或損。以至亡爲也。亡爲而亡不爲。

    \chapter{}

    聖人恒无心。以百姓心爲心。善者善之。不善者亦善之。德善。信者信之。不信者亦信之。德信。聖人在天下歙歙。爲天下渾其心。百姓皆注其耳目焉。聖人皆孩之。

    \chapter{}

    出生入死。生之徒十有三。死之徒十有三。而民生生。動皆之死地之十有三。夫何故。以其生生也。蓋聞善攝生者。陵行不辟兕虎。入軍不被甲兵。兕無所投其角。虎無所措其爪。兵無所容其刃。夫何故。以其無死地焉。

    \chapter{}

    道生之。德畜之。物形之。器成之。是以萬物莫不尊道而貴德。道之尊。德之貴。夫莫之爵而恒自然。道生之。(德)畜之。長之。育之。亭之。毒之。養之。覆之。生而不有。爲而不恃。長而不宰。是謂玄德。

    \chapter{}

    天下有始。以爲天下母。既得其母。以知其子。既知其子。復守其母。没身不殆。
    
    塞其兑。閉其門。終身不勤。開其兑。濟其事。終身不救。
    
    見小曰明。守柔曰强。用其光。復歸其明。毋遺身殃。是謂襲常。

    楚简乙本:

    閉其門。塞其兑。終身不嵍。啟其兌。塞其事。終身不逨。

    \chapter{}

    使我介有知。行於大道。唯施是畏。大道甚夷。而民好徑。朝甚除。田甚蕪。倉甚虚。服文采。帶利劍。厭飲食。而齎財有餘。是謂盗夸。非道也哉。

    \chapter{}

    善建者不拔。善抱者不脱。子孫以祭祀不輟。修之身。其德乃真。修之家。其德乃餘。修之鄉。其德乃長。修之國。其德乃豐。修之天下。其德乃博。以身觀身。以家觀家。以鄉觀鄉。以國觀國。以天下觀天下。吾何以知天下之然哉。以此。

    楚简乙本:

    善建者不拔。善伂者不兌。子孫以其祭祀不乇。修之身。其德乃貞。修之家。其德又舍。修之鄉。其德乃長。修之邦。其德乃豐。修之天下☐☐☐☐☐☐☐家。以鄉觀鄉。以邦觀邦。以天下觀天下。吾可以智天☐☐☐☐☐。

    \chapter{}

    含德之厚者。比於赤子。蜂蠆虺蛇不螫。攫鳥猛獸不搏。骨弱筋柔而握固。未知牝牡之合而朘怒。精之至也。終日號而不嗄(嚘)。和之至也。知和曰常。知常曰明。益生曰祥。心使氣曰强。物壯則老。謂之不道。不道早已。

    楚简甲本:

    含德之厚者。比於赤子。螝䘍蟲它弗螫。攫鳥猛獸弗扣。骨溺筋柔而握固。未智牝牡之合丄怒。精之至也。終日唬而不憂。和之至也。和曰同〈常〉。智和曰明。益生曰羕。心史氣曰强。勿壯則老。是謂不道。

    \chapter{}

    知者不言。言者不知。塞其兑(㙂)。閉其門。挫其銳。解其紛。和其光。同其塵。是謂玄同。故不可得而親。亦不可得而疏。不可得而利。亦不可得而害。不可得而貴。亦不可得而賤。故爲天下貴。

    楚简甲本:

    智之者弗言。言之者弗智。閉其兑。塞其門。和其光。同其塵。挫其銳。解其紛。是謂玄同。古不可得天〈而〉親。亦不可得而疏。不可得而利。亦不可得而害。不可得而貴。亦可不可得而賤。古爲天下貴。

    \chapter{}

    以正治國。以奇用兵。以無事取天下。吾何以知其然哉。天下多忌諱。而民彌貧。民多利器。而國家滋昏。人多知。而奇物滋起。法物滋彰。而盗賊多有。是以聖人之言曰。我無爲而民自化。我好靜而民自正。我無事而民自富。我欲不欲而民自樸。

    楚简甲本:

    以正之邦。以奇用兵。以亡事取天下。吾可以智其然也。夫天多期韋而民彌畔。民多利器而邦慈昏。人多智天〈而〉可勿慈起。法勿慈彰。盜賊多又。是以聖人之言曰。我亡事而民自富。我亡爲而民自化。我好青而民自正。我谷不谷而民自樸。

    \chapter{}

    其政悶悶。其民淳淳。其政察察。其民缺缺。禍兮福之所倚。福兮禍之所伏。孰知其極。其無正。正復爲奇。善復爲妖。人之迷。其日固久。是以(聖人)方而不割。廉而不劌。直而不肆。光而不燿。

    \chapter{}

    治人事天莫若嗇。夫唯嗇。是以早服。早服是謂重積德。重積德則無不克。無不克則莫知其極。莫知其極可以有國。有國之母。可以長久。是謂深根固柢。長生久視之道。

    楚简乙本:

    治人事天。莫若嗇。夫唯嗇。是以早是以早備。是謂⋯⋯[無]不克。[無]不克則莫智其互。莫智其互。可以又䧕。又䧕之母。可以長⋯⋯長生久視之道也。

    \chapter{}

    治大國若烹小鮮。以道莅天下。其鬼不神。非其鬼不神也。其神不傷人也。非其神不傷人也。聖人亦不傷也。夫兩不相傷。故德交歸焉。

    \chapter{}

    大國者下流。天下之牝。天下之交。牝恒以靜勝牡。爲其静也。故宜爲下。故大國以下小國。則取小國。小國以下大國。則取於大國。故或下以取。或下而取。故大國不過欲兼畜人。小國不過欲入事人。夫皆得其欲。則大者宜爲下。

    \chapter{}

    道者萬物之注。善人之寶。不善人之所保。美言可以市。尊行可以加人。人之不善。何棄之有。故立天子。置三卿。雖有拱璧以先駟馬。不如坐而進此。古之所以貴此者何。不曰求以得。有罪以免邪。故爲天下貴。

    \chapter{}

    爲無爲。事無事。味無味。大小。多少。
    
    報怨以德。圖難於其易。爲大於其細。天下之難作於易。天下之大作於細。是以聖人終不爲大。故能成其大。

    夫輕諾必寡信。多易必多難。是以聖人猶難之。故終無難矣。

    楚简甲本:

    爲亡爲。事亡事。味亡味。大少之多易必多難。是以聖人猶難之。古終亡難。

    \chapter{}

    其安易持。其未兆易謀。其脆易泮。其微易散。爲之於未有。治之於未亂。合抱之木。生於毫末。九層之臺。起於累土。百仁之高。始於足下。

    爲者敗之。執者失之。是以聖人無爲故無敗。無執故無失。民之從事。恒於幾成而敗之。慎終如始。則無敗事。

    是以聖人欲不欲。不貴難得之貨。學不學。復衆人之所過。能輔萬物之自然。而不敢爲。

    楚简甲本:

    其安也。易持也。其未兆也。易謀也。其毳也。易畔也。其幾也。易㣤也。爲之於其亡又也。紿之於其未亂。合☐☐☐☐☐☐末。九成之臺。甲☐☐☐☐☐☐☐☐☐足下。

    爲之者敗之。執之者遠之。是以聖人亡爲古亡敗。亡執古亡失。臨事之紀。慎冬女怡。此亡敗事矣。

    聖人谷不谷。不貴難得之貨。学不学。復眾之所過。是古聖人能尃萬勿之自然而弗能爲。

    楚简丙本:

    爲之者敗之。執之者失之。聖人無爲。古無敗也。無執。古☐☐☐。慎終若詒。則無敗事矣。人之敗也。互於其且成也敗之。是以☐人欲不欲。不貴難得之貨。學不學。復眾之所過。是以能㭪萬勿之自然而弗敢爲。

    \chapter{}

    古之爲道者。非以明民。將以愚之。民之難治。以其智也。故以智治國。國之賊。以不智治國。國之德。恒知此兩者。亦稽式也。恒知稽式。是謂玄德。玄德深矣。遠矣。與物反矣。乃至大順。

    \chapter{}

    江海所以能爲百浴王者。以其善下之。故能爲百浴王。是以聖人之欲上民。必以言下之。其欲先民。必以身後之。是以處上而民不重。處前而民不害。天下樂推而不厭。以其不爭。故天下莫能與之爭。

    楚简甲本:

    江海所以爲百浴王。以其能爲百浴下。是以能爲百浴王。聖人之在民前也。以身後之。其在民上也。以言下之。其在民上也。民弗厚也。其在民前也。民弗害也。天下樂進而弗詀。以其不靜也。古天下莫能與之靜。

    \chapter{}

    天下皆謂我道大。大而不肖。夫唯大。故不肖。若肖。久矣其細也夫。我恒有三寶。持而寶之。一曰慈。二曰儉。三曰不敢爲天下先。慈。故能勇。儉。故能廣。不敢爲天下先。故能爲成器長。今舍慈且勇。舍儉且廣。舍後且先。死矣。夫慈。以戰則勝。以守則固。天將建之。如以慈垣之。

    \chapter{}

    善爲士者不武。善戰者不怒。善勝敵者不與。善用人者爲之下。是謂不爭之德。是謂用人。是謂配天。古之極也。

    \chapter{}

    用兵有言。吾不敢爲主而爲客。不敢進寸而退尺。是謂行無行。攘無臂。執無兵。乃无敵。禍莫大於無敵。無敵近亡吾寶。故抗兵相若。而哀者勝矣。

    \chapter{}

    吾言甚易知。甚易行。而天下莫能知。莫能行。言有宗。事有君。夫唯無知。是以不我知。知我者希。則我貴矣。是以聖人被褐懷玉。

    \chapter{}

    知不知上。不知知病。是以聖人之不病。以其病病。是以不病。

    \chapter{}

    民不畏威。則大威將至。毋狎其所居。毋厭其所生。夫唯弗厭。是以不厭。是以聖人自知不自見。自愛不自貴。故去彼取此。

    \chapter{}

    勇於敢則殺。勇於不敢則活。此兩者或利或害。天之所惡。孰知其故。天之道。不戰而善勝。不言而善應。不召而自來。繟而善謀。天網恢恢。疏而不失。

    \chapter{}

    若民恒且畏不畏死。奈何以殺懼之。若民恒且畏死。而爲奇者吾得而殺之。孰敢。若民恒且必畏死。則恒有司殺者。夫代司殺者殺。是謂代大匠斲。夫代大匠斲者。則希不傷其手矣。

    \chapter{}

    人之饑。以其取食税之多。是以饑。百姓之不治。以其上之有以爲也。是以不治。民之輕死。以其求生之厚。是以輕死。夫唯無以生爲者。是賢貴生。

    \chapter{}

    人之生也柔弱。其死也堅强。萬物草木之生也柔脆。其死也枯槁。故曰堅强者死之徒。柔弱者生之徒。是以兵强則不勝。木强則競。强大處下。柔弱處上。

    \chapter{}

    天之道。猶張弓也。高者抑之。下者舉之。有餘者損之。不足者補之。故天之道。損有餘而益不足。人之道。損不足而奉有餘。孰能有餘而有以取奉於天者。唯有道者。是以聖人爲而不有。成功而不居。若此其不欲見賢也。

    \chapter{}

    天下莫柔弱於水。而攻堅强者莫之能勝。以其無以易之。柔(水)之勝剛。弱之勝强。天下莫不知也。而莫能行也。是以聖人之言云。受國之垢。是謂社稷主。受國不祥。是爲天下王。正言若反。

    \chapter{}

    和大怨。必有餘怨。安可以爲善。是以聖人執左(右)契。而不以責於人。故有德司契。無德司徹。天道無親。恒與善人。

    \chapter{}

    小國寡民。使有十百人之器而不用。使民重死而遠徙。有舟輿。無所乘之。有甲兵。無所陳之。使民復結繩而用之。甘其食。美其服。安其居。樂其俗。鄰國相望。鷄犬之聲相聞。民至老死不相往來。

    \chapter{}

    信言不美。美言不信。知者不博。博者不知。善者不多。多者不善。聖人无積。既以爲人。己愈有。既以與人。己愈多。故天之道。利而不害。人之道。爲而不爭。

    \chapter*{参考文献}
    《老子道德經注校釋》,樓宇烈 校釋,中華書局。

    《帛書老子校注》,高明 撰,中華書局。

    《郭店楚簡老子集釋》,彭裕商、吴毅强 集釋,巴蜀書社。

\end{document}
