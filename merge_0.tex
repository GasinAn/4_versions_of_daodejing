\documentclass[a5paper]{ctexbook}
\usepackage{geometry}
\usepackage{hyperref}
\geometry{a5paper}
\title{《道德经》三版对照}
\author{安梯西登}
\date{}
\begin{document}
    \maketitle

    \tableofcontents

    \chapter{}
    王弼本:

    道可道。非常道。名可名。非常名。無名天地之始。有名萬物之母。故常無欲。以觀其妙。常有欲。以觀其徼。此兩者同出而異名。同謂之玄。玄之又玄。衆妙之門。

    
    帛书甲本:

    道可道也。非恒道也。名可名也。非恒名也。无名萬物之始也。有名萬物之母也。〔故〕垣无欲也。以觀其眇。恒有欲也。以觀其所噭。兩者同出。異名同謂。玄之有玄。衆眇之〔門〕。

    帛书乙本:

    道可道也。〔非恒道也。名可名也。非〕恒名也。无名萬物之始也。有名萬物之母也。故恒无欲也。〔以觀其眇〕。恒又欲也。以觀其所噭。兩者同出。異名同謂。玄之又玄。衆眇之門。

    \chapter{}
    王弼本:

    天下皆知美之爲美。斯惡已。皆知善之爲善。斯不善已。故有無相生。難易相成。長短相較。高下相傾。音聲相和。前後相隨。是以聖人處無爲之事。行不言之教。萬物作焉而不辭。生而不有。爲而不恃。功成而弗居。夫唯弗居。是以不去。

    
    帛书甲本:

    天下皆知美爲美。惡已。皆知善。斯不善矣。有无之相生也。難易之相成也。長短之相刑也。高下之相盈也。音聲之相和也。先後之相隨。恒也。是以聖人居无爲之事。行〔不言之教〕。〔萬物作而弗始〕也。爲而弗志也。成功而弗居也。夫唯居。是以弗去。

    帛书乙本:

    天下皆知美之爲美。惡已。皆知善。斯不善矣。〔有无之相〕生也。難易之相成也。長短之相刑也。高下之相盈也。音聲之相和也。先後之相隨。恒也。是以聖人居无爲之事。行不言之教。萬物作而弗始。爲而弗侍也。成功而弗居也。夫唯弗居。是以弗去。

    楚简甲本:

    天下皆智美之爲美也。惡已。皆智善。此其不善已。又亡之相生也。難易之相成也。長短之相型(形)也。高下之相浧也。音聲之相和也。先後之相隨也。是以聖人居亡爲之事。行不言之𡥈(教)。萬勿作而弗󶴢(始)也。爲而弗志也。成而弗居。天〈夫〉唯弗居也。是以弗去也。

    \chapter{}
    王弼本:

    不尚賢。使民不爭。不貴難得之貨。使民不爲盗。不見可欲。使民心不亂。是以聖人之治。虚其心。實其腹。弱其志。强其骨。常使民無知無欲。使夫智者不敢爲也。爲無爲。則無不治。

    
    帛书甲本:

    不上賢。〔使民不争。不貴難得之貨。使〕民不爲〔盗。不見可欲。使〕民不亂。是以聖人之〔治也。虚其心。實其腹。弱其志〕。强其骨。〔恒〕使民无知无欲也。使〔夫知不敢。弗爲而已。則无不治矣〕。

    帛书乙本:

    不上賢。使民不争。不貴難得之貨。使民不爲盗。不見可欲。使民不亂。是以聖人之治也。虚其心。實其腹。弱其志。强其骨。恒使民无知无欲也。使夫知不敢。弗爲而已。則无不治矣。

    \chapter{}
    王弼本:

    道沖而用之或不盈。淵兮似萬物之宗。挫其銳。解其紛。和其光。同其塵。湛兮似或存。吾不知誰之子。象帝之先。

    
    帛书甲本:

    〔道沖而用之有弗〕盈也。淵呵似萬物之宗。銼其。解其紛。和其光。同〔其塵〕。〔湛呵似〕或存。吾不知〔其誰之〕子也。象帝之先。

    帛书乙本:

    道沖而用之有弗盈也。淵呵似萬物之宗。銼其兑。解其芬。和其光。同其塵。湛呵似或存。吾不知其誰之子也。象帝之先。

    \chapter{}
    王弼本:

    天地不仁。以萬物爲芻狗。聖人不仁。以百姓爲芻狗。天地之間。其猶橐籥乎。虚而不屈。動而愈出。多言數窮。不如守中。

    
    帛书甲本:

    天地不仁。以萬物爲芻狗。聖人不仁。以百姓〔爲芻〕狗。天地〔之間。其〕猶橐籥與。虚而不淈。動而愈出。多聞數窮,不若守於中。

    帛书乙本:

    天地不仁。以萬物爲芻狗。聖人不仁。〔以〕百姓爲芻狗。天地之間。其猶橐籥與。虚而不淈。動而愈出。多聞數窮,不若守於中。

    楚简甲本:

    天地之間。其猶橐籥與。虛而不屈。動而愈出。

    \chapter{}
    王弼本:

    谷神不死。是謂玄牝。玄牝之門。是謂天地根。緜緜若存。用之不勤。

    
    帛书甲本:

    浴神〔不〕死。是謂玄牝。玄牝之門。是謂〔天〕地之根。緜緜呵若存。用之不堇。

    帛书乙本:

    浴神不死。是謂玄牝。玄牝之門。是謂天地之根。緜緜呵其若存。用之不堇。

    \chapter{}
    王弼本:

    天長地久。天地所以能長且久者。以其不自生。故能長生。是以聖人後其身而身先。外其身而身存。非以其無私邪。故能成其私。

    
    帛书甲本:

    天長地久。天地之所以能〔長〕且久者。以其不自生也。故能長生。是以聖人退其身而身先。外其身而身存。不以其无〔私〕輿。故能成其私。

    帛书乙本:

    天長地久。天地之所以能長且久者。以其不自生也。故能長生。是以聖人退其身而身先。外其身而身先。外其身而身存。不以其无私輿。故能成其私。

    \chapter{}
    王弼本:

    上善若水。水善利萬物而不爭。處衆人之所惡。故幾於道。居善地。心善淵。與善仁。言善信。正善治。事善能。動善時。夫唯不爭。故無尤。

    
    帛书甲本:

    上善治水。水善利萬物而有静。居衆之所惡。故幾於道矣。居善地。心善淵。予善信。正善治。事善能。動善時。夫唯不静。故无尤。

    帛书乙本:

    上善如水。水善利萬物而有争。居衆人之所惡。故幾於道矣。居善地。心善淵。予善天。言善信。正善治。事善能。動善時。夫唯不争。故无尤。

    \chapter{}
    王弼本:

    持而盈之。不如其已。揣而棁之。不可長保。金玉滿堂。莫之能守。富貴而驕。自遺其咎。功遂身退。天之道。

    
    帛书甲本:

    㨁而盈之。不〔若其已。揣而〕兑☐之。〔不〕可長葆之。金玉盈室。莫之守也。貴富而驕。自遺咎也。功遂身退。天〔之道也〕。

    帛书乙本:

    㨁而盈之。不若其已。揣而允之。不可長葆也。金玉〔盈〕室。莫之能守也。貴富而驕。自遺咎也。功遂身退。天之道也。

    楚简甲本:

    杫而浧之。不不若已。湍而羣之。不可長保也。金玉浧室。莫能守也。貴富驕。自遺咎也。功遂身退。天之道也。

    \chapter{}
    王弼本:

    載營魄抱一。能無離乎。專氣致柔。能嬰兒乎。滌除玄覽。能無疵乎。愛民治國。能無知乎。天門開闔。能無雌乎。明白四達。能無爲乎。生之畜之。生而不有。爲而不恃。長而不宰。是謂玄德。

    
    帛书甲本:

    〔載營柏抱一。能毋离乎。槫氣至柔〕。能嬰兒乎。脩除玄藍。能毋疵乎。〔愛民栝國。能毋以知乎〕。⋯⋯。生之畜之。生而弗〔有。長而弗宰也。是謂玄〕德。

    帛书乙本:

    載營柏抱一。能毋离乎。槫氣至柔。能嬰兒乎。脩除玄監。能毋有疵乎。愛民栝國。能毋以知乎。天門啟闔。能爲雌乎。明白四達。能毋以知乎。生之畜之。生而弗有。長而弗宰也。是謂玄德。

    \chapter{}
    王弼本:

    三十輻共一轂。當其無。有車之用。埏埴以爲器。當其無。有器之用。鑿户牖以爲室。當其無。有室之用。故有之以爲利。無之以爲用。

    
    帛书甲本:

    卅〔輻同一轂。當〕其无。〔有車〕之用〔也〕。埏埴爲器。當其无。有埴器〔之用也〕。〔鑿户牖〕。當其无。有〔室之〕用也。故有之以爲利。无之以爲用。

    帛书乙本:

    卅輻同一轂。當其无。有車之用也。埏埴而爲器。當其无。有埴器之用也。鑿户牖。當其无。有室之用也。故有之以爲利。无之以爲用。

    \chapter{}
    王弼本:

    五色令人目盲。五音令人耳聾。五味令人口爽。馳騁畋獵令人心發狂。難得之貨令人行妨。是以聖人爲腹不爲目。故去彼取此。

    
    帛书甲本:

    五色使人目盲。馳騁田獵使人〔心發狂〕。難得之貨使人之行方。五味使人之口爽。五音使人之耳聾。是以聖人之治也。爲腹不〔爲目〕。故去彼耳此。

    帛书乙本:

    五色使人目盲。馳騁田獵使人心發狂。難得之貨使人之行仿。五味使人之口爽。五音使人耳〔聾〕。是以聖人之治也。爲腹而不爲目。故去彼而取此。

    \chapter{}
    王弼本:

    寵辱若驚。貴大患若身。何謂寵辱若驚。寵爲下。得之若驚。失之若驚。是謂寵辱若驚。何謂貴大患若身。吾所以有大患者。爲吾有身。及吾無身。吾有何患。故貴以身爲天下。若可寄天下。愛以身爲天下。若可託天下。

    
    帛书甲本:

    寵辱若驚。貴大患若身。何謂寵辱若驚。寵之爲下。得之若驚。失〔之〕若驚。是謂寵辱若驚。何謂貴大患若身。吾所以有大患者。爲吾有身也。及吾无身。有何患。故貴爲身於爲天下。若可以託天下矣。愛以身爲天下。女可以寄天下。

    帛书乙本:

    寵辱若驚。貴大患若身。何謂寵辱若驚。寵之爲下也。得之若驚。失之若驚。是謂寵辱若驚。何謂貴大患若身。吾所以有大患者。爲吾有身也。及吾無身。有何患。故貴爲身於爲天下。若可以託天下〔矣〕。愛以身爲天下。女可以寄天下矣。

    楚简乙本:

    寵辱若纓。貴大患若身。可謂寵辱。寵爲下也。得之若纓。失之若纓。是謂寵辱纓。☐☐☐☐☐若身。吾所以又大患者。爲吾又身。﨤(及)吾亡身。或可☐☐☐☐☐☐爲天下。若可以厇天下矣。愛以身爲天下。若可以迲天下矣。

    \chapter{}
    王弼本:

    視之不見名曰夷。聽之不聞名曰希。搏之不得名曰微。此三者不可致詰。故混而爲一。其上不皦。其下不昧。繩繩不可名。復歸於無物。是謂無狀之狀。無物之象。是謂惚恍。迎之不見其首。隨之不見其後。執古之道。以御今之有。能知古始。是謂道紀。

    
    帛书甲本:

    視之而弗見名之曰微。聽之而弗聞名之曰希。捪之而弗得名之曰夷。三者不可至計。故混〔而爲一〕。一者。其上不攸。其下不忽。尋尋呵不可名也。復歸於无物。是謂无狀之狀。无物之〔象。是謂惚恍〕。〔隨而不見其後。迎〕而不見其首。執今之道。以御今之有。以知古始。是謂〔道紀〕。

    帛书乙本:

    視之而弗見〔名〕之曰微。聽之而弗聞命之曰希。捪之而弗得命之曰夷。三者不可至計。故混而爲一。一者。其上不謬。其下不忽。尋尋呵不可命也。復歸於无物。是謂无狀之狀。无物之象。是謂惚恍。隨而不見其後。迎而不見其首。執今之道。以御今之有。以知古始。是謂道紀。

    \chapter{}
    王弼本:

    古之善爲士者。微妙玄通。深不可識。夫唯不可識。故强爲之容。豫焉若冬涉川。猶兮若畏四鄰。儼兮其若容。涣兮若冰之將釋。敦兮其若樸。曠兮其若谷。混兮其若濁。孰能濁以靜之徐清。孰能安以久動之徐生。保此道者不欲盈。夫唯不盈。故能蔽不新成。

    
    帛书甲本:

    〔古之善爲道者。微眇玄達〕。深不可志。夫唯不可志。故强爲之容。曰。豫呵其若冬〔涉水。猶呵其若〕畏四〔鄰。嚴呵〕其若客。涣呵其若淩澤。王呵其若楃。湷〔呵其若濁。曠呵其〕若浴。濁而情之余清。女以動之余生。葆此道不欲盈。夫唯不欲〔盈。是以能敝而不〕成。

    帛书乙本:

    古之善爲道者。微眇玄達。深不可志。夫唯不可志。故强爲之容。曰。豫呵其若冬涉水。猶呵其若畏四鄰。嚴呵其若客。涣呵其若淩澤。沌呵其若樸。湷呵其若濁。曠呵其若浴。濁而静之徐清。女以動之徐生。葆此道〔不〕欲盈。是以能敝而不成。

    楚简甲本:

    古之善爲士者。必非(微)溺玄達。深不可志。是以爲之頌(容)。豫乎奴冬涉川。猶乎其奴畏四鄰。敢乎其奴客。渙乎其奴懌。屯乎其奴樸。坉乎其奴濁。竺(孰)能濁以朿(靜)者。將舍(徐)清。竺(孰)能庀以迬者。將舍(徐)生。保此道者。不谷(欲)尚呈。

    \chapter{}
    王弼本:

    致虚極。守靜篤。萬物並作。吾以觀復。夫物芸芸。各復歸其根。歸根曰靜。是謂復命。復命曰常。知常曰明。不知常。妄作。凶。知常容。容乃公。公乃王。王乃天。天乃道。道乃久。没身不殆。

    
    帛书甲本:

    至虚極也。守情表也。萬物旁作。吾以觀其復也。天物芸芸。各復歸於其〔根〕。〔曰静〕。静。是謂復命。復命常也。知常明也。不知常。㠵。㠵作。兇。知常容。容乃公。公乃王。王乃天。天乃道。〔道乃久〕。沕身不怠。

    帛书乙本:

    至虚極也。守静督也。萬物旁作。吾以觀其復也。天物芸芸。各復歸於其根。曰静。静。是謂復命。復命常也。知常明也。不知常。芒。芒作。凶。知常容。容乃公。公乃王。〔王乃〕天。天乃道。道乃〔久〕。没身不殆。

    楚简甲本:

    至虛𠄨(恒)也。守中󶴮(篤)也。萬勿方(旁)作。居以須󵯿(復)也。天道員員。各󵯿(復)其堇。

    \chapter{}
    王弼本:

    太上。下知有之。其次親而譽之。其次畏之。其次侮之。信不足焉有不信焉。悠兮其貴言。功成事遂。百姓皆謂我自然。

    
    帛书甲本:

    太上。下知有之。其次親譽之。其次畏之。其下侮之。信不足案有不信。〔猶呵〕其貴言也。成功遂事。而百姓謂我自然。

    帛书乙本:

    太上。下知又〔之。其次〕親譽之。其次畏之。其下侮之。信不足安有不信。猶呵其貴言也。成功遂事。而百姓謂我自然。

    楚简丙本:

    大上下智有之。其即新(親)譽之。其既〈即〉畏之。其即侮之。信不足。安又不信。猶乎其貴言也。成事遂功。而百姓曰我自然也。(接下章)

    \chapter{}
    王弼本:

    大道廢。有仁義。慧智出。有大僞。六親不和。有孝慈。國家昏亂。有忠臣。

    
    帛书甲本:

    故大道廢。安有仁義。知慧出。安有大僞。六親不和。安有孝慈。邦家昏亂。安有貞臣。

    帛书乙本:

    故大道廢。安有仁義。知慧出。安有〔大僞〕。六親不和。安又孝慈。國家昏亂。安有貞臣。

    楚简丙本:

    (接上章) 古大道癹。安又仁義。六新(親)不和。安又有孝慈。邦家昏☐。安又正臣。

    \chapter{}
    王弼本:

    絶聖棄智。民利百倍。絶仁棄義。民復孝慈。絶巧棄利。盗賊無有。此三者。以爲文不足。故令有所屬。見素抱樸。少私寡欲。

    
    帛书甲本:

    絶聲棄知。民利百負。絶仁棄義。民復孝慈。絶巧棄利。盗賊无有。此三言也。以爲文未足。故令之有所屬。見素抱〔樸。少私而寡欲〕。

    帛书乙本:

    絶聖棄知。而民利百倍。絶仁棄義。而民復孝慈。絶巧棄利。盗賊无有。此三言也。以爲文未足。故令之有所屬。見素抱樸。少私而寡欲。

    楚简甲本:

    絕智弃󶴉(辯)。民利百伓(倍)。絕攷(巧)棄利。󶴊(盜)惻(賊)亡又。絕𢠿(僞)棄󶴍(慮)。民复(復)季子。三言以爲󶴎(使)不足。或命之或乎豆(屬)。視索(素)保󶴏(樸)。少厶(私)須(寡)欲。

    \chapter{}
    王弼本:

    絶學無憂。唯之與阿。相去幾何。善之與惡。相去若何。人之所畏。不可不畏。荒兮其未央哉。衆人熙熙。如享太牢。如春登臺。我獨泊兮其未兆。如嬰兒之未孩。儽儽兮若無所歸。衆人皆有餘。而我獨若遺。我愚人之心也哉。沌沌兮。俗人昭昭。我獨昏昏。俗人察察。我獨悶悶。澹兮其若海。飂兮若無止。衆人皆有以。而我獨頑似鄙。我獨異於人。而貴食母。

    
    帛书甲本:

    ⋯⋯。唯與訶。其相去幾何。美與惡。其相去何若。人之〔所畏〕。亦不〔可以不畏人〕。〔朢呵其未央哉〕。衆人熙熙。若鄉於大牢。而春登臺。我泊焉未佻。若〔嬰兒未咳〕。纍呵如〔无所歸〕。〔衆人〕皆有餘。我獨遺。我禺人之心也。沌沌呵。俗〔人昭昭。我獨若〕昏呵。俗人察察。我獨悶悶呵。忽呵其若〔海〕。朢呵其若无所止。〔衆人皆有以。我獨䦎〕以悝。我欲獨異於人。而貴食母。

    帛书乙本:

    絶學无憂。唯與呵。其相去幾何。美與惡。其相去何若。人之所畏。亦不可以不畏人。朢呵其未央哉。衆人熙熙。若鄉於大牢。而春登臺。我博焉未垗。若嬰兒未咳。纍呵似无所歸。衆人皆又余。我愚人之心也。沌沌呵。俗人昭昭。我獨若昏呵。俗人察察。我獨悶悶呵。沕呵其若海。朢呵若无所止。衆人皆有以。我獨䦎以鄙。吾欲獨異於人。而貴食母。

    楚简乙本:

    絕學無憂。唯與可。相去幾可。美與惡。相去可若。人之所畏。亦不可以不畏人。

    \chapter{}
    王弼本:

    孔德之容。惟道是從。道之爲物。惟恍惟惚。惚兮恍兮。其中有象。恍兮惚兮。其中有物。窈兮冥兮。其中有精。其精甚真。其中有信。自古及今。其名不去。以閲衆甫。吾何以知衆甫之狀哉。以此。

    
    帛书甲本:

    孔德之容。唯道是從。道之物。唯恍唯惚。〔惚呵恍〕呵。中有象呵。恍呵惚呵。中有物呵。窈呵冥呵。中有請吔。其請甚真。其中〔有信〕。自今及古。其名不去。以順衆父。吾何以知衆父之然。以此。

    帛书乙本:

    孔德之容。唯道是從。道之物。唯恍唯惚。惚呵恍呵。中又象呵。恍呵惚呵。中有物呵。窈呵冥呵。其中有請呵。其請甚真。其中有信。自今及古。其名不去。以順衆父。吾何以知衆父之然也。以此。

    \chapter{}
    王弼本:

    曲則全。枉則直。窪則盈。敝則新。少則得。多則惑。是以聖人抱一爲天下式。不自見故明。不自是故彰。不自伐故有功。不自矜故長。夫唯不爭。故天下莫能與之爭。古之所謂曲則全者。豈虚言哉。誠全而歸之。

    
    帛书甲本:

    曲則金。枉則定。洼則盈。敝則新。少則得。多則惑。是以聖人執一以爲天下牧。不〔自〕視故明。不自見故章。不自伐故有功。弗矜故能長。夫唯不争。故莫能與之争。古〔之所謂曲全者。幾〕語哉。誠金歸之。

    帛书乙本:

    曲則全。汪則正。洼則盈。敝則新。少則得。多則惑。是以聖人執一以爲天下牧。不自視故章。不自見也故明。不自伐故有功。弗矜故能長。夫唯不争。故莫能與之争。古之所謂曲全者。幾語哉。誠全歸之。

    \chapter{}
    王弼本:

    希言自然。故飄風不終朝。驟雨不終日。孰爲此者。天地。天地尚不能久。而況於人乎。故從事於道者。道者同於道。德者同於德。失者同於失。同於道者。道亦樂得之。同於德者。德亦樂得之。同於失者。失亦樂得之。信不足焉有不信焉。

    
    帛书甲本:

    希言自然。飄風不冬朝。暴雨不冬日。孰爲此。天地〔而弗能久。有況於人乎〕。故從事而道者同於道。德者同於德。者者同於失。同〔於德者〕。道亦德之。同於〔失〕者。道亦失之。

    帛书乙本:

    希言自然。飄風不冬朝。暴雨不冬日。孰爲此。天地而弗能久。有況於人乎。故從事而道者同於道。德者同於德。失者同於失。同於德者。道亦德之。同於失者。道亦失之。

    \chapter{}
    王弼本:

    企者不立。跨者不行。自見者不明。自是者不彰。自伐者無功。自矜者不長。其在道也。曰餘食贅行。物或惡之。故有道者不處。

    
    帛书甲本:

    炊者不立。自視不章。〔自〕見者不明。自伐者无功。自矜者不長。其在道。曰餘食贅行。物或惡之。故有欲者〔弗〕居。

    帛书乙本:

    炊者不立。自視不章。〔自〕見者不明。自伐者无功。自矜者不長。其在道也。曰餘食贅行。物或惡之。故有欲者弗居。

    \chapter{}
    王弼本:

    有物混成。先天地生。寂兮寥兮。獨立不改。周行而不殆。可以爲天下母。吾不知其名。字之曰道。强爲之名曰大。大曰逝。逝曰遠。遠曰反。故道大。天大。地大。王亦大。域中有四大。而王居其一焉。人法地。地法天。天法道。道法自然。

    
    帛书甲本:

    有物昆成。先天地生。寂呵寥呵。獨立〔而不󱁡〕。可以爲天地母。吾未知其名。字之曰道。吾强爲之名曰大。大曰筮。筮曰〔遠。遠曰反〕。〔道大〕。天大。地大。王亦大。國中有四大。而王居一焉。人法地。地法〔天。天法道。道法自然〕。

    帛书乙本:

    有物昆成。先天地生。寂呵寥呵。獨立而不󱁡。可以爲天地母。吾未知其名也。字之曰道。吾强爲之名曰大。大曰筮。筮曰遠。遠曰反。道大。天大。地大。王亦大。國中有四大。而王居一焉。人法地。地法天。天法道。道法自然。

    楚简甲本:

    又󶴷蟲城(成)。先天地生。敓寥。蜀(獨)立不亥(改)。可以爲天下母。未智其名。字之曰道。吾󶴔(强)爲之名曰大。大曰󶴹。󶴹曰󶴿〈遠〉。󶴿〈遠〉曰反(返)。天大。地大。道大。王亦大。󶴺(域)中又四大安。王󶵀(居)一安。人法地。地法天。天法道。道法自然。

    \chapter{}
    王弼本:

    重爲輕根。靜爲躁君。是以聖人終日行不離輜重。雖有榮觀燕處超然。奈何萬乘之主。而以身輕天下。輕則失本。躁則失君。

    
    帛书甲本:

    〔重〕爲輕根。清爲躁君。是以君子衆日行。不蘺其輜重。唯有環官燕處〔則昭〕若。若何萬乘之王。而以身輕於天下。輕則失本。躁則失君。

    帛书乙本:

    重爲輕根。静爲躁君。是以君子冬日行。不遠其輜重。雖有環官燕處則昭若。若何萬乘之王。而以身輕於天下。輕則失本。躁則失君。

    \chapter{}
    王弼本:

    善行無轍迹。善言無瑕讁。善數不用籌策。善閉無關楗而不可開。善結無繩約而不可解。是以聖人常善救人。故無棄人。常善救物。故無棄物。是謂襲明。故善人者。不善人之師。不善人者。善人之資。不貴其師。不愛其資。雖智大迷。是謂要妙。

    
    帛书甲本:

    善行者无轍迹。〔善〕言者无瑕適。善數者不以籌策。善閉者无關籥而不可啓也。善結者〔无繩〕約而不可解也。是以聖人恆善㤹人。而无棄人。物无棄財。是謂襲明。故善〔人。善人〕之師。不善人。善人之資也。不貴其師。不愛其資。唯知乎大迷。是謂眇要。

    帛书乙本:

    善行者无轍迹。善言者无瑕適。善數者不用籌策。善閉者无關籥而不可啓也。善結者无繩約而不可解也。是以聖人恆善㤹人。而无棄人。物无棄財。是謂襲明。故善人。善人之師。不善人。善人之資也。不貴其師。不愛其資。雖知乎大迷。是謂眇要。

    \chapter{}
    王弼本:

    知其雄。守其雌。爲天下谿。爲天下谿。常德不離。復歸於嬰兒。知其白。守其黑。爲天下式。爲天下式。常德不忒。復歸於無極。知其榮。守其辱。爲天下谷。爲天下谷。常德乃足。復歸於樸。樸散則爲器。聖人用之則爲官長。故大制不割。

    
    帛书甲本:

    知其雄。守其雌。爲天下溪。爲天下溪。恆德不雞。恆德不雞。復歸嬰兒。知其日。守其辱。爲天下浴。爲天下浴。恆德乃〔足〕。恒德乃〔足。復歸於樸〕。知其。守其黑。爲天下式。爲天下式。恒德不貣。恒德不貣。復歸於无極。楃散〔則爲器。聖〕人用則爲官長。夫大制无割。

    帛书乙本:

    知其雄。守其雌。爲天下雞。爲天下雞。恆德不离。恆德不离。復〔歸於嬰兒〕。〔知〕其白。守其辱。爲天下浴。爲天下浴。恒德乃足。恆德乃足。復歸於樸。知其白。守其黑。爲天下式。爲天下式。恒德不貸。恒德不貸。復歸於无極。樸散則爲器。聖人用則爲官長。夫大制无割。

    \chapter{}
    王弼本:

    將欲取天下而爲之。吾見其不得已。天下神器。不可爲也。爲者敗之。執者失之。故物或行或隨。或歔或吹。或强或羸。或挫或隳。是以聖人去甚。去奢。去泰。

    
    帛书甲本:

    將欲取天下而爲之。吾見其弗〔得已。夫天下神〕器也。非可爲者也。爲者敗之。執者失之。物或行或隨。或炅或〔䂳。或强或羸〕。或杯或撱。是以聖人去甚。去大。去奢。

    帛书乙本:

    將欲取〔天下而爲之。吾見其弗〕得已。夫天下神器也。非可爲者也。爲之者敗之。執之者失之。故物或行或隨。或熱或䂳。或陪或墮。是以聖人去甚。去大。去奢。

    \chapter{}
    王弼本:

    以道佐人主者。不以兵强天下。其事好還。師之所處。荆棘生焉。大軍之後。必有凶年。善有果而已。不敢以取强。果而勿矜。果而勿伐。果而勿驕。果而不得已。果而勿强。物壯則老。是謂不道。不道早已。

    
    帛书甲本:

    以道佐人主。不以兵〔强於〕天下。〔其事好還。師之〕所居。楚棘生之。善者果而已矣。毋以取强焉。果而毋驕。果而勿矜。果而〔勿伐〕。果而毋得已居。是謂〔果〕而不强。物壯而老。是謂之不道。不道蚤已。

    帛书乙本:

    以道佐人主。不以兵强於天下。其〔事好還。師之所居。楚〕棘生之。善者果而已矣。毋以取强焉。果而毋驕。果而勿矜。果〔而勿〕伐。果而毋得已居。是謂果而强。物壯而老。謂之不道。不道蚤已。

    楚简甲本:

    以道差(佐)人宔(主)者。不谷(欲)以兵󶴘(強)於天下。善者果而已。不以取強。果而弗癹。果而弗驕。果而弗󶴙(矜)。是謂果而弗󶴘(強)。其事好長。

    \chapter{}
    王弼本:

    夫佳兵者。不祥之器。物或惡之。故有道者不處。君子居則貴左。用兵則貴右。兵者。不祥之器。非君子之器。不得已而用之。恬淡爲上。勝而不美。而美之者。是樂殺人。夫樂殺人者。則不可以得志於天下矣。吉事尚左。凶事尚右。偏將軍居左。上將軍居右。言以喪禮處之。殺人之衆。以哀悲泣之。戰勝。以喪禮處之。

    
    帛书甲本:

    夫兵者。不祥之器〔也〕。物或惡之。故有欲者弗居。君子居則貴左。用兵則貴右。故兵者非君子之器也。〔兵者〕不祥之器也。不得已而用之。銛襲爲上。勿美也。若美之。是樂殺人也。夫樂殺人。不可以得志於天下矣。是以吉事上左。喪事上右。是以偏將軍居左。上將軍居右。言以喪禮居之也。殺人衆。以悲哀立之。戰勝。以喪禮處之。

    帛书乙本:

    夫兵者。不祥之器也。物或惡〔之。故有欲者弗居〕。〔君子〕居則貴左。用兵則貴右。故兵者非君子之器。兵者不祥〔之〕器也。不得已而用之。銛𢤱爲上。勿美也。若美之。是樂殺人也。夫樂殺人。不可以得志於天下矣。是以吉事〔上左。喪事上右〕。是以偏將軍居左。而上將軍居右。言以喪禮居之也。殺〔人衆。以悲哀〕立之。〔戰〕勝。而以喪禮處之。

    楚简丙本:

    君子居則貴左。用兵則貴右。古曰兵者☐☐☐☐☐☐得已而用之。銛龔爲上。弗美也。美之。是樂殺人。夫樂☐☐☐以得志於天下。古吉事上左。喪事上右。是以偏將軍居左。上將軍居右。言以喪禮居之也。古殺☐☐則以哀悲位之。戰勝喪禮居之。

    \chapter{}
    王弼本:

    道常無名。樸雖小。天下莫能臣也。侯王若能守之。萬物將自賓。天地相合以降甘露。民莫之令而自均。始制有名。名亦既有。夫亦將知止。知止可以不殆。譬道之在天下。猶川谷之於江海。

    
    帛书甲本:

    道恆无名。握唯〔小。而天下弗敢臣。侯〕王若能守之。萬物將自賓。天地相谷。以俞甘露。民莫之〔令而自均〕焉。始制有〔名。名亦既〕有。夫〔亦將知止。知止〕所以不〔殆〕。俾道之在〔天下也。猶小〕浴之與江海也。

    帛书乙本:

    道恆无名。樸唯小。而天下弗敢臣。侯王若能守之。萬物將自賓。天地相合。以俞甘露。〔民莫之〕令而自均焉。始制有名。名亦既有。夫亦將知止。知止所以不殆。卑〔道之〕在天下也。猶小浴之與江海也。

    楚简甲本:

    道𠄨(恒)亡名。僕(樸)唯(雖)妻(細)。天地弗敢臣。侯王女能守之。萬勿將自󵦐(賓)。

    天地相合也。以逾甘露。民莫之命(令)天〈而〉自均安。󶴪(始)折(制)有名。名亦既又。夫亦將智止。智止所以不󶴪(殆)。卑(譬)道之在天下也。猶少浴之與江海。

    \chapter{}
    王弼本:

    知人者智。自知者明。勝人者有力。自勝者强。知足者富。强行者有志。不失其所者久。死而不亡者壽。

    
    帛书甲本:

    知人者知也。自知〔者明也。勝人〕者有力也。自勝者〔强也〕。〔知足者富〕也。强行者有志也。不失其所者久也。死不忘者壽也。

    帛书乙本:

    知人者知也。自知明也。勝人者有力也。自勝者强也。知足者富也。强行者有志也。不失其所者久也。死而不忘者壽也。

    \chapter{}
    王弼本:

    大道氾兮。其可左右。萬物恃之而生而不辭。功成不名有。衣養萬物而不爲主。常無欲。可名於小。萬物歸焉而不爲主。可名爲大。以其終不自爲大。故能成其大。

    
    帛书甲本:

    道〔渢呵。其可左右也。成功〕遂事而弗名有也。萬物歸焉而弗爲主。則恒无欲也。可名於小。萬物歸焉〔而弗〕爲主。可名於大。是〔以〕聖人之能成大也。以其不爲大也。故能成大。

    帛书乙本:

    道渢呵。其可左右也。成功遂〔事而〕弗名有也。萬物歸焉而弗爲主。則恒无欲也。可名於小。萬物歸焉而弗爲主。可命於大。是以聖人之能成大也。以其不爲大也。故能成大。

    \chapter{}
    王弼本:

    執大象。天下往。往而不害。安平太。樂與餌。過客止。道之出口。淡乎其無味。視之不足見。聽之不足聞。用之不足既。

    
    帛书甲本:

    執大象。〔天下〕往。往而不害。安平太。樂與餌。過格止。故道之出言也。曰談呵其无味也。〔視之〕不足見也。聽之不足聞也。用之不可既也。

    帛书乙本:

    執大象。天下往。往而不害。安平太。樂與〔餌〕。過格止。故道之出言也。曰淡呵其无味也。視之不足見也。聽之不足聞也。用之不可既也。

    楚简丙本:

    埶大象。天下往。往而不害。安平大。樂與餌。過客止。古道☐☐☐。啖可其無味也。視之不足見。聖(聽)之不足𦖞(聞)。而不可既也。

    \chapter{}
    王弼本:

    將欲歙之。必固張之。將欲弱之。必固强之。將欲廢之。必固興之。將欲奪之。必固與之。是謂微明。柔弱勝剛强。魚不可脱於淵。國之利器不可以示人。

    
    帛书甲本:

    將欲歙之。必固張之。將欲弱之。〔必固〕强之。將欲去之。必固與之。將欲奪之。必固予之。是謂微明。柔弱勝强。魚不〔可〕脱於淵。邦利器不可以示人。

    帛书乙本:

    將欲歙之。必固張之。將欲弱之。必固强之。將欲去之。必固與之。將欲奪之。必固予〔之〕。是謂微明。柔弱勝强。魚不可脱於淵。國利器不可以示人。

    \chapter{}
    王弼本:

    道常無爲而無不爲。侯王若能守之。萬物將自化。化而欲作。吾將鎮之以無名之樸。無名之樸。夫亦將無欲。不欲以靜。天下將自定。

    
    帛书甲本:

    道恒无名。侯王若守之。萬物將自𢡺。𢡺而欲〔作。吾將闐之以无〕名之楃。〔闐之以〕无名之楃。夫將不辱。不辱以情。天地將自正。

    帛书乙本:

    道恒无名。侯王若能守之。萬物將自化。化而欲作。吾將闐之以无名之樸。闐之以无名之樸。夫將不辱。不辱以静。天地將自正。

    楚简甲本:

    道𠄨(恒)亡爲也。侯王能守之。而萬勿將自𢠿(化)。𢠿(化)而𨿜(欲)作。將貞(鎮)之以亡名之󶴯(樸)。夫亦將智足。智足以朿(靜)。萬勿將自定。

    \chapter{}
    王弼本:

    上德不德。是以有德。下德不失德。是以無德。上德無爲而無以爲。下德爲之而有以爲。上仁爲之而無以爲。上義爲之而有以爲。上禮爲之而莫之應。則攘臂而扔之。故失道而後德。失德而後仁。失仁而後義。失義而後禮。夫禮者。忠信之薄而亂之首。前識者。道之華而愚之始。是以大丈夫處其厚。不居其薄。處其實。不居其華。故去彼取此。

    
    帛书甲本:

    〔上德不德。是以有德。下德不失德。是以无〕德。上德无〔爲而〕无以爲也。上仁爲之〔而无〕以爲也。上義爲之而有以爲也。上禮〔爲之而莫之應也。則〕攘臂而扔之。故失道而后德。失德而后仁。失仁而后義。〔失義而后禮。夫禮者。忠信之泊也〕。而亂之首也。〔前識者〕。道之華也。而愚之首也。是以大丈夫居其厚而不居其泊。居其實不居其華。故去彼取此。

    帛书乙本:

    上德不德。是以有德。下德不失德。是以无德。上德无爲而无以爲也。上仁爲之而无以爲也。上德爲之而有以爲也。上禮爲之而莫之應也。則攘臂而扔之。故失道而后德。失德而后仁。失仁而后義。失義而后禮。夫禮者。忠信之泊也。而亂之首也。前識者。道之華也。而愚之首也。是以大丈夫居〔其厚而不〕居其泊。居其實而不居其華。故去彼而取此。

    \chapter{}
    王弼本:

    昔之得一者。天得一以清。地得一以寧。神得一以靈。谷得一以盈。萬物得一以生。侯王得一以爲天下貞。其致之。天無以清將恐裂。地無以寧將恐發。神無以靈將恐歇。谷無以盈將恐竭。萬物無以生將恐滅。侯王無以貴高將恐蹶。故貴以賤爲本。高以下爲基。是以侯王自謂孤寡不穀。此非以賤爲本邪。非乎。故致數輿無輿。不欲琭琭如玉。珞珞如石。

    
    帛书甲本:

    昔之得一者。天得一以清。地得〔一〕以寧。神得一以靈。浴得一以盈。侯〔王得一〕而以爲〔天下〕正。其致之也。謂天毋已清將恐〔蓮〕。謂地毋〔已寧〕將恐〔發〕。謂神毋已靈〔將〕恐歇。謂浴毋已盈將恐渴。謂侯王毋已貴〔以高將恐欮〕。故必貴而以賤爲本。必高矣而以下爲基。夫是以侯王自謂〔孤〕寡不穀。此其〔賤之本與。非也〕。故致數與无與。是故不欲〔琭琭〕若玉。珞〔珞若石〕。

    帛书乙本:

    昔得一者。天得一以清。地得一以寧。神得一以靈。浴得一盈。侯王得一以爲天下正。其至也。謂天毋已清將恐蓮。地毋已寧將恐發。神毋〔已靈將〕恐歇。谷毋已〔盈〕將渴。侯王毋已貴以高將恐欮。故必貴以賤爲本。必高矣而以下爲基。夫是以侯王自謂孤寡不穀。此其賤之本與。非也。故至數輿无輿。是故不欲琭琭若玉。珞珞若石。

    \chapter{}
    王弼本:

    反者。道之動。弱者。道之用。天下萬物生於有。有生於無。

    
    帛书甲本:

    〔反也者〕。道之動也。弱也者。道之用也。天〔下之物生於有。有生於无〕。

    帛书乙本:

    反也者。道之動也。〔弱也〕者。道之用也。天下之物生於有。有〔生〕於无。

    楚简甲本:

    返也者。道動也。溺(弱)也者。道之用也。天下之勿生於又。生於亡。

    \chapter{}
    王弼本:

    上士聞道。勤而行之。中士聞道。若存若亡。下士聞道。大笑之。不笑不足以爲道。故建言有之。明道若昧。進道若退。夷道若纇。上德若谷。大白若辱。廣德若不足。建德若偷。質真若渝。大方無隅。大器晚成。大音希聲。大象無形。道隱無名。夫唯道善貸且成。

    
    帛书甲本:

    ⋯⋯。〔夫唯道〕善始〔且善成〕。

    帛书乙本:

    上〔士聞〕道。堇能行之。中士聞道。若存若亡。下士聞道。大笑之。弗笑〔不足〕以爲道。是以建言有之曰。明道如費。進道如退。夷道如類。上德如浴。大白如辱。廣德如不足。建德如〔偷〕。質〔真若渝〕。大方无禺。大器免成。大音希聲。天象无刑。道襃无名。夫唯道善始且善成。

    楚简乙本:

    上士聞道。堇能行於其中。中士聞道。若昏(聞)若亡。下士聞道。大𦬫(笑)之。弗大𦬫(笑)。不足以爲道矣。是以建言又之:明道女孛(昧)。遲(夷)道[如繢]。☐道若退。上德女浴。大白女辱。󼧊(廣)德女不足。建德女󲳴☐。☐貞(真)女愉。大方亡禺(隅)。大器曼(慢)成。大音祗聲。天(大)象亡坓(形)。道⋯⋯

    \chapter{}
    王弼本:

    道生一。一生二。二生三。三生萬物。萬物負陰而抱陽。沖氣以爲和。人之所惡。唯孤寡不穀。而王公以爲稱。故物或損之而益。或益之而損。人之所教。我亦教之。强梁者不得其死。吾將以爲教父。

    
    帛书甲本:

    〔道生一。一生二。二生三。三生萬物。萬物負陰而抱陽〕。中氣以爲和。天下之所惡。唯孤寡不穀。而王公以自名也。勿或損之〔而益。益〕之而損。故人〔之所〕教。夕議而教人。故强良者不得死。我〔將〕以爲學父。

    帛书乙本:

    道生一。一生二。二生三。三生〔萬物。萬物負陰而抱陽。中氣〕以爲和。人之所惡。唯〔孤〕寡不穀。而王公以自〔名也〕。〔勿或益之而〕損。損之而益。⋯⋯。〔故强良者不得死〕。〔我〕將以〔爲學〕父。

    \chapter{}
    王弼本:

    天下之至柔。馳騁天下之至堅。無有入無間。吾是以知無爲之有益。不言之教。無爲之益。天下希及之。

    
    帛书甲本:

    天下之至柔。〔馳〕騁於天下之致堅。无有入於无間。吾是以知无爲〔之有〕益也。不〔言之〕教。无爲之益。〔天〕下希能及之矣。

    帛书乙本:

    天下之至〔柔〕。馳騁乎天下〔之致堅〕。〔无有入於〕无間。吾是以〔知无爲之有益〕也。不〔言之教。无爲之益。天下希能及之〕矣。

    \chapter{}
    王弼本:

    名與身孰親。身與貨孰多。得與亡孰病。是故甚愛必大費。多藏必厚亡。知足不辱。知止不殆。可以長久。

    
    帛书甲本:

    名與身孰亲。身與貨孰多。得與亡孰病。甚〔愛必大費。多藏必厚〕亡。故知足不辱。知止不殆。可以長久。

    帛书乙本:

    名與〔身孰亲。身與貨孰多。得與亡孰病〕。⋯⋯。

    楚简甲本:

    名與身䈞(孰)新(親)。身與貨䈞(孰)多。󰴼(得)與󶵔(亡)䈞(孰)󶓄(病)。甚愛必大󶵖(費)。󶵗(厚)󶤖(藏)必多󶵔(亡)。古智足不辱。智止不怠(殆)。可以長舊(久)。

    \chapter{}
    王弼本:

    大成若缺。其用不弊。大盈若沖。其用不窮。大直若屈。大巧若拙。大辯若訥。躁勝寒。靜勝熱。清靜爲天下正。

    
    帛书甲本:

    大成若缺。其用不幣。大盈若𥁵。其用不𡩫。大直如詘。大巧如拙。大贏如㶧。躁勝寒。靚勝炅。請靚可以爲天下正。

    帛书乙本:

    〔大成若缺。其用不幣。大〕盈如沖。其〔用不𡩫〕。〔大直如詘。大巧〕如拙。〔大贏如〕絀。躁勝寒。〔靚勝炅。請靚可以爲天下正〕。

    楚简乙本:

    大成若缺。其用不󶵢(弊)。大浧若中(盅)。其用不󶵣(窮)。大攷(巧)若㑁(拙)。大成若詘。大植(直)若屈。

    躁󼡲(勝)蒼(滄)。青(静)󼡲(勝)然(熱)。清青(靜)爲天下定。

    \chapter{}
    王弼本:

    天下有道。卻走馬以糞。天下無道。戎馬生於郊。禍莫大於不知足。咎莫大於欲得。故知足之足。常足矣。

    
    帛书甲本:

    天下有〔道。卻〕走馬以糞。天下无道。戎馬生於郊。罪莫大於可欲。禍莫大於不知足。咎莫憯於欲得。〔故知足之足〕。恆足矣。

    帛书乙本:

    〔天下有〕道。卻走馬〔以〕糞。无道。戎馬生於郊。罪莫大可欲。禍〔莫大於不知足。咎莫憯於欲得〕。〔故知足之足。恆〕足矣。

    楚简甲本:

    辠(罪)莫厚乎甚欲。咎莫僉(憯)乎谷(欲)得。禍莫大乎不智足。智足之爲足。此𠄨(恒)足矣。

    \chapter{}
    王弼本:

    不出户知天下。不闚牖見天道。其出彌遠。其知彌少。是以聖人不行而知。不見而名。不爲而成。

    
    帛书甲本:

    不出於户。以知天下。不闚於牖。以知天道。其出也彌遠。其〔知彌少〕。〔是以聖人不行而知。不見而名。弗〕爲而〔成〕。

    帛书乙本:

    不出於户。以知天下。不闚於〔牖。以〕知天道。其出彌遠者。其知彌〔少〕。〔是以聖人不行而知。不見〕而名。弗爲而成。

    \chapter{}
    王弼本:

    爲學日益。爲道日損。損之又損。以至於無爲。無爲而無不爲。取天下常以無事。及其有事。不足以取天下。

    
    帛书甲本:

    ⋯⋯。取天下也恒〔无事。及其有事也。不足以取天下〕。

    帛书乙本:

    爲學者日益。聞道者日損。損之有損。以至於无〔爲〕。⋯⋯。取天下恒无事。及其有事也。〔不〕足以取天〔下〕。

    楚简乙本:

    學者日益。爲道者日損。損之或損。以至亡爲也。亡爲而亡不爲。

    \chapter{}
    王弼本:

    聖人無常心。以百姓心爲心。善者吾善之。不善者吾亦善之。德善。信者吾信之。不信者吾亦信之。德信。聖人在天下歙歙。爲天下渾其心。聖人皆孩之。

    
    帛书甲本:

    〔聖人恒无心〕。以百〔姓〕之心爲〔心〕。善者善之。不善者亦善〔之。德善也〕。〔信者信之。不信者亦信之。德〕信也。〔聖人〕之在天下歙歙焉。爲天下渾心。百姓皆屬耳目焉。聖人〔皆孩之〕。

    帛书乙本:

    〔聖〕人恒无心。以百姓之心爲心。善〔者善之。不善者亦善之。德〕善也。信者信之。不信者亦信之。德信也。聖人之在天下也歙歙焉。〔爲天下渾心〕。〔百姓〕皆注其〔耳目焉。聖人皆孩之〕。

    \chapter{}
    王弼本:

    出生入死。生之徒十有三。死之徒十有三。人之生動之死地亦十有三。夫何故。以其生生之厚。蓋聞善攝生者。陸行不遇兕虎。入軍不被甲兵。兕無所投其角。虎無所措其爪。兵無所容其刃。夫何故。以其無死地。

    
    帛书甲本:

    〔出〕生〔入死。生之徒十〕有〔三。死之〕徒十有三。而民生生動皆之死地之十有三。夫何故也。以其生生也。蓋〔聞善〕執生者。陵行不〔辟〕矢虎。入軍不被甲兵。矢无所投其角。虎无所措其爪。兵无所容〔其刃。夫〕何故也。以其无死地焉。

    帛书乙本:

    〔出〕生入死。生之〔徒十有三。死〕之徒十又三。而民生生動皆之死地之十有三。〔夫〕何故也。以其生生。蓋聞善執生者。陵行不辟𧰽虎。入軍不被兵革。𧰽无〔所投其角。虎无所措〕其爪。兵〔无所容其刃。夫何故〕也。以其无〔死地焉〕。

    \chapter{}
    王弼本:

    道生之德畜之。物形之勢成之。是以萬物莫不尊道而貴德。道之尊。德之貴。夫莫之命而常自然。故道生之。德畜之。長之育之。亭之毒之。養之覆之。生而不有。爲而不恃。長而不宰。是謂玄德。

    
    帛书甲本:

    道生之而德畜之。物刑之而器成之。是以萬物尊道而貴〔德〕。〔道〕之尊。德之貴也。夫莫之𡬠而恒自然也。道生之。畜之。長之遂之。亭〔之毒之。養之復之〕。〔生而〕弗有也。爲而弗寺也。長而弗宰也。此之謂玄德。

    帛书乙本:

    道生之德畜之。物刑之而器成之。是以萬物尊道而貴德。道之尊也。德之貴也。夫莫之爵也而恒自然也。道生之。畜之。〔長之遂〕之。亭之毒之。養之復之。〔生而弗有。爲而弗寺。長而〕弗宰。是謂玄德。

    \chapter{}
    王弼本:

    天下有始。以爲天下母。既得其母。以知其子。既知其子。復守其母。没身不殆。塞其兑。閉其門。終身不勤。開其兑。濟其事。終身不救。見小曰明。守柔曰强。用其光。復歸其明。無遺身殃。是爲習常。

    
    帛书甲本:

    天下有始。以爲天下母。既得其母。以知其〔子〕。復守其母。没身不殆。塞其󱁂。閉其門。終身不堇。啟其悶。濟其事。終身〔不棘〕。〔見〕小曰〔明〕。守柔曰强。用其光。復歸其明。毋遺身央。是謂襲常。

    帛书乙本:

    天下有始。以爲天下母。既得其母。以知其子。既知其子。復守其母。没身不佁。塞其㙂。閉其門。冬身不堇。啟其㙂。齊其〔事。終身〕不棘。見小曰明。守〔柔曰〕强。用〔其光。復歸其明。毋〕遺身央。是謂〔襲〕常。

    楚简乙本:

    閟(閉)其門。賽(塞)其𨓚(兌)。終身不嵍。啟其𨓚(兌)。賽(塞)其事。終身不󶵠。

    \chapter{}
    王弼本:

    使我介然有知。行於大道。唯施是畏。大道甚夷。而民好徑。朝甚除。田甚蕪。倉甚虚。服文綵。帶利劍。厭飲食財貨有餘。是謂盗夸。非道也哉。

    
    帛书甲本:

    使我𢴲有知。〔行於〕大道。唯〔他是畏〕。〔大道〕甚夷。民甚好解。朝甚除。田甚芜。倉甚虚。服文采。帶利〔劍。厭飲〕食〔資財有餘〕。⋯⋯。

    帛书乙本:

    使我介有知。行於大道。唯他是畏。大道甚夷。民甚好𠎿。朝甚除。田甚芜。倉甚虚。服文采。帶利劍。猒食而資財〔有餘〕。〔是謂盗〕木。非〔道也哉〕。

    \chapter{}
    王弼本:

    善建者不拔。善抱者不脱。子孫以祭祀不輟。修之於身。其德乃真。修之於家。其德乃餘。修之於鄉。其德乃長。修之於國。其德乃豐。修之於天下。其德乃普。故以身觀身。以家觀家。以鄉觀鄉。以國觀國。以天下觀天下。吾何以知天下然哉。以此。

    
    帛书甲本:

    善建〔者不〕拔。〔善抱者不脱〕。子孫以祭祀〔不絶〕。〔修之身。其德乃真。修之家。其德有〕餘。修之〔鄉。其德乃長。修之國。其德乃夆。修之天下。其德乃博〕。以身〔觀〕身。以家觀家。以鄉觀鄉。以邦觀邦。以天〔下觀天下。吾何以知天下之然兹。以此〕。

    帛书乙本:

    善建者〔不拔。善抱者不脱〕。子孫以祭祀不絶。修之身。其德乃真。修之家。其德有餘。修之鄉。其德乃長。修之國。其德乃夆。修之天下。其德乃博。以身觀身。以家觀〔家。以國觀〕國。以天下觀天下。〔吾何以知〕天下之然兹。以〔此〕。

    楚简乙本:

    善建者不拔。善伂(抱)者不兌(脫)。子孫以其祭祀不乇。修之身。其德乃貞(真)。修之家。其德又舍(餘)。修之鄉。其德乃長。修之邦。其德乃奉(豐)。修之天下☐☐☐☐☐☐☐家。以鄉觀鄉。以邦觀邦。以天下觀天下。吾可以智天☐☐☐☐☐。

    \chapter{}
    王弼本:

    含德之厚。比於赤子。蜂蠆虺蛇不螫。猛獸不據。攫鳥不搏。骨弱筋柔而握固。未知牝牡之合而全作。精之至也。終日號而不嗄。和之至也。知和曰常。知常曰明。益生曰祥。心使氣曰强。物壯則老。謂之不道。不道早已。

    
    帛书甲本:

    〔含德〕之厚〔者〕。比於赤子。逢𢔯𧍥地弗螫。㩴鳥猛獸弗搏。骨弱筋柔而握固。未知牝牡〔之會而朘怒〕。精〔之〕至也。終日號而不𢖻。和之至也。和曰常。知和曰明。益生曰祥。心使氣曰强。〔物壯〕卽老。謂之不道。不道〔蚤已〕。

    帛书乙本:

    含德之厚者。比於赤子。𧒒癘䖝蛇弗赫。據鳥孟獸弗捕。骨筋弱柔而握固。未知牝牡之會而朘怒。精之至也。冬日號而不嚘。和〔之至也〕。〔知和曰〕常。知常曰明。益生〔曰〕祥。心使氣曰强。物〔壯〕則老。謂之不道。不道蚤已。

    楚简甲本:

    含德之厚者。比於赤子。󶵎(螝)䘍蟲它(蛇)弗𧍷。攫鳥󶵏(猛)獸弗扣。骨溺(弱)蓳(筋)󶵐(柔)而捉固。未智牝戊(牡)之合󶵑󶵒(怒)。精之至也。終日唬而不𪬐(啞)。和之至也。和曰󶵓〈󼲗(常)〉。智和曰明。賹(益)生曰羕(祥)。心󶴎(使)󶴓(氣)曰󶴔(强)。勿𡒉(壯)則老。是謂不道。

    \chapter{}
    王弼本:

    知者不言。言者不知。塞其兑。閉其門。挫其銳。解其分。和其光。同其塵。是謂玄同。故不可得而親。不可得而疎。不可得而利。不可得而害。不可得而貴。不可得而賤。故爲天下貴。

    
    帛书甲本:

    〔知者〕弗言。言者弗知。塞其悶。閉其〔門。和〕其光。同其𡑁。坐其閲。解其紛。是謂玄同。故不可得而親。亦不可得而疏。不可得而利。亦不可得而害。不可〔得〕而貴。亦不可得而賤。故爲天下貴。

    帛书乙本:

    知者弗言。言者弗知。塞其㙂。閉其門。和其光。同其塵。銼其兑而解其紛。是謂玄同。故不可得而親也。亦〔不可得〕而〔疏。不可得〕而利。〔亦不可〕得而害。不可得而貴。亦不可得而賤。故爲天下貴。

    楚简甲本:

    智之者弗言。言之者弗智。𨳮〈閉〉其𨓚(兌)。賽(塞)其門。和其光。迵(同)其󶴤(塵)。󶴤其󶩴。解其紛。是謂玄同。古不可得天〈而〉親。亦不可得而疏。不可得而利。亦不可得而害。不可得而貴。亦可不可得而賤。古爲天下貴。

    \chapter{}
    王弼本:

    以正治國。以奇用兵。以無事取天下。吾何以知其然哉。以此。天下多忌諱而民彌貧。民多利器。國家滋昏。人多伎巧。奇物滋起。法令滋彰。盗賊多有。故聖人云。我無爲而民自化。我好靜而民自正。我無事而民自富。我無欲而民自樸。

    
    帛书甲本:

    以正之邦。以畸用兵。以无事取天下。吾何〔以知其然〕也哉。夫天下〔多忌諱〕。而民彌貧。民多利器。而邦家兹昏。人多知。而何物兹〔起。法物兹章。而〕盗賊〔多有〕。〔是以聖人之言曰〕。我无爲也而民自化。我好静而民自正。我无事民〔自富。我欲不欲而民自樸〕。

    帛书乙本:

    以正之國。以畸用兵。以無事取天下。吾何以知其然也哉。夫天下多忌諱。而民彌貧。民多利器。〔而邦家兹〕昏。〔人多知。而何物兹起。法〕物兹章。而盗賊〔多有〕。是以〔聖〕人之言曰。我无爲而民自化。我好静而民自正。我无事而民自富。我欲不欲而民自樸。

    楚简甲本:

    以正之邦。以󶵊(奇)用兵。以亡事取天下。吾可以智其然也。夫天多期(忌)韋(諱)而民爾(彌)畔(叛)。民多利器而邦慈昏。人多智天〈而〉𢦪(何)勿慈(滋)󶵋(起)。法勿慈(滋)章(彰)。覜(盜)惻(賊)多又。是以聖人之言曰。我亡事而民自富。我亡爲而民自󶵍(化)。我好青(靜)而民自正。我谷(欲)不谷(欲)而民自樸。

    \chapter{}
    王弼本:

    其政悶悶。其民淳淳。其政察察。其民缺缺。禍兮福之所倚。福兮禍之所伏。孰知其極。其無正。正復爲奇。善復爲妖。人之迷。其日固久。是以聖人方而不割。廉而不劌。直而不肆。光而不燿。

    
    帛书甲本:

    〔其正悶悶。其民淳淳〕。其正察察。其邦缺缺。禍福之所倚。福禍之所伏。〔孰知其極〕。⋯⋯。

    帛书乙本:

    其正悶悶。其民淳淳。其正察察。其〔邦缺缺〕。〔禍福之所倚。福禍之〕所伏。孰知其極。〔其〕无正也。正〔復爲奇〕。善復爲〔妖。人〕之迷也。其日固久矣。是以方而不割。兼而不刺。直而不紲。光而不眺。

    \chapter{}
    王弼本:

    治人事天莫若嗇。夫唯嗇。是謂早服。早服謂之重積德。重積德則無不克。無不克則莫知其極。莫知其極可以有國。有國之母。可以長久。是謂深根固柢。長生久視之道。

    
    帛书甲本:

    ⋯⋯。〔重積德則無不克。無不克則莫知其極。莫知其極〕可以有國。有國之母。可以長久。是謂深󱁆固氐。〔長生久視之〕道也。

    帛书乙本:

    治人事天莫若嗇。夫唯嗇。是以蚤服。蚤服是謂重積〔德〕。重積〔德則無不克。無不克則〕莫知其〔極〕。莫知其〔極可以〕有國。有國之母。可〔以長久〕。是謂〔深〕根固氐。長生久視之道也。

    楚简乙本:

    紿(治)人事天。莫若嗇。夫唯嗇。是以󶵙(早)是以󶵙備(服)。是謂⋯⋯[無]不克。[無]不克則莫智其𠄨(極)。莫智其𠄨(極)。可以又䧕。又䧕之母。可以長⋯⋯長生舊(久)視之道也。

    \chapter{}
    王弼本:

    治大國若烹小鮮。以道莅天下。其鬼不神。非其鬼不神。其神不傷人。非其神不傷人。聖人亦不傷人。夫兩不相傷。故德交歸焉。

    
    帛书甲本:

    〔治大國若烹小鮮。以道立〕天下。其鬼不神。非其鬼不神也。其神不傷人也。非其神不傷人也。聖人亦弗傷〔也〕。〔夫兩〕不相〔傷。故〕德交歸焉。

    帛书乙本:

    治大國若烹小鮮。以道立天下。其鬼不神。非其鬼不神也。其神不傷人也。非其神不傷人也。〔聖人亦〕弗傷也。夫兩〔不〕相傷。故德交歸焉。

    \chapter{}
    王弼本:

    大國者下流。天下之交。天下之牝。牝常以靜勝牡。以靜爲下。故大國以下小國。則取小國。小國以下大國。則取大國。故或下以取。或下而取。大國不過欲兼畜人。小國不過欲入事人。夫兩者各得其所欲。大者宜爲下。

    
    帛书甲本:

    大邦者下流也。天下之牝。天下之郊也。牝恒以靚勝牡。爲其靚〔也。故〕宜爲下。大邦〔以〕下小〔國〕。則取小邦。小邦以下大邦。則取於大邦。故或下以取。或下而取。〔故〕大邦者不過欲兼畜人。小邦者不過欲入事人。夫皆得其欲。〔大者宜〕爲下。

    帛书乙本:

    大國〔者下流也。天下之〕牝也。天下之交也。牝恒以静勝牡。爲其静也。故宜爲下也。故大國以下〔小〕國。則取小國。小國以下大國。則取於大國。故或下〔以取。或〕下而取。故大國者不〔過〕欲并畜人。小國不過欲入事人。夫〔皆得〕其欲。則大者宜爲下。

    \chapter{}
    王弼本:

    道者萬物之奥。善人之寶。不善人之所保。美言可以市。尊行可以加人。人之不善。何棄之有。故立天子。置三公。雖有拱璧以先駟馬。不如坐進此道。古之所以貴此道者何不曰以求得。有罪以免邪。故爲天下貴。

    
    帛书甲本:

    〔道〕者萬物之注也。善人之寶也。不善人之所󱀘也。美言可以市。尊行可以賀人。人之不善也。何〔棄之〕有。故立天子。置三卿。雖有共之璧以先四馬。不善坐而進此。古之所以貴此者何也。不謂〔求以〕得。有罪以免輿。故爲天下貴。

    帛书乙本:

    道者萬物之注也。善人之寶也。不善人之所保也。美言可以市。尊行可以賀人。人之不善。何〔棄之有〕。〔故〕立天子。置三鄉。雖有〔共之〕璧以先四馬。不若坐而進此。古〔之所以貴此者何也〕。不謂求以得。有罪以免與。故爲天下貴。

    \chapter{}
    王弼本:

    爲無爲。事無事。味無味。大小多少。報怨以德。圖難於其易。爲大於其細。天下難事必作於易。天下大事必作於細。是以聖人終不爲大。故能成其大。夫輕諾必寡信。多易必多難。是以聖人猶難之。故終無難矣。

    
    帛书甲本:

    爲无爲。事无事。味无未。大小多少。報怨以德。圖難乎〔其易也。爲大乎其細也〕。天下之難作於易。天下之大作於細。是以聖人冬不爲大。故能〔成其大〕。〔夫輕若必寡信。多易〕必多難。是〔以聖〕人猶難之。故終於无難。

    帛书乙本:

    爲无爲。〔事无事。味无未。大小多少。報怨以德〕。〔圖難乎其易也。爲大〕乎其細也。天下之〔難作於〕易。天下之大〔作於細。是以聖人冬不爲大。故能成其大〕。夫輕若〔必寡〕信。多易必多難。是以聖人〔猶難〕之。故〔終於无難〕。

    楚简甲本:

    爲亡爲。事亡事。未亡未。大少之多易必多難。是以聖人猶難之。古終亡難。

    \chapter{}
    王弼本:

    其安易持。其未兆易謀。其脆易泮。其微易散。爲之於未有。治之於未亂。合抱之木。生於毫末。九層之臺。起於累土。千里之行。始於足下。爲者敗之。執者失之。是以聖人無爲故無敗。無執故無失。民之從事。常於幾成而敗之。慎終如始。則無敗事。是以聖人欲不欲。不貴難得之貨。學不學。復衆人之所過。以輔萬物之自然。而不敢爲。

    
    帛书甲本:

    其安也。易持也。⋯⋯。〔合抱之木。生於〕毫末。九成之臺。作於羸土。百仁之高。台於足〔下〕。〔爲之者敗之。執者失之。是以聖人无爲〕也。〔故〕无敗〔也〕。无執也。故无失也。民之從事也。恒於其成事而敗之。故慎終若始。則〔无敗事矣〕。〔是以聖人〕欲不欲。而不貴難得之貨。學不學。而復衆人之所過。能輔萬物之自〔然。而〕弗敢爲。

    帛书乙本:

    ⋯⋯。〔合抱之〕木。生於毫末。九成之臺。作於虆土。百千之高。始於足下。爲之者敗之。執者失之。是以聖人无爲〔也。故无敗也。无執也。故无失也〕。民之從事也。恒於其成而敗之。故曰慎冬若始。則无敗事矣。是以聖人欲不欲。而不貴難得之貨。學不學。復衆人之所過。能輔萬物之自然。而弗敢爲。

    楚简甲本:

    其安也。易杫也。其未󶵆(兆)也。易𢘃(謀)也。其毳(脆)也。易畔(判)也。其幾也。易㣤(散)也。爲之於其亡又也。𥿆(治)之於其未亂。𣌭(合)☐☐☐☐☐☐末。九成之臺。甲☐☐☐☐☐☐☐☐☐足下。

    爲之者敗之。執之者遠之。是以聖人亡爲古亡敗。亡執古亡失。臨事之紀。誓(慎)冬(終)女󶴢(始)。此亡敗事矣。

    聖人谷(欲)不谷(欲)。不貴難󶴫(得)之貨。𡥈(學)不𡥈(學)。復眾之所󶴬(過)。是古聖人能尃(輔)萬勿之自然而弗能爲。

    楚简丙本:

    爲之者敗之。執之者失之。聖人無爲。古無敗也。無執。古☐☐☐。󶴤(慎)終若󶴪(始)。則無敗事喜(矣)。人之敗也。𠄨(恒)於其𠭯(且)成也敗之。是以☐人欲不欲。不貴難得之貨。學不學。復眾之所󶴭(過)。是以能㭪(輔)󼧕(萬)勿之自然而弗敢爲。

    \chapter{}
    王弼本:

    古之善爲道者。非以明民。將以愚之。民之難治。以其智多。故以智治國。國之賊。不以智治國。國之福。知此兩者亦稽式。常知稽式。是謂玄德。玄德深矣。遠矣。與物反矣。然後乃至大順。

    
    帛书甲本:

    故曰爲道者非以明民也。將以愚之也。民之難〔治也。以其〕知也。故以知知邦。邦之賊也。以不知知邦。〔國之〕德也。恒知此兩者。亦稽式也。恒知稽式。此謂玄德。玄德深矣。遠矣。與物〔反〕矣。乃至大順。

    帛书乙本:

    古之爲道者。非以明〔民也。將以愚〕之也。夫民之難治也。以其知也。故以知知國。國之賊也。以不知知國。國之德也。恒知此兩者。亦稽式也。恒知稽式。是謂玄德。玄德深矣。遠矣。〔與〕物反也。乃至大順。

    \chapter{}
    王弼本:

    江海所以能爲百谷王者。以其善下之。故能爲百谷王。是以欲上民。必以言下之。欲先民。必以身後之。是以聖人處上而民不重。處前而民不害。是以天下樂推而不厭。以其不爭。故天下莫能與之爭。

    
    帛书甲本:

    〔江〕海之所以能爲百浴王者。以其善下之。是以能爲百浴王。是以聖人之欲上民也。必以其言下之。其欲先〔民也〕。必以其身後之。故居前而民弗害也。居上而民弗重也。天下樂隼而弗猒也。非以其无静與。〔故天下莫能與〕静。

    帛书乙本:

    江海所以能爲百浴〔王者。以〕其〔善〕下之也。是以能爲百浴王。是以聖人之欲上民也。必以其言下之。其欲先民也。必以其身後之。故居上而民弗重也。居前而民弗害。天下皆樂誰而弗猒也。不以其无争與。故〔天〕下莫能與争。

    楚简甲本:

    江海所以爲百浴王。以其能爲百浴下。是以能爲百浴王。聖人之在民前也。以身後之。其在民上也。以言下之。其在民上也。民弗厚也。其在民前也。民弗害也。天下樂進而弗詀(厭)。以其不靜也。古天下莫能與之靜。

    \chapter{}
    王弼本:

    天下皆謂我道大。似不肖。夫唯大。故似不肖。若肖久矣其細也夫。我有三寶。持而保之。一曰慈。二曰儉。三曰不敢爲天下先。慈故能勇。儉故能廣。不敢爲天下先。故能成器長。今舍慈且勇。舍儉且廣。舍後且先。死矣。夫慈。以戰則勝。以守則固。天將救之。以慈衛之。

    
    帛书甲本:

    〔天下皆謂我大。大而不宵〕。夫唯〔大〕。故不宵。若宵。細久矣。我恒有三寶之。一曰兹。二曰檢。〔三曰不敢爲天下先〕。〔夫兹故能勇。檢〕故能廣。不敢爲天下先。故能爲成事長。今舍其兹且勇。舍其後且先。則必死矣。夫兹。〔以戰〕則勝。以守則固。天將建之。女以兹垣之。

    帛书乙本:

    天下〔皆〕謂我大。大而不宵。夫唯不宵。故能大。若宵。久矣其細也夫。我恒有三寶。市而寶之。一曰兹。二曰檢。三曰不敢爲天下先。夫兹故能勇。檢敢能廣。不敢爲天下先。故能爲成器長。今舍其兹且勇。舍其檢且廣。舍其後且先。則死矣。夫兹。以戰則勝。以守則固。天將建之。如以兹垣之。

    \chapter{}
    王弼本:

    善爲士者不武。善戰者不怒。善勝敵者不與。善用人者爲之下。是謂不爭之德。是謂用人之力。是謂配天古之極。

    
    帛书甲本:

    善爲士者不武。善戰者不怒。善勝敵者弗〔與〕。善用人者爲之下。〔是〕謂不諍之德。是謂用人。是謂天古之極也。

    帛书乙本:

    故善爲士者不武。善戰者不怒。善勝敵者弗與。善用人者爲之下。是謂不爭〔之〕德。是謂用人。是謂肥天古之極也。

    \chapter{}
    王弼本:

    用兵有言。吾不敢爲主而爲客。不敢進寸而退尺。是謂行無行。攘無臂。扔無敵。執無兵。禍莫大於輕敵。輕敵幾喪吾寶。故抗兵相加。哀者勝矣。

    
    帛书甲本:

    用兵有言曰。吾不敢爲主而爲客。吾不進寸而退尺。是謂行无行。襄无臂。執无兵。乃无敵矣。禍莫於於无適。无適斤亡吾吾寶矣。故稱兵相若。則哀者勝矣。

    帛书乙本:

    用兵又言曰。吾不敢爲主而爲客。不敢進寸而退尺。是謂行无行。攘无臂。執无兵。乃无敵。禍莫大於無敵。無敵近亡吾寶矣。故抗兵相若。而哀者勝〔矣〕。

    \chapter{}
    王弼本:

    吾言甚易知。甚易行。天下莫能知。莫能行。言有宗。事有君。夫唯無知。是以不我知。知我者希。則我者貴。是以聖人被褐懷玉。

    
    帛书甲本:

    吾言甚易知也。甚易行也。而人莫之能知也。而莫之能行也。言有君。事有宗。夫唯无知也。是以不〔我知。知我者希。則〕我貴矣。是以聖人被褐而懷玉。

    帛书乙本:

    吾言易知也。易行也。而天下莫之能知也。莫之能行也。夫言又宗。事又君。夫唯无知也。是以不我知。知者希。則我貴矣。是以聖人被褐而懷玉。

    \chapter{}
    王弼本:

    知不知上。不知知病。夫唯病病。是以不病。聖人不病。以其病病。是以不病。

    
    帛书甲本:

    知不知尚矣。不知不知病矣。是以聖人之不病。以其〔病病。是以不病〕。

    帛书乙本:

    知不知尚矣。不知知病矣。是以聖人之不〔病〕也。以其病病也。是以不病。

    \chapter{}
    王弼本:

    民不畏威。則大威至。無狎其所居。無厭其所生。夫唯不厭。是以不厭。是以聖人自知不自見。自愛不自貴。故去彼取此。

    
    帛书甲本:

    〔民之不〕畏畏。則大〔畏將至〕矣。毋閘其所居。毋猒其所生。夫唯弗猒。是〔以不猒〕。〔是以聖人自知而不自見也。自愛〕而不自貴也。故去彼取此。

    帛书乙本:

    民之不畏畏。則大畏將至矣。毋𠇺其所居。毋猒其所生。夫唯弗猒。是以不猒。是以聖人自知而不自見也。自愛而不自貴也。故去彼而取此。

    \chapter{}
    王弼本:

    勇於敢則殺。勇於不敢則活。此兩者或利或害。天之所惡。孰知其故。是以聖人猶難之。天之道。不爭而善勝。不言而善應。不召而自來。繟然而善謀。天網恢恢。疏而不失。

    
    帛书甲本:

    勇於敢者〔則殺。勇〕於不敢者則栝。〔此兩者或利或害。天之所惡。孰知其故〕。〔天之道。不戰而善勝〕。不言而善應。不召而自來。彈而善謀。〔天網恢恢。疏而不失〕。

    帛书乙本:

    勇於敢則殺。勇於不敢則栝。〔此〕兩者或利或害。天之所惡。孰知其故。天之道。不戰而善勝。不言而善應。弗召而自來。戰而善謀。天網恢恢。疏而不失。

    \chapter{}
    王弼本:

    民不畏死。奈何以死懼之。若使民常畏死。而爲奇者吾得執而殺之。孰敢。常有司殺者殺。夫代司殺者殺。是謂代大匠斲。夫代大匠斲者。希有不傷其手矣。

    
    帛书甲本:

    〔若民恒且不畏死〕。奈何以殺愳之也。若民恒是死。則而爲者吾將得而殺之。夫孰敢矣。若民〔恒且〕必畏死。則恒有司殺者。夫伐司殺者殺。是伐大匠斲也。夫伐大匠斲者。則〔希〕不傷其手矣。

    帛书乙本:

    若民恒且畏不畏死。若何以殺䂂之也。使民恒且畏死。而爲畸者〔吾〕得而殺之。夫孰敢矣。若民恒且必畏死。則恒又司殺者。夫代司殺者殺。是代大匠斲。夫代大匠斲。則希不傷其手。

    \chapter{}
    王弼本:

    民之饑。以其上食税之多。是以饑。民之難治。以其上之有爲。是以難治。民之輕死。以其求生之厚。是以輕死。夫唯無以生爲者。是賢於貴生。

    
    帛书甲本:

    人之飢也。以其取食税之多也。是以飢。百姓之不治也。以其上有以爲〔也〕。是以不治。民之輕死。以其求生之厚也。是以輕死。夫唯无以生爲者。是賢貴生。

    帛书乙本:

    人之飢也。以其取食税之多。是以飢。百姓之不治也。以其上之有以爲也。〔是〕以不治。民之輕死也。以其求生之厚也。是以輕死。夫唯无以生爲者。是賢貴生。

    \chapter{}
    王弼本:

    人之生也柔弱。其死也堅强。萬物草木之生也柔脆。其死也枯槁。故堅强者死之徒。柔弱者生之徒。是以兵强則不勝。木强則兵。强大處下。柔弱處上。

    
    帛书甲本:

    人之生也柔弱。其死也𦵕仞賢强。萬物草木之生也柔脆。其死也枯槁。故曰堅强者死之徒也。柔弱微細生之徒也。兵强則不勝。木强則恒。强大居下。柔弱微細居上。

    帛书乙本:

    人之生也柔弱。其死也󱁌信堅强。萬〔物草〕木之生也柔脆。其死也枯槁。故曰堅强死之徒也。柔弱生之徒也。〔是〕以兵强則不勝。木强則競。故强大居下。柔弱居上。

    \chapter{}
    王弼本:

    天之道。其猶張弓與。高者抑之。下者舉之。有餘者損之。不足者補之。天之道損有餘而補不足。人之道則不然。損不足以奉有餘。孰能有餘以奉天下。唯有道者。是以聖人爲而不恃。功成而不處。其不欲見賢。

    
    帛书甲本:

    天下〔之道。猶張弓〕者也。高者印之。下者舉之。有餘者損之。不足者補之。故天之道。損有〔余而益不足。人之道則〕不然。損〔不足而〕奉有餘。孰能有餘而有以取奉於天者乎。〔唯有道者乎〕。〔是以聖人爲而弗又。成功而弗居也。若此其不欲〕見賢也。

    帛书乙本:

    天之道。猶張弓也。高者印之。下者舉之。有余者損之。不足者〔補之〕。〔故天之道〕。損有余而益不足。人之道。損不足而奉又余。夫孰能又余而〔有以取〕奉於天者。唯又道者乎。是以聖人爲而弗又。成功而弗居也。若此其不欲見賢也。

    \chapter{}
    王弼本:

    天下莫柔弱於水。而攻堅强者莫之能勝。其無以易之。弱之勝强。柔之勝剛。天下莫不知。莫能行。是以聖人云。受國之垢。是謂社稷主。受國不祥。是爲天下王。正言若反。

    
    帛书甲本:

    天下莫柔〔弱於水。而攻〕堅强者莫之能〔勝〕也。以其无〔以〕易〔之也〕。〔水之勝剛也。弱之〕勝强。天〔下莫弗知也。而莫能〕行也。故聖人之言云曰。受邦之訽。是謂社稷之主。受邦之不祥。是謂天下之王。〔正言〕若反。

    帛书乙本:

    天下莫柔弱於水。〔而攻堅强者莫之能勝〕。以其无以易之也。水之勝剛也。弱之勝强也。天下莫弗知也。而〔莫能行〕也。是故聖人言云曰。受國之訽。是謂社稷之主。受國之不祥。是謂天下之王。正言若反。

    \chapter{}
    王弼本:

    和大怨。必有餘怨。安可以爲善。是以聖人執左契。而不責於人。有德司契。無德司徹。天道無親。常與善人。

    
    帛书甲本:

    和大怨。必有餘怨。焉可以爲善。是以聖右契。而不以責於人。故有德司契。〔无〕德司勶。夫天道无親。恒與善人。

    帛书乙本:

    禾大〔怨。必有餘怨。焉可以〕爲善。是以聖人執左契。而不以責於人。故又德司契。无德司勶。〔夫天道无親。恒與善人〕。

    \chapter{}
    王弼本:

    小國寡民。使有什伯之器而不用。使民重死而不遠徙。雖有舟輿。無所乘之。雖有甲兵。無所陳之。使人復結繩而用之。甘其食。美其服。安其居。樂其俗。鄰國相望。鷄犬之聲相聞。民至老死不相往來。

    
    帛书甲本:

    小邦寡民。使十百人之器毋用。使民重死而遠徙。有車舟无所乘之。有甲兵无所陳〔之。使民復結繩而〕用之。甘其食。美其服。樂其俗。安其居。鄰邦相望。鷄狗之聲相聞。民至〔老死不相往來〕。

    帛书乙本:

    小國寡民。使有十百人器而勿用。使民重死而遠徙。又舟車无所乘之。有甲兵无所陳之。使民復結繩而用之。甘其食。美其服。樂其俗。安其居。鄰國相望。鷄犬之〔聲相〕聞。民至老死不相往來。

    \chapter{}
    王弼本:

    信言不美。美言不信。善者不辯。辯者不善。知者不博。博者不知。聖人不積。既以爲人。己愈有。既以與人。己愈多。天之道。利而不害。聖人之道。爲而不爭。

    
    帛书甲本:

    〔信言不美。美言〕不〔信。知〕者不博。〔博〕者不知。善〔者不多。多〕者不善。聖人无積。〔既〕以爲〔人。己愈有。既以予人矣。己愈多〕。⋯⋯。

    帛书乙本:

    信言不美。美言不信。知者不博。博者不知。善者不多。多者不善。聖人无積。既以爲人。己愈有。既以予人矣。己愈多。故天之道。利而不害。人之道。爲而弗爭。

    \chapter*{参考文献}
    《郭店楚簡老子集釋》,彭裕商、吴毅强 集釋,巴蜀書社。

    《老子道德經河上公章句》,王卡 點校,中華書局。

    《老子道德經注校釋》,樓宇烈 校釋,中華書局。
\end{document}
