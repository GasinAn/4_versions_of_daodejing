\documentclass[a5paper]{ctexbook}
\usepackage{geometry}
\geometry{a5paper}
\title{《道德经》四版对照}
\author{安梯西登}
\date{}
\begin{document}
    \maketitle

    \tableofcontents

    \chapter{}
    王弼本:

    道可道。非常道。名可名。非常名。無名天地之始。有名萬物之母。故常無欲。以觀其妙。常有欲。以觀其徼。此兩者同出而異名。同謂之玄。玄之又玄。衆妙之門。

    河上公本:

    道可道。非常道。名可名。非常名。無名天地之始。有名萬物之母。故常無欲。以觀其妙。常有欲。以觀其徼。此兩者同出而異名。同謂之玄。玄之又玄。衆妙之門。

    帛书甲本:

    道可道也。非恒道也。名可名也。非恒名也。无名萬物之始也。有名萬物之母也。〔故〕垣无欲也。以觀其眇。恒有欲也。以觀其所噭。兩者同出。異名同胃。玄之有玄。衆眇之〔門〕。

    帛书乙本:

    道可道也。〔非恒道也。名可名也。非〕恒名也。无名萬物之始也。有名萬物之母也。故恒无欲也。〔以觀其眇〕。恒又欲也。以觀其所噭。兩者同出。異名同胃。玄之又玄。衆眇之門。

    \chapter{}
    王弼本:

    天下皆知美之爲美。斯惡已。皆知善之爲善。斯不善已。故有無相生。難易相成。長短相較。高下相傾。音聲相和。前後相隨。是以聖人處無爲之事。行不言之教。萬物作焉而不辭。生而不有。爲而不恃。功成而弗居。夫唯弗居。是以不去。

    河上公本:

    天下皆知美之爲美。斯惡已。皆知善之爲善。斯不善已。故有無相生。難易相成。長短相形。高下相傾。音聲相和。前後相隨。是以聖人處無爲之事。行不言之教。萬物作焉而不辭。生而不有。爲而不恃。功成而弗居。夫惟弗居。是以不去。

    帛书甲本:

    天下皆知美爲美。惡已。皆知善。訾不善矣。有无之相生也。難易之相成也。長短之相刑也。高下之相盈也。意聲之相和也。先後之相隋。恒也。是以聲人居无爲之事。行〔不言之教〕。〔萬物昔而弗始〕也。爲而弗志也。成功而弗居也。夫唯居。是以弗去。

    帛书乙本:

    天下皆知美之爲美。亞已。皆知善。斯不善矣。〔有无之相〕生也。難易之相成也。長短之相刑也。高下之相盈也。音聲之相和也。先後之相隋。恒也。是以𦔻人居无爲之事。行不言之教。萬物昔而弗始。爲而弗侍也。成功而弗居也。夫唯弗居。是以弗去。

    \chapter{}
    王弼本:

    不尚賢。使民不爭。不貴難得之貨。使民不爲盗。不見可欲。使民心不亂。是以聖人之治。虚其心。實其腹。弱其志。强其骨。常使民無知無欲。使夫智者不敢爲也。爲無爲。則無不治。

    河上公本:

    不尚賢。使民不爭。不貴難得之貨。使民不爲盗。不見可欲。使心不亂。是以聖人〔之〕治。虚其心。實其腹。弱其志。强其骨。常使民無知無欲。使夫智者不敢爲也。爲無爲。則無不治。

    帛书甲本:

    不上賢。〔使民不争。不貴難得之貨。使〕民不爲〔盗。不見可欲。使〕民不亂。是以聲人之〔治也。虚其心。實其腹。弱其志〕。强其骨。〔恒〕使民无知无欲也。使〔夫知不敢。弗爲而已。則无不治矣〕。

    帛书乙本:

    不上賢。使民不争。不貴難得之貨。使民不爲盗。不見可欲。使民不亂。是以𦔻人之治也。虚其心。實其腹。弱其志。强其骨。恒使民无知无欲也。使夫知不敢。弗爲而已。則无不治矣。

    \chapter{}
    王弼本:

    道沖而用之或不盈。淵兮似萬物之宗。挫其銳。解其紛。和其光。同其塵。湛兮似或存。吾不知誰之子。象帝之先。

    河上公本:

    道冲而用之或不盈。淵乎似萬物之宗。挫其鋭。解其紛。和其光。同其塵。湛兮似若存。吾不知誰之子。象帝之先。

    帛书甲本:

    〔道沖而用之有弗〕盈也。潚呵始萬物之宗。銼其。解其紛。和其光。同〔其塵〕。〔湛呵似〕或存。吾不知〔其誰之〕子也。象帝之先。

    帛书乙本:

    道沖而用之有弗盈也。淵呵似萬物之宗。銼其兑。解其芬。和其光。同其塵。湛呵似或存。吾不知其誰之子也。象帝之先。

    \chapter{}
    王弼本:

    天地不仁。以萬物爲芻狗。聖人不仁。以百姓爲芻狗。天地之間。其猶橐籥乎。虚而不屈。動而愈出。多言數窮。不如守中。

    河上公本:

    天地不仁。以萬物爲芻狗。聖人不仁。以百姓爲芻狗。天地之間。其猶槖籥乎。虚而不屈。動而愈出。多言數窮。不如守中。

    帛书甲本:

    天地不仁。以萬物爲芻狗。聲人不仁。以百省〔爲芻〕狗。天地〔之間。其〕猶橐籥與。虚而不淈。蹱而俞出。多言數窮。不如守中。

    帛书乙本:

    天地不仁。以萬物爲芻狗。𦔻人不仁。〔以〕百姓爲芻狗。天地之間。其猶橐籥與。虚而不淈。勭而俞出。多言數窮。不如守中。

    \chapter{}
    王弼本:

    谷神不死。是謂玄牝。玄牝之門。是謂天地根。緜緜若存。用之不勤。

    河上公本:

    谷神不死。是謂玄牝。玄牝之門。是謂天地根。綿綿若存。用之不勤。

    帛书甲本:

    浴神〔不〕死。是胃玄牝。玄牝之門。是胃〔天〕地之根。緜緜呵若存。用之不堇。

    帛书乙本:

    浴神不死。是胃玄牝。玄牝之門。是胃天地之根。緜緜呵其若存。用之不堇。

    \chapter*{参考文献}
    \small
    楚简版:《郭店楚簡老子集釋》。彭裕商、吴毅强 集釋。巴蜀書社

    帛书版:《帛書老子校注》。高明 撰。中華書局

    河上公版:《老子道德經河上公章句》。王卡 點校。中華書局

    王弼版:《老子道德經注校釋》。樓宇烈 校釋。中華書局
\end{document}
