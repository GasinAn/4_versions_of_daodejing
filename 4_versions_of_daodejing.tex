\documentclass[a5paper]{ctexbook}
\usepackage{geometry}
\geometry{a5paper}
\title{《道德经》四版对照}
\author{安梯西登}
\date{}
\begin{document}
    \maketitle

    \tableofcontents

    \chapter{}
    王弼本:

    道可道。非常道。名可名。非常名。無名天地之始。有名萬物之母。故常無欲。以觀其妙。常有欲。以觀其徼。此兩者同出而異名。同謂之玄。玄之又玄。衆妙之門。

    河上公本:

    道可道。非常道。名可名。非常名。無名天地之始。有名萬物之母。故常無欲。以觀其妙。常有欲。以觀其徼。此兩者同出而異名。同謂之玄。玄之又玄。衆妙之門。

    帛书甲本:

    道可道也。非恒道也。名可名也。非恒名也。无名萬物之始也。有名萬物之母也。〔故〕垣无欲也。以觀其眇。恒有欲也。以觀其所噭。兩者同出。異名同胃。玄之有玄。衆眇之〔門〕。

    帛书乙本:

    道可道也。〔非恒道也。名可名也。非〕恒名也。无名萬物之始也。有名萬物之母也。故恒无欲也。〔以觀其眇〕。恒又欲也。以觀其所噭。兩者同出。異名同胃。玄之又玄,衆眇之門。

    \chapter*{参考文献}
    \small
    楚简版:《郭店楚簡老子集釋》,彭裕商、吴毅强 集釋,巴蜀書社

    帛书版:《帛書老子校注》,高明 撰,中華書局

    河上公版:《老子道德經河上公章句》,王卡 點校,中華書局

    王弼版:《老子道德經注校釋》,樓宇烈 校釋,中華書局
\end{document}
