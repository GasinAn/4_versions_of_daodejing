\documentclass[a5paper]{ctexbook}
\usepackage{geometry}
\geometry{a5paper}
\title{《道德经》四版对照}
\author{安梯西登}
\date{}
\begin{document}
    \maketitle

    \tableofcontents

    \chapter{}
    王弼本:

    道可道。非常道。名可名。非常名。無名天地之始。有名萬物之母。故常無欲。以觀其妙。常有欲。以觀其徼。此兩者同出而異名。同謂之玄。玄之又玄。衆妙之門。

    河上公本:

    道可道。非常道。名可名。非常名。無名天地之始。有名萬物之母。故常無欲。以觀其妙。常有欲。以觀其徼。此兩者同出而異名。同謂之玄。玄之又玄。衆妙之門。

    帛书甲本:

    道可道也。非恒道也。名可名也。非恒名也。无名萬物之始也。有名萬物之母也。〔故〕垣无欲也。以觀其眇。恒有欲也。以觀其所噭。兩者同出。異名同胃。玄之有玄。衆眇之〔門〕。

    帛书乙本:

    道可道也。〔非恒道也。名可名也。非〕恒名也。无名萬物之始也。有名萬物之母也。故恒无欲也。〔以觀其眇〕。恒又欲也。以觀其所噭。兩者同出。異名同胃。玄之又玄。衆眇之門。

    \chapter{}
    王弼本:

    天下皆知美之爲美。斯惡已。皆知善之爲善。斯不善已。故有無相生。難易相成。長短相較。高下相傾。音聲相和。前後相隨。是以聖人處無爲之事。行不言之教。萬物作焉而不辭。生而不有。爲而不恃。功成而弗居。夫唯弗居。是以不去。

    河上公本:

    天下皆知美之爲美。斯惡已。皆知善之爲善。斯不善已。故有無相生。難易相成。長短相形。高下相傾。音聲相和。前後相隨。是以聖人處無爲之事。行不言之教。萬物作焉而不辭。生而不有。爲而不恃。功成而弗居。夫惟弗居。是以不去。

    帛书甲本:

    天下皆知美爲美。惡已。皆知善。訾不善矣。有无之相生也。難易之相成也。長短之相刑也。高下之相盈也。意聲之相和也。先後之相隋。恒也。是以聲人居无爲之事。行〔不言之教〕。〔萬物昔而弗始〕也。爲而弗志也。成功而弗居也。夫唯居。是以弗去。

    帛书乙本:

    天下皆知美之爲美。亞已。皆知善。斯不善矣。〔有无之相〕生也。難易之相成也。長短之相刑也。高下之相盈也。音聲之相和也。先後之相隋。恒也。是以𦔻人居无爲之事。行不言之教。萬物昔而弗始。爲而弗侍也。成功而弗居也。夫唯弗居。是以弗去。

    \chapter{}
    王弼本:

    不尚賢。使民不爭。不貴難得之貨。使民不爲盗。不見可欲。使民心不亂。是以聖人之治。虚其心。實其腹。弱其志。强其骨。常使民無知無欲。使夫智者不敢爲也。爲無爲。則無不治。

    河上公本:

    不尚賢。使民不爭。不貴難得之貨。使民不爲盗。不見可欲。使心不亂。是以聖人〔之〕治。虚其心。實其腹。弱其志。强其骨。常使民無知無欲。使夫智者不敢爲也。爲無爲。則無不治。

    帛书甲本:

    不上賢。〔使民不争。不貴難得之貨。使〕民不爲〔盗。不見可欲。使〕民不亂。是以聲人之〔治也。虚其心。實其腹。弱其志〕。强其骨。〔恒〕使民无知无欲也。使〔夫知不敢。弗爲而已。則无不治矣〕。

    帛书乙本:

    不上賢。使民不争。不貴難得之貨。使民不爲盗。不見可欲。使民不亂。是以𦔻人之治也。虚其心。實其腹。弱其志。强其骨。恒使民无知无欲也。使夫知不敢。弗爲而已。則无不治矣。

    \chapter{}
    王弼本:

    道沖而用之或不盈。淵兮似萬物之宗。挫其銳。解其紛。和其光。同其塵。湛兮似或存。吾不知誰之子。象帝之先。

    河上公本:

    道冲而用之或不盈。淵乎似萬物之宗。挫其鋭。解其紛。和其光。同其塵。湛兮似若存。吾不知誰之子。象帝之先。

    帛书甲本:

    〔道沖而用之有弗〕盈也。潚呵始萬物之宗。銼其。解其紛。和其光。同〔其塵〕。〔湛呵似〕或存。吾不知〔其誰之〕子也。象帝之先。

    帛书乙本:

    道沖而用之有弗盈也。淵呵似萬物之宗。銼其兑。解其芬。和其光。同其塵。湛呵似或存。吾不知其誰之子也。象帝之先。

    \chapter{}
    王弼本:

    天地不仁。以萬物爲芻狗。聖人不仁。以百姓爲芻狗。天地之間。其猶橐籥乎。虚而不屈。動而愈出。多言數窮。不如守中。

    河上公本:

    天地不仁。以萬物爲芻狗。聖人不仁。以百姓爲芻狗。天地之間。其猶槖籥乎。虚而不屈。動而愈出。多言數窮。不如守中。

    帛书甲本:

    天地不仁。以萬物爲芻狗。聲人不仁。以百省〔爲芻〕狗。天地〔之間。其〕猶橐籥與。虚而不淈。蹱而俞出。多言數窮。不如守中。

    帛书乙本:

    天地不仁。以萬物爲芻狗。𦔻人不仁。〔以〕百姓爲芻狗。天地之間。其猶橐籥與。虚而不淈。勭而俞出。多言數窮。不如守中。

    \chapter{}
    王弼本:

    谷神不死。是謂玄牝。玄牝之門。是謂天地根。緜緜若存。用之不勤。

    河上公本:

    谷神不死。是謂玄牝。玄牝之門。是謂天地根。綿綿若存。用之不勤。

    帛书甲本:

    浴神〔不〕死。是胃玄牝。玄牝之門。是胃〔天〕地之根。緜緜呵若存。用之不堇。

    帛书乙本:

    浴神不死。是胃玄牝。玄牝之門。是胃天地之根。緜緜呵其若存。用之不堇。

    \chapter{}
    王弼本:

    天長地久。天地所以能長且久者。以其不自生。故能長生。是以聖人後其身而身先。外其身而身存。非以其無私邪。故能成其私。

    河上公本:

    天長地久。天地所以能長且久者。以其不自生。故能長生。是以聖人後其身而身先。外其身而身存。非以其無私耶。故能成其私。

    帛书甲本:

    天長地久。天地之所以能〔長〕且久者。以其不自生也。故能長生。是以聲人芮其身而身先。外其身而身存。不以其无〔私〕輿。故能成其私。

    帛书乙本:

    天長地久。天地之所以能長且久者。以其不自生也。故能長生。是以𦔻人退其身而身先。外其身而身先。外其身而身存。不以其无私輿。故能成其私。

    \chapter{}
    王弼本:

    上善若水。水善利萬物而不爭。處衆人之所惡。故幾於道。居善地。心善淵。與善仁。言善信。正善治。事善能。動善時。夫唯不爭。故無尤。

    河上公本:

    上善若水。水善利萬物而不爭。處衆人之所惡。故幾於道。居善地。心善淵。與善仁。言善信。正善治。事善能。動善時。夫唯不爭。故無尤。

    帛书甲本:

    上善治水。水善利萬物而有静。居衆之所惡。故幾於道矣。居善地。心善潚。予善信。正善治。事善能。蹱善時。夫唯不静。故无尤。

    帛书乙本:

    上善如水。水善利萬物而有争。居衆人之所亞。故幾於道矣。居善地。心善淵。予善天。言善信。正善治。事善能。動善時。夫唯不争。故无尤。

    \chapter{}
    王弼本:

    持而盈之。不如其已。揣而棁之。不可長保。金玉滿堂。莫之能守。富貴而驕。自遺其咎。功遂身退。天之道。

    河上公本:

    持而盈之。不知其已。揣而鋭之。不可長保。金玉滿堂。莫之能守。富貴而驕。自遺其咎。功成名遂身退。天之道。

    帛书甲本:

    㨁而盈之。不〔若其已。揣而〕兑☐之。〔不〕可長葆之。金玉盈室。莫之守也。貴富而䮦。自遺咎也。功述身芮。天〔之道也〕。

    帛书乙本:

    㨁而盈之。不若其已。𢵦而兑之。不可長葆也。金玉〔盈〕室。莫之能守也。貴富而驕。自遺咎也。功遂身退。天之道也。

    \chapter{}
    王弼本:

    載營魄抱一。能無離乎。專氣致柔。能嬰兒乎。滌除玄覽。能無疵乎。愛民治國。能無知乎。天門開闔。能無雌乎。明白四達。能無爲乎。生之畜之。生而不有。爲而不恃。長而不宰。是謂玄德。

    河上公本:

    載營魄抱一。能無離。專氣致柔。能嬰兒。滌除玄覽。能無疵。愛民治國。能無爲。天門開闔能爲雌。明白四達。能無知。生之畜之。生而不有。爲而不恃。長而不宰。是謂玄德。

    帛书甲本:

    〔載營柏抱一。能毋离乎。槫氣至柔〕。能嬰兒乎。脩除玄藍。能毋疵乎。〔愛民栝國。能毋以知乎〕。⋯⋯。生之畜之。生而弗〔有。長而弗宰也。是胃玄〕德。

    帛书乙本:

    載營柏抱一。能毋离乎。槫氣至柔。能嬰兒乎。脩除玄監。能毋有疵乎。愛民栝國。能毋以知乎。天門啟闔。能爲雌乎。明白四達。能毋以知乎。生之畜之。生而弗有。長而弗宰也。是胃玄德。

    \chapter{}
    王弼本:

    三十輻共一轂。當其無。有車之用。埏埴以爲器。當其無。有器之用。鑿户牖以爲室。當其無。有室之用。故有之以爲利。無之以爲用。

    河上公本:

    三十輻共一轂。當其無。有車之用。埏埴以爲器。當其無。有器之用。鑿户牖以爲室。當其無。有室之用。故有之以爲利。無之以爲用。

    帛书甲本:

    卅〔楅同一轂。當〕其无。〔有車〕之用〔也〕。𡑋埴爲器。當其无。有埴器〔之用也〕。〔鑿户牖〕。當其无。有〔室之〕用也。故有之以爲利。无之以爲用。

    帛书乙本:

    卅楅同一轂。當其无。有車之用也。𡑋埴而爲器。當其无。有埴器之用也。鑿户牖。當其无。有室之用也。故有之以爲利。无之以爲用。

    \chapter{}
    王弼本:

    五色令人目盲。五音令人耳聾。五味令人口爽。馳騁畋獵令人心發狂。難得之貨令人行妨。是以聖人爲腹不爲目。故去彼取此。

    河上公本:

    五色令人目盲。五音令人耳聾。五味令人口爽。馳騁田獵令人心發狂。難得之貨令人行妨。是以聖人爲腹不爲目。故去彼取此。

    帛书甲本:

    五色使人目明。馳騁田臘使人〔心發狂〕。難得之𧷴使人之行方。五味使人之口𠷹。五音使人之耳聾。是以聲人之治也。爲腹不〔爲目〕。故去罷耳此。

    帛书乙本:

    五色使人目盲。馳騁田臘使人心發狂。難得之貨使人之行仿。五味使人之口爽。五音使人耳〔聾〕。是以𦔻人之治也。爲腹而不爲目。故去彼而取此。

    \chapter{}
    王弼本:

    寵辱若驚。貴大患若身。何謂寵辱若驚。寵爲下得之若驚失之若驚。是謂寵辱若驚。何謂貴大患若身。吾所以有大患者。爲吾有身。及吾無身。吾有何患。故貴以身爲天下。若可寄天下。愛以身爲天下。若可託天下。

    河上公本:

    寵辱若驚。貴大患若身。何謂寵辱。〔寵爲上〕辱爲下得之若驚失之若驚。是謂寵辱若驚。何謂貴大患若身。吾所以有大患者。爲吾有身。及吾無身。吾有何患。故貴以身爲天下者。則可寄於天下。愛以身爲天下者。乃可以託於天下。

    帛书甲本:

    龍辱若驚。貴大梡若身。苛胃龍辱若驚。龍之爲下得之若驚失〔之〕若驚。是胃龍辱若驚。何胃貴大梡若身。吾所以有大梡者。爲吾有身也。及吾无身。有何梡。故貴爲身於爲天下。若可以𨒙天下矣。愛以身爲天下。女可以寄天下。

    帛书乙本:

    弄辱若驚。貴大患若身。何胃弄辱若驚。弄之爲下也得之若驚失之若驚。是胃弄辱若驚。何胃貴大患若身。吾所以有大患者。爲吾有身也。及吾無身。有何患。故貴爲身於爲天下。若可以橐天下〔矣〕。愛以身爲天下。女可以寄天下矣。

    \chapter*{参考文献}
    \small
    楚简版:《郭店楚簡老子集釋》。彭裕商、吴毅强 集釋。巴蜀書社

    帛书版:《帛書老子校注》。高明 撰。中華書局

    河上公版:《老子道德經河上公章句》。王卡 點校。中華書局

    王弼版:《老子道德經注校釋》。樓宇烈 校釋。中華書局
\end{document}
