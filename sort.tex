\documentclass[a5paper]{ctexbook}
\usepackage{geometry}
\usepackage{hyperref}
\geometry{a5paper}
\title{《道德经》整理}
\author{安梯西登}
\date{}
\begin{document}
    
    \maketitle

    \ \newpage

    道可道。非常道。名可名。非常名。無名天地之始。有名萬物之母。故常無欲。以觀其妙。常有欲。以觀其徼。此兩者同出而異名。同謂之玄。玄之又玄。衆妙之門。

    有物混成。先天地生。寂兮寥兮。獨立不改。周行而不殆。可以爲天下母。吾不知其名。字之曰道。强爲之名曰大。大曰逝。逝曰遠。遠曰反。故道大。天大。地大。王亦大。域中有四大。而王居其一焉。人法地。地法天。天法道。道法自然。

    道沖而用之或不盈。淵兮似萬物之宗。挫其銳。解其紛。和其光。同其塵。湛兮似或存。吾不知誰之子。象帝之先。

    谷神不死。是謂玄牝。玄牝之門。是謂天地根。緜緜若存。用之不勤。

    大道氾兮。其可左右。萬物恃之而生而不辭。功成不名有。衣養萬物而不爲主。常無欲。可名於小。萬物歸焉而不爲主。可名爲大。以其終不自爲大。故能成其大。

    視之不見名曰夷。聽之不聞名曰希。搏之不得名曰微。此三者不可致詰。故混而爲一。其上不皦。其下不昧。繩繩不可名。復歸於無物。是謂無狀之狀。無物之象。是謂惚恍。迎之不見其首。隨之不見其後。執古之道。以御今之有。能知古始。是謂道紀。

    孔德之容。惟道是從。道之爲物。惟恍惟惚。惚兮恍兮。其中有象。恍兮惚兮。其中有物。窈兮冥兮。其中有精。其精甚真。其中有信。自古及今。其名不去。以閲衆甫。吾何以知衆甫之狀哉。以此。

    天下有始。以爲天下母。既得其母。以知其子。既知其子。復守其母。没身不殆。塞其兑。閉其門。終身不勤。開其兑。濟其事。終身不救。見小曰明。守柔曰强。用其光。復歸其明。無遺身殃。是爲習常。

    道生一。一生二。二生三。三生萬物。萬物負陰而抱陽。沖氣以爲和。人之所惡。唯孤寡不穀。而王公以爲稱。故物或損之而益。或益之而損。人之所教。我亦教之。强梁者不得其死。吾將以爲教父。

    天之道。其猶張弓與。高者抑之。下者舉之。有餘者損之。不足者補之。天之道。損有餘而補不足。人之道則不然。損不足以奉有餘。孰能有餘以奉天下。唯有道者。是以聖人爲而不恃。功成而不處。其不欲見賢。

    道常無名。樸雖小。天下莫能臣也。侯王若能守之。萬物將自賓。天地相合以降甘露。民莫之令而自均。始制有名。名亦既有。夫亦將知止。知止可以不殆。譬道之在天下。猶川谷之於江海。

    道常無爲而無不爲。侯王若能守之。萬物將自化。化而欲作。吾將鎮之以無名之樸。無名之樸。夫亦將無欲。不欲以靜。天下將自定。

    道者萬物之奥。善人之寶。不善人之所保。美言可以市。尊行可以加人。人之不善。何棄之有。故立天子。置三公。雖有拱璧以先駟馬。不如坐進此道。古之所以貴此道者何。不曰以求得。有罪以免邪。故爲天下貴。

    反者。道之動。弱者。道之用。天下萬物生於有。有生於無。

    上士聞道。勤而行之。中士聞道。若存若亡。下士聞道。大笑之。不笑不足以爲道。故建言有之。明道若昧。進道若退。夷道若纇。上德若谷。大白若辱。廣德若不足。建德若偷。質真若渝。大方無隅。大器晚成。大音希聲。大象無形。道隱無名。夫唯道善貸且成。

    大成若缺。其用不弊。大盈若沖。其用不窮。大直若屈。大巧若拙。大辯若訥。躁勝寒。靜勝熱。清靜爲天下正。

    天下皆知美之爲美。斯惡已。皆知善之爲善。斯不善已。故有無相生。難易相成。長短相較。高下相傾。音聲相和。前後相隨。是以聖人處無爲之事。行不言之教。萬物作焉而不辭。生而不有。爲而不恃。功成而弗居。夫唯弗居。是以不去。
    
    曲則全。枉則直。窪則盈。敝則新。少則得。多則惑。是以聖人抱一。爲天下式。不自見故明。不自是故彰。不自伐故有功。不自矜故長。夫唯不爭。故天下莫能與之爭。古之所謂曲則全者。豈虚言哉。誠全而歸之。

    企者不立。跨者不行。自見者不明。自是者不彰。自伐者無功。自矜者不長。其在道也。曰餘食贅行。物或惡之。故有道者不處。

    將欲歙之。必固張之。將欲弱之。必固强之。將欲廢之。必固興之。將欲奪之。必固與之。是謂微明。柔弱勝剛强。魚不可脱於淵。國之利器不可以示人。

    信言不美。美言不信。善者不辯。辯者不善。知者不博。博者不知。聖人不積。既以爲人。己愈有。既以與人。己愈多。天之道。利而不害。聖人之道。爲而不爭。

    持而盈之。不如其已。揣而棁之。不可長保。金玉滿堂。莫之能守。富貴而驕。自遺其咎。功遂身退。天之道。

    人之生也柔弱。其死也堅强。萬物草木之生也柔脆。其死也枯槁。故堅强者死之徒。柔弱者生之徒。是以兵强則不勝。木强則兵。强大處下。柔弱處上。

    天下之至柔。馳騁天下之至堅。無有入無閒。吾是以知無爲之有益。不言之教。無爲之益。天下希及之。
    
    天下莫柔弱於水。而攻堅强者莫之能勝。其無以易之。弱之勝强。柔之勝剛。天下莫不知。莫能行。是以聖人云。受國之垢。是謂社稷主。受國不祥。是爲天下王。正言若反。

    上善若水。水善利萬物而不爭。處衆人之所惡。故幾於道。居善地。心善淵。與善仁。言善信。正善治。事善能。動善時。夫唯不爭。故無尤。

    重爲輕根。靜爲躁君。是以聖人終日行不離輜重。雖有榮觀。燕處超然。奈何萬乘之主。而以身輕天下。輕則失本。躁則失君。

    三十輻共一轂。當其無。有車之用。埏埴以爲器。當其無。有器之用。鑿户牖以爲室。當其無。有室之用。故有之以爲利。無之以爲用。

    天地不仁。以萬物爲芻狗。聖人不仁。以百姓爲芻狗。天地之間。其猶橐籥乎。虚而不屈。動而愈出。多言數窮。不如守中。

    天長地久。天地所以能長且久者。以其不自生。故能長生。是以聖人後其身而身先。外其身而身存。非以其無私邪。故能成其私。

    知人者智。自知者明。勝人者有力。自勝者强。知足者富。强行者有志。不失其所者久。死而不亡者壽。
    
    知不知。上。不知知。病。夫唯病病。是以不病。聖人不病。以其病病。是以不病。

    知者不言。言者不知。塞其兑。閉其門。挫其銳。解其分。和其光。同其塵。是謂玄同。故不可得而親。不可得而疎。不可得而利。不可得而害。不可得而貴。不可得而賤。故爲天下貴。

    不出户。知天下。不闚牖。見天道。其出彌遠。其知彌少。是以聖人不行而知。不見而名。不爲而成。

    知其雄。守其雌。爲天下谿。爲天下谿。常德不離。復歸於嬰兒。知其白。守其黑。爲天下式。爲天下式。常德不忒。復歸於無極。知其榮。守其辱。爲天下谷。爲天下谷。常德乃足。復歸於樸。樸散則爲器。聖人用之則爲官長。故大制不割。

    致虚極。守靜篤。萬物並作。吾以觀復。夫物芸芸。各復歸其根。歸根曰靜。是謂復命。復命曰常。知常曰明。不知常。妄作。凶。知常容。容乃公。公乃王。王乃天。天乃道。道乃久。没身不殆。

    載營魄抱一。能無離乎。專氣致柔。能嬰兒乎。滌除玄覽。能無疵乎。愛民治國。能無知乎。天門開闔。能無雌乎。明白四達。能無爲乎。生之。畜之。生而不有。爲而不恃。長而不宰。是謂玄德。

    五色令人目盲。五音令人耳聾。五味令人口爽。馳騁畋獵令人心發狂。難得之貨令人行妨。是以聖人爲腹不爲目。故去彼取此。

    名與身孰親。身與貨孰多。得與亡孰病。是故甚愛必大費。多藏必厚亡。知足不辱。知止不殆。可以長久。

    寵辱若驚。貴大患若身。何謂寵辱若驚。寵。爲下得之若驚。失之若驚。是謂寵辱若驚。何謂貴大患若身。吾所以有大患者。爲吾有身。及吾無身。吾有何患。故貴以身爲天下。若可寄天下。愛以身爲天下。若可託天下。

    古之善爲士者。微妙玄通。深不可識。夫唯不可識。故强爲之容。豫焉若冬涉川。猶兮若畏四鄰。儼兮其若容。涣兮若冰之將釋。敦兮其若樸。曠兮其若谷。混兮其若濁。孰能濁以靜之徐清。孰能安以久動之徐生。保此道者不欲盈。夫唯不盈。故能蔽不新成。

    絶學無憂。唯之與阿。相去幾何。善之與惡。相去若何。人之所畏。不可不畏。荒兮其未央哉。衆人熙熙。如享太牢。如春登臺。我獨泊兮其未兆。如嬰兒之未孩。儽儽兮若無所歸。衆人皆有餘。而我獨若遺。我愚人之心也哉。沌沌兮。俗人昭昭。我獨昏昏。俗人察察。我獨悶悶。澹兮其若海。飂兮若無止。衆人皆有以。而我獨頑似鄙。我獨異於人。而貴食母。

    執大象。天下往。往而不害。安平太。樂與餌。過客止。道之出口。淡乎其無味。視之不足見。聽之不足聞。用之不足既。

    昔之得一者。天得一以清。地得一以寧。神得一以靈。谷得一以盈。萬物得一以生。侯王得一以爲天下貞。其致之。天無以清將恐裂。地無以寧將恐發。神無以靈將恐歇。谷無以盈將恐竭。萬物無以生將恐滅。侯王無以貴高將恐蹶。故貴以賤爲本。高以下爲基。是以侯王自謂孤寡不穀。此非以賤爲本邪。非乎。故致數輿無輿。不欲琭琭如玉。珞珞如石。

    希言自然。故飄風不終朝。驟雨不終日。孰爲此者。天地。天地尚不能久。而況於人乎。故從事於道者。道者同於道。德者同於德。失者同於失。同於道者。道亦樂得之。同於德者。德亦樂得之。同於失者。失亦樂得之。信不足。焉有不信焉。
    
    上德不德。是以有德。下德不失德。是以無德。上德無爲而無以爲。下德爲之而有以爲。上仁爲之而無以爲。上義爲之而有以爲。上禮爲之而莫之應。則攘臂而扔之。故失道而後德。失德而後仁。失仁而後義。失義而後禮。夫禮者。忠信之薄而亂之首。前識者。道之華而愚之始。是以大丈夫處其厚。不居其薄。處其實。不居其華。故去彼取此。

    道生之。德畜之。物形之。勢成之。是以萬物莫不尊道而貴德。道之尊。德之貴。夫莫之命而常自然。故道生之。德畜之。長之。育之。亭之。毒之。養之。覆之。生而不有。爲而不恃。長而不宰。是謂玄德。

    含德之厚。比於赤子。蜂蠆虺蛇不螫。猛獸不據。攫鳥不搏。骨弱筋柔而握固。未知牝牡之合而全作。精之至也。終日號而不嗄。和之至也。知和曰常。知常曰明。益生曰祥。心使氣曰强。物壯則老。謂之不道。不道早已。

    善建者不拔。善抱者不脱。子孫以祭祀不輟。修之於身。其德乃真。修之於家。其德乃餘。修之於鄉。其德乃長。修之於國。其德乃豐。修之於天下。其德乃普。故以身觀身。以家觀家。以鄉觀鄉。以國觀國。以天下觀天下。吾何以知天下然哉。以此。

    治大國若烹小鮮。以道莅天下。其鬼不神。非其鬼不神。其神不傷人。非其神不傷人。聖人亦不傷人。夫兩不相傷。故德交歸焉。
    
    以正治國。以奇用兵。以無事取天下。吾何以知其然哉。以此。天下多忌諱。而民彌貧。民多利器。國家滋昏。人多伎巧。奇物滋起。法令滋彰。盗賊多有。故聖人云。我無爲而民自化。我好靜而民自正。我無事而民自富。我無欲而民自樸。
    
    其政悶悶。其民淳淳。其政察察。其民缺缺。禍兮福之所倚。福兮禍之所伏。孰知其極。其無正。正復爲奇。善復爲妖。人之迷。其日固久。是以聖人方而不割。廉而不劌。直而不肆。光而不燿。
    
    治人事天莫若嗇。夫唯嗇。是謂早服。早服謂之重積德。重積德則無不克。無不克則莫知其極。莫知其極可以有國。有國之母。可以長久。是謂深根固柢。長生久視之道。

    太上。下知有之。其次。親而譽之。其次。畏之。其次。侮之。信不足。焉有不信焉。悠兮其貴言。功成事遂。百姓皆謂我自然。
    
    大道廢。有仁義。慧智出。有大僞。六親不和。有孝慈。國家昏亂。有忠臣。
    
    絶聖棄智。民利百倍。絶仁棄義。民復孝慈。絶巧棄利。盗賊無有。此三者。以爲文不足。故令有所屬。見素抱樸。少私寡欲。

    爲學日益。爲道日損。損之又損。以至於無爲。無爲而無不爲。取天下常以無事。及其有事。不足以取天下。

    其安易持。其未兆易謀。其脆易泮。其微易散。爲之於未有。治之於未亂。合抱之木。生於毫末。九層之臺。起於累土。千里之行。始於足下。爲者敗之。執者失之。是以聖人無爲。故無敗。無執。故無失。民之從事。常於幾成而敗之。慎終如始。則無敗事。是以聖人欲不欲。不貴難得之貨。學不學。復衆人之所過。以輔萬物之自然。而不敢爲。
    
    古之善爲道者。非以明民。將以愚之。民之難治。以其智多。故以智治國。國之賊。不以智治國。國之福。知此兩者亦稽式。常知稽式。是謂玄德。玄德深矣。遠矣。與物反矣。然後乃至大順。
    
    江海所以能爲百谷王者。以其善下之。故能爲百谷王。是以欲上民。必以言下之。欲先民。必以身後之。是以聖人處上而民不重。處前而民不害。是以天下樂推而不厭。以其不爭。故天下莫能與之爭。
    
    天下皆謂我道大。似不肖。夫唯大。故似不肖。若肖。久矣其細也夫。我有三寶。持而保之。一曰慈。二曰儉。三曰不敢爲天下先。慈。故能勇。儉。故能廣。不敢爲天下先。故能成器長。今舍慈且勇。舍儉且廣。舍後且先。死矣。夫慈。以戰則勝。以守則固。天將救之。以慈衛之。
    
    不尚賢。使民不爭。不貴難得之貨。使民不爲盗。不見可欲。使民心不亂。是以聖人之治。虚其心。實其腹。弱其志。强其骨。常使民無知無欲。使夫智者不敢爲也。爲無爲。則無不治。

    聖人無常心。以百姓心爲心。善者。吾善之。不善者。吾亦善之。德善。信者吾信之。不信者吾亦信之。德信。聖人在天下歙歙。爲天下渾其心。聖人皆孩之。

    和大怨。必有餘怨。安可以爲善。是以聖人執左契。而不責於人。有德司契。無德司徹。天道無親。常與善人。
    
    善行無轍迹。善言無瑕讁。善數不用籌策。善閉無關楗而不可開。善結無繩約而不可解。是以聖人常善救人。故無棄人。常善救物。故無棄物。是謂襲明。故善人者。不善人之師。不善人者。善人之資。不貴其師。不愛其資。雖智大迷。是謂要妙。
    
    將欲取天下而爲之。吾見其不得已。天下神器。不可爲也。爲者敗之。執者失之。故物或行或隨。或歔或吹。或强或羸。或挫或隳。是以聖人去甚。去奢。去泰。

    使我介然有知。行於大道。唯施是畏。大道甚夷。而民好徑。朝甚除。田甚蕪。倉甚虚。服文綵。帶利劍。厭飲食財貨有餘。是謂盗夸。非道也哉。
    
    以道佐人主者。不以兵强天下。其事好還。師之所處。荆棘生焉。大軍之後。必有凶年。善有果而已。不敢以取强。果而勿矜。果而勿伐。果而勿驕。果而不得已。果而勿强。物壯則老。是謂不道。不道早已。
    
    夫佳兵者。不祥之器。物或惡之。故有道者不處。君子居則貴左。用兵則貴右。兵者。不祥之器。非君子之器。不得已而用之。恬淡爲上。勝而不美。而美之者。是樂殺人。夫樂殺人者。則不可以得志於天下矣。吉事尚左。凶事尚右。偏將軍居左。上將軍居右。言以喪禮處之。殺人之衆。以哀悲泣之。戰勝。以喪禮處之。
    
    天下有道。卻走馬以糞。天下無道。戎馬生於郊。禍莫大於不知足。咎莫大於欲得。故知足之足。常足矣。
    
    出生入死。生之徒十有三。死之徒十有三。人之生動之死地。亦十有三。夫何故。以其生生之厚。蓋聞善攝生者。陸行不遇兕虎。入軍不被甲兵。兕無所投其角。虎無所措其爪。兵無所容其刃。夫何故。以其無死地。
    
    善爲士者不武。善戰者不怒。善勝敵者不與。善用人者爲之下。是謂不爭之德。是謂用人之力。是謂配天古之極。
    
    用兵有言。吾不敢爲主而爲客。不敢進寸而退尺。是謂行無行。攘無臂。扔無敵。執無兵。禍莫大於輕敵。輕敵幾喪吾寶。故抗兵相加。哀者勝矣。

    民不畏威。則大威至。無狎其所居。無厭其所生。夫唯不厭。是以不厭。是以聖人自知。不自見。自愛。不自貴。故去彼取此。
    
    勇於敢則殺。勇於不敢則活。此兩者。或利或害。天之所惡。孰知其故。是以聖人猶難之。天之道。不爭而善勝。不言而善應。不召而自來。繟然而善謀。天網恢恢。疏而不失。
    
    民不畏死。奈何以死懼之。若使民常畏死。而爲奇者吾得執而殺之。孰敢。常有司殺者殺。夫代司殺者殺。是謂代大匠斲。夫代大匠斲者。希有不傷其手矣。
    
    民之饑。以其上食税之多。是以饑。民之難治。以其上之有爲。是以難治。民之輕死。以其求生之厚。是以輕死。夫唯無以生爲者。是賢於貴生。

    爲無爲。事無事。味無味。大小多少。報怨以德。圖難於其易。爲大於其細。天下難事必作於易。天下大事必作於細。是以聖人終不爲大。故能成其大。夫輕諾必寡信。多易必多難。是以聖人猶難之。故終無難矣。
    
    大國者下流。天下之交。天下之牝。牝常以靜勝牡。以靜爲下。故大國以下小國。則取小國。小國以下大國。則取大國。故或下以取。或下而取。大國不過欲兼畜人。小國不過欲入事人。夫兩者各得其所欲。大者宜爲下。
    
    小國寡民。使有什伯之器而不用。使民重死而不遠徙。雖有舟輿。無所乘之。雖有甲兵。無所陳之。使人復結繩而用之。甘其食。美其服。安其居。樂其俗。鄰國相望。鷄犬之聲相聞。民至老死不相往來。
    
    吾言甚易知。甚易行。天下莫能知。莫能行。言有宗。事有君。夫唯無知。是以不我知。知我者希。則我者貴。是以聖人被褐懷玉。
    
\end{document}
